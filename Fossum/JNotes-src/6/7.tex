\begin{minipage}[t]{\sw}
\slidenumber
\LARGE
{\bf Language INFIX} (continued)\exx
We borrow the syntax of \verb'letrec' expressions in Language V5
to create a new syntactic category called a \verb'block',
with the following grammar rules:
{\Large
\begin{qv}
<factor>:BlockFactor ::= <block>
<block>              ::= LBRACE <blockDecls> <exp> <exps> RBRACE
<blockDecls>         **= DEF <VAR> EQUALS <exp> SEMI
<exps>               **= SEMI <exp>
\end{qv}
}
As shown in the \verb'INFIX' code file,
the evaluation semantics of a \verb'block' is
exactly the same as for a \verb'letrec' in Language V5,
except that the body of a \verb'block' allows 
for multiple (semicolon-separated) expressions 
as in a sequence expression defined in Language V5.
Here is an example that evaluates to 8.
{\Large
\begin{qv}
{
  def x=3;
  def y=5;
  19 ; x+y % the value of 19 is discarded
};
\end{qv}
}
We use semicolons in the syntax of a block to terminate
each of the block's variable definitions (using \verb'def').
As shown on the next slide,
\verb'block' is also used to define the body of a procedure.
\end{minipage}
