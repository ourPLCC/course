\begin{minipage}[t]{\sw}
\slidenumber
\LARGE
{\bf Language INFIX} (continued)\exx
Since the variable bindings in a block create an extended environment
using the \verb'letrec' strategy,
the RHS expressions in a \verb'def'
can refer to variables defined in the same block,
and procedures can be self-recursive.\exx
A \verb'proc' definition in \verb'INIFX' is similar to \verb'V5',
except that the body of the procedure is a block instead of an \verb'exp',
meaning that it can have its own ``local'' variables using \verb'def's.
Here's an example:
{\Large
\begin{qv}
proc(x) {def y=3; def z=add1(y); x+y+z}
\end{qv}
}
A procedure application in \verb'INIFX' is expressed
in exactly the same way as in \verb'V4'.
The following program evaluates to 120:
{\Large
\begin{qv}
{
  def f = proc(x)
    {if x then x*.f(sub1(x)) else one endif};
  def one = 1;
  .f(5)
} ;
\end{qv}
}
Observe that body of the the procedure \verb'f' can refer
to \verb'f' recursively,
and can even refer to the free variable \verb'one'
before it is defined in the same block,
since the semantics of the outer \verb'block'
behave as in \verb'letrec'.
\end{minipage}
