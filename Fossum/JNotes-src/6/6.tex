\begin{minipage}[t]{\sw}
\slidenumber
\LARGE
{\bf Language INFIX} (continued)\exx
One final problem with using infix notation for arithmetic expressions
is that there is nothing specific to mark the end of the expression.
With prefix notation, the end of a primitive application is always
a right parenthesis,
but with infix notation, there is nothing similar.
In most cases, it's easy to identify the end of an expression.
Consider, for example, the following:
\begin{qv}
if sub1(x) then x+3 else x+4
\end{qv}
The end of the expression \verb'sub1(x)'
is marked by the token \verb'then',
and the end of the expression \verb'x+3'
is marked by the token  \verb'else',
since \verb'then' and \verb'else' cannot appear
after a \verb'term' or a \verb'factor'.
But the final \verb'x+4' might have additional terms or factors
that do not appear on the same line.
To fix this, we add an \verb'endif' at the end of \verb'if'
expressions.\exx
We use the special token \verb';' to mark the the end of a program.
Here's an example of a complete program in the language \verb'INFIX':
\begin{qv}
if sub1(x) then x+3 else x+4 endif ;
\end{qv}
\end{minipage}
