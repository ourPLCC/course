\begin{minipage}[t]{\sw}
\slidenumber
\LARGE
{\bf Language INFIX} (continued)\exx
A related problem is called {\em precedence},
illustrated by the expression
\begin{qv}
1+2*3
\end{qv}
If the \verb'plcc' tool chose left associativity
(which is what it might have done, correctly, in the previous example),
this would be interpreted as \verb'arg1' being \verb'1+2'
with \verb'arg2' being \verb'3',
but then the result would be interpreted as 9,
whereas the correct (mathematical) interpretation would be 7.
The problem is that in mathematifcal infix notation,
multiplication has a higher precedence than addition.
(Note that virtually all programming languages that use infix notation
use the mathematical interpretation of these kinds of expressions.)\exx
We correct this by introducing grammar rules
that make it easy to implement semantics for associativity and precedence.
Our grammar rules will also permit using parentheses
to treat a group of terms as a single semantic unit,
so that
\begin{qv}
(1+2)*3
\end{qv}
evaluates to 9.
\end{minipage}
