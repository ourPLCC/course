\begin{minipage}[t]{\sw}
\slidenumber
\LARGE
{\bf Language TYPE0} (continued)\exx
\Large
We can add a number of new primitives applied to integer arguments
that return \verb'BoolVal' values,
such as the following, with obvious meanings:
\large
\begin{qv}
<?
<=?
>?
>=?
=?
<>?
\end{qv}
\Large
The purpose of the \verb'?' at the end of these
is to remind you that they are testing something
and that they return boolean values
(they are {\em predicates}).
We also change the value returned by the \verb'zero?' primitive
to a \verb'BoolVal'.\exx
The following expressions evaluate as indicated:
\large
\begin{qv}
<?(3,3)
  % => false (actually, a false BoolVal)
<=?(3,3)
  % => true
<>?(x,y)
  % => depends on the current bindings of x and y
=?(proc(t) t, proc(u) u)
  % => exception -- can't apply to procedures
if 1 then 2 else 3
  % => exception -- 1 is not a boolean
\end{qv}
\Large
\end{minipage}
