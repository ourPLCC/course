\begin{minipage}[t]{\sw}
\slidenumber
\LARGE
{\bf Language TYPE1} (continued)\exx
Although primitives are not procedures
(you can't bind a variable to \verb'add1', for example),
they do have specific type behaviors
that can be described in terms of procedure types.
The \verb'+' primitive behaves like \verb'[int,int=>int]',
and the \verb'add1' primitive behaves like \verb'[int=>int]'.
Each of the primitives are associated with a \verb'ProcType'
that is used for type checking.
We define these types in the \verb'Type' class
as static instance variables.\exx
Rather than building each of these types by hand with constructors,
we define and use a special static method \verb'compile'
in the \verb'Type' class that takes a compact string representation
of a procedure type such as \verb'[int,int=>int]'
(in this case, the string is \verb'"ii>i"')
and returns a \verb'ProcType' object
with the proper formal parameter types and result type.
You can see how the \verb'compile' method is used
in the following example definitions
and their corresponding interpretations:
{\Large
\begin{qv}
public static ProcType ii_i = compile("ii>i"); // [int,int=>int]
public static ProcType i_i  = compile("i>i");  // [int=>int]
public static ProcType ii_b = compile("ii>b"); // [int,int=>bool]
public static ProcType bb_b = compile("bb>b"); // [bool,bool=>bool]
\end{qv}
}
\end{minipage}
