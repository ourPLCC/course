\begin{minipage}[t]{\sw}
\slidenumber
\LARGE
{\bf Language TYPE1} (continued)\exx
Our approach is to evaluate the type (\verb'Type') of an expression,
doing type checks of sub-expressions along the way.
If there are no type errors,
we can then evaluate the value (\verb'Val') of the expression
and return this value as we have done before.\exx
First, when we encounter a variable in an expression,
we want to determine the type of that variable.
If the variable is a formal parameter in the body of a procedure,
its type is determined by the type declaration of the formal parameter.
For any other variable,
its type can be determined using static scope rules.
This suggests that we have a way of binding the variable to its type
during type checking
in a way that is similar to binding the variable to its value
during expression evaluation.
We achieve this by creating a {\em type environment}
that is almost identical to a value environment
that we have been using so far,
except that our type environment binds variables
to types rather than to values.\exx
The appropriate classes are shown on the next slide.
\end{minipage}
