\begin{minipage}[t]{\sw}
\slidenumber
\LARGE
{\bf Language TYPE1} (continued)\exx
\Large
Here is the definition of \verb'checkEquals' in the \verb'ProcType' class:
\large
\begin{qv}
public void checkEquals(Type t) {
    t.checkProcType(this);
}

// check to see if the type of the ProcType object t is the same
// as this ProcType object
public void checkProcType(ProcType t) {
    // first check the return types
    this.returnType.checkEquals(t.returnType);
    // then check the types of the formal parameters
    checkEqualTypes(this.paramTypeList, t.paramTypeList);
}
\end{qv}
\Large
The \verb'checkEqualTypes' static method
in the \verb'Type' class is straight-forward:
first check for equal sizes;
then iterate through both of the lists,
checking pairwise for type equality.
\large
\begin{qv}
public static void checkEqualTypes(List<Type> t1List, List<Type> t2List) {
    if (t1List.size() != t2List.size())
        throw new PLCCException("Type error", "argument number mismatch");
    Iterator<Type> t1i = t1List.iterator();
    Iterator<Type> t2i = t2List.iterator();
    while (t1i.hasNext()) {
        Type t1 = t1i.next();
        Type t2 = t2i.next();
        t1.checkEquals(t2);
    }
}
\end{qv}
\end{minipage}
