\begin{minipage}[t]{\sw}
\slidenumber
\LARGE
{\bf PLCC -- Syntax (continued)}\exx
If the LHS is a simple nonterminal,
the Java class name associated with the BNF rule is the nonterminal name
with its first letter converted to uppercase.
In this example
{\Large
\begin{qv}
<proc> ::= PROC LPAREN <formals> RPAREN <exp>
\end{qv}
}
the Java class name is \verb'Proc'.\exx
If the LHS is a nonterminal annotated
by adding a colon and a Java class name,
then the class name associated with the BNF rule
is the annotated Java class name.
In this case, an abstract base class is also created
whose Java class name is the nonterminal name
with its first letter converted to uppercase (as described above),
with the annotated Java class name as a subclass.
In this example
{\Large
\begin{qv}
<exp>:AppExp ::= DOT <exp> LPAREN <rands> RPAREN
\end{qv}
}
the Java class name associated with this BNF rule is \verb'AppExp',
and this class extends the base class \verb'Exp'.
\end{minipage}
\clearpage
