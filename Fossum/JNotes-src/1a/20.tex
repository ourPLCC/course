\begin{minipage}[t]{\sw}
\slidenumber
\LARGE
{\bf PLCC -- Semantics (continued)}\exx
For an instance of class \verb'Interior',
an appropriate \verb'toString' method can be:
{\Large
\begin{qv}
public String toString() {
    return "("+symbol.toString()+" "+left+" "+right+")";
}
\end{qv}
}
This relies on the proper (recursive) \verb'toString' behavior
of the \verb'left' and \verb'right' fields,
both of which are defined as instances of the \verb'Tree' class.\exx
Recall that PLCC generates a Java class
for each of the BNF grammar rules
given in the syntax section of the language specification.
We can define {\em semantic actions} for these classes
by adding entries to the semantics section 
of the language specification file having the form
{\Large
\begin{qv}
ClassName
%%%
...
%%%
\end{qv}
}
where \verb'ClassName' stands for the name of a PLCC-generated class name,
such as \verb'Tree' in Language TREE.
PLCC inserts the lines of code bracketed by the '\verb'%%%'' lines
verbatim into the \verb'ClassName.java' file.
Most often, these lines define one or more Java methods
that can be applied to instances of the class.
\end{minipage}
\clearpage
