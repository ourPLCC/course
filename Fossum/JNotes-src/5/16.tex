\begin{minipage}[t]{\sw}
\slidenumber
\LARGE
{\bf Language OBJ} (continued)\exx
... continued from previous slide ...
{\Large
\begin{qv}
    // create the object
    ObjectVal objectVal = new ObjectVal(env);

    // bind 'super' field to the parent object
    fieldBindings.add("super", new ValRef(parent)); // parent object
    // bind 'self' field to the base object being created
    // (to speed up lookups)
    fieldBindings.add("self", objRef); // deep
    // bind 'this' field to this object environment
    fieldBindings.add("this", new ValRef(objectVal)); // shallow
    return objectVal;
}
\end{qv}
}
Observe that this code binds \verb'self' to \verb'objRef' in every set
of field bindings created recursively
up to the top-level \verb'EnvClass' class.
This is not necessary, since the top-level \verb'EnvClass' object
is guaranteed to have a field binding of \verb'self' to \verb'objRef',
so any reference to \verb'self' will eventually be found
in the chain of environments.
However, by putting the binding in every intermediate object,
any reference to \verb'self' will be found sooner
in \verb'applyEnv' lookups.
\end{minipage}
