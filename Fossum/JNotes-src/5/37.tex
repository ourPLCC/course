\begin{minipage}[t]{\sw}
\slidenumber
\LARGE
{\bf Language OBJ} (continued)\exx
In the previous example,
the \verb'init' method is invoked separately,
after the object is created using \verb'new',
and not as part of the object creation itself.
Naming this procedure ``\verb'init''' is not a requirement.
A class can have several methods that initialize its fields,
much as a \verb'Java' class can have several constructors.
Unlike \verb'Java', the \verb'OBJ' language can apply its methods --
even the ones intended to initialize the fields --
at any time.
\begin{qv}
define c =
    class
        field x
        method init = proc() {set x = 5 ; self}
        method foo = proc() {set x = add1(x) ; self}
    end

define o = .<new c>init()
<o>x              % => 5
<o>{.foo() ; x}   % => 6
<o>{.init() ; x } % => 5
\end{qv}
\end{minipage}
