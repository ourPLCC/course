\begin{minipage}[t]{\sw}
\slidenumber
\LARGE
{\bf Language OBJ} (continued)\exx
As each object is created up the superclass chain,
we insert three predefined fields
into its list of field bindings:
\verb'this', \verb'self', and \verb'super'.
We bind the field identifier \verb'self'
to the base object being created,
which is the same as described above.
We bind the field identifier \verb'this'
to the object being created at the particular point
in the superclass chain (we call this a {\em shallow} binding).
And we bind the field identifier \verb'super'
to the superclass object.
The code for creating ghese bindings is shown on Slide 5.16.\exx
We disallow duplicate field names in class definitions.
We also disallow field names that duplicate the predefined identifiers
\verb'this', \verb'self', and \verb'super'.\exx
Notice that an object can see all
of the \verb'static' bindings up the superclass chain,
but that if a \verb'static' variable is bound to a procedure
(or some other value that captures an environment),
the procedure captures only the static environment of the class
and cannot ``see'' any of the fields or methods --
including the \verb'self' and \verb'this' identifiers --
in its environment.
\end{minipage}
