\begin{minipage}[t]{\sw}
\slidenumber
\LARGE
{\bf Language V4} (continued)\exx
In general, a \verb'let' expression
{\Large
\begin{qv}
let
  v1 = e1
  v2 = e2
  ...
in
  e
\end{qv}
}
can be re-written as an equivalent procedure application expression
{\Large
\begin{qv}
.proc(v1, v2, ...) e (e1, e2, ...)
\end{qv}
}
So why not ditch the \verb'let' construct?
The reason is simple:
it's easier to think about a program with a \verb'let' in it
than one without.
The \verb'let' construct aligns
the LHS variables \verb'v1', \verb'v2', {\em etc.}
physically close to their corresponding RHS expressions
\verb'e1', \verb'e2', {\em etc.},
so it's cognitively easy for the reader
to see how these LHS variables become bound
to the values of their RHS expressions.
In the equivalent procedure application,
the formal parameters \verb'v1', \verb'v2', {\em etc.}
are physically distant from their corresponding RHS expressions,
making it difficult to visualize these bindings.\exx
This is an example of {\em syntactic sugar}:
a syntactic and semantic construct
that has another, equivalent way of expressing it in the language
but that programmers find easier to read, understand, and use.
\end{minipage}
\clearpage
