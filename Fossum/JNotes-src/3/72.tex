\begin{minipage}[t]{\sw}
\slidenumber
\LARGE
{\bf Drawing Envinronments} (continued)\exx
In the definition of the \verb'ProcVal' class,
a \verb'ProcVal' object has three fields:
\Large
\begin{qv}
public Formals formals; // list of formal parameters
public Exp body;        // procedure body
public Env env;         // captured environment
\end{qv}
\LARGE
Here, the {\em captured environnment} is the environment
in which the procedure is defined.
For example, consider the following expression:\\[1.5ex]
\Large
\begin{verbbox}
let
  x = 3
in
  proc(t) +(t,x)
\end{verbbox}
\emm\theverbbox\\[1.5ex]
\LARGE
This expression evaluates to a \verb'ProcVal':
its \verb'formals' field consists of a list having a single string, `\verb't'',
its \verb'body' is the expression `\verb'+(t,x)'',
and its captured environment is the one defined by the \verb'let',
having a single binding of \verb'x' to the value \verb'3'.
(Recall that we also use the term {\em closure}
to refer to a a \verb'ProcVal' object.)
We normally display a \verb'ProcVal' object as a rectangle
with its three compartments, in this order:
formals, body, and captured environment.
We show the \verb'formals' as a comma-separated list of identifiers,
the \verb'body' as an (un-evaluated) expression,
and the captured environment (\verb'env') as an arrow
pointing to the appropriate node in the environment
in which the procedure definition occurs.
The following slide shows the \verb'ProcVal' that results
from the evaluation of the above expression.
\end{minipage}
\clearpage
