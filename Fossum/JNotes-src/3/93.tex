\begin{minipage}[t]{\sw}
\slidenumber
\LARGE
{\bf Language V6} (continued)\exx
Here is an example of expressions
that use the \verb'define' feature
in our source language:
\Large
\begin{qv}
define i = 1
define ii = add1(i)
define iii = add1(ii)
define v = 5
define x = 10
define f = proc(x) if zero?(x) then 1 else *(x,.f(.g(x)))
.f(v)   % ERROR: g is unbound
define g = proc(x) sub1(x)
.f(v)   % => 120 -- g is now bound
.f(iii) % => 6
\end{qv}
\LARGE
As long as you stay in the \verb'Rep' loop,
the \verb'define'd variable bindings are remembered.\exx
Notice that, in the definition for \verb'f',
the body of the procedure refers to a procedure named \verb'g',
but \verb'g' hasn't been defined yet.
The attempt, in the next line, to evaluate \verb'.f(v)' fails.
After defining \verb'g' on the following line,
evaluating \verb'.f(v)' works.
This is because
by the time you attempt to apply \verb'f' the second time,
the \verb'g' procedure has been defined,
and the body of \verb'f' now recognizes its definition.
\end{minipage}
\clearpage
