\begin{minipage}[t]{\sw}
\slidenumber
\LARGE
{\bf Language V0} (continued)\exx
Assuming that we have created the \verb'grammar' file
in a directory named \verb'V0',
running the \verb'plccmk' tool creates a \verb'Java' subdirectory
with source files named \verb'Program.java',
\verb'LitExp.java', and so forth,
that correspond to the abstract syntax classes shown in Slide 3.7.
In the \verb'Java' directory,
you can also see Java source files named \verb'Token.java',
\verb'Scan.java',
\verb'Parse.java',
and \verb'Rep.java'.\exx
The \verb'Rep' program
repeatedly prompts you for input (with `\verb'-->''),
parses the input,
and prints the result --
again, a \verb'String' representation of the parse tree.
If you want to run this program
from the directory that has the \verb'grammar' file --
\verb'V0' in this case --
you can run it as follows:
{\Large
\begin{qv}
$ java -cp Java Rep
--> add1( + (2,3))
...
\end{qv}
}
As we discussed in Chapter 1,
parsing is the process by which a sequence of tokens
(a {\em program}) can be determined
to belong to the language defined by the grammar.
We showed examples of leftmost derivations
and how the derivation process can detect whether or not
the program is syntactically correct.

\end{minipage}
\clearpage
