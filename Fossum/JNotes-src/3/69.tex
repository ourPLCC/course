\begin{minipage}[t]{\sw}
\slidenumber
\LARGE
{\bf Drawing Envinronments} (continued)\exx
Now consider the following expression,
with nested \verb'let's.
The inner \verb'let' extends the environment defined
by the outer let (with one node, as shown on the previous page),
so the envinroment of the inner \verb'let' is a linked
list with two nodes.
\Large
\begin{qv}
let
  x = 3
  y = 5
in
  let
    x = +(x,y) % the RHS evaluates to 8
    z = x      % the RHS evaluates to 3 (why?)
  in
    +(x,y)
\end{qv}
\LARGE
In the following diagram,
the leftmost node is the environment created by the inner \verb'let':\\
\Large
\centerline{\psfig{figure=xyz.eps,height=2in}}
\LARGE
The inner \verb'let' body expression evaluates to 13 (why?).
\end{minipage}
\clearpage
