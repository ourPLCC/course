\begin{minipage}[t]{\sw}
\slidenumber
\LARGE
{\bf Language V0} (continued)\exx
The \verb'grammar' file in Language V0
has three parts, separated by lines with a single `\verb'%'':
the {\bf lexical specification section},
the {\bf syntax section},
and the {\bf semantics (code) section}.\exx
(Recall that if your \verb'grammar' file has only the lexical specification,
the \verb'plccmk' tool produces Java code
for a scanner (\verb'Scan'),
but nothing else.
If your \verb'grammar' file has only the lexical specification and syntax rules,
the \verb'plccmk' tool produces Java code 
for a scanner (\verb'Scan') and a parser (\verb'Parse') for the grammar,
but nothing else.)\exx
The code section is the heart of the language semantics.
In this section, the Java classes defined by the grammar rules
are given life by defining their behavior --
specifically, by defining the \verb'$run()' method
in the start symbol class.
We will presently see how such a \verb'$run()' method can be used
to print the arithmetic value of an expression,
but for now we are content with simply printing
a copy of the expression itself.\exx
{\em The code section of Language V0 defines the language semantics.}
\end{minipage}
\clearpage
