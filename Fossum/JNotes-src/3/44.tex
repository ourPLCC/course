\begin{minipage}[t]{\sw}
\slidenumber
\LARGE
{\bf Language V3} (continued)\exx
\Large
\emm\VerbBox{\colorbox{light}}{\begin{tabular}{@{}ll}
\verb'<exp>:LetExp' & \verb'::= LET <letDecls> IN <exp>'\\
  & \VerbBox{\fbox}{\verb'LetExp(LetDecls letDecls, Exp exp)'}\\
\verb'<letDecls>' & \verb'**= <VAR> EQUALS <exp>'\\
  & \VerbBox{\fbox}{\verb'LetDecls(List<Token> varList, List<Exp> expList)'}\\
\end{tabular}}\exx
\LARGE
To evaluate a \verb'LetExp', we perform the following steps:
\begin{enumerate}
\item
  create a set of local bindings
  by binding each of the \verb'<VAR>' symbols
  to the values of their corresponding \verb'<exp>' expressions
  in the \verb'<letDecls>' part,
  where the \verb'<exp>' expressions to the right
  of the \verb'EQUALS' are all evaluated
  in the enclosing environment;
\item
  extend the enclosing environment with these local bindings
  to create a new environment; and
\item
  use this new environment to evaluate the \verb'<exp>' expression
  in the \verb'LetExp', and return this value
  as the value of the \verb'letExp' expression.
\end{enumerate}
The \verb'<exp>' part of a \verb'let' expression
is called the {\em body} of the \verb'let' expression.\exx
Some examples if \verb'let' expressions are on the next slide,
where \verb'=>' means ``evaluates to''.
Remember: {\em a \verb'let' expression is an expression,
and as such, it evaluates to something}!
\end{minipage}
\clearpage
