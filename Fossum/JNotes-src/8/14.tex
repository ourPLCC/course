\begin{minipage}[t]{\sw}
\slidenumber
\LARGE
{\bf Continuations} (continued)\exx
A \verb'letrec' simply creates a new environment
with bindings of identifiers to \verb'ProcVal's.
Moreover, evaluating a \verb'proc'
(don't confuse this with {\em applying} a \verb'proc')
requires no more than gathering together
the formal parameter list, the procedure body expression,
and the captured environment.
Thus the environment in which we evaluate the \verb'letrec' body
is obtained by calling the \verb'addBindings' method
in the \verb'LetrecDecls' class,
unchanged from the \verb'REF' language.
Recall that \verb'addBindings' returns an environment
that extends the given environment
by binding the LHS variables
to (references to) their corresponding RHS \verb'ProcVal's.
The \verb'eval' method then returns a \verb'EvalCont' continuation
that evaluates the body of the \verb'letrec'
in this extended environment
and passes its value to the \verb'vcont' continuation.
Here is the code for \verb'eval' in the \verb'LetrecExp' class:
\Large
\begin{qv}
LetrecExp
%%%
    public ACont eval(Env env, VCont vcont) {
        Env nenv = letrecDecls.addBindings(env);
        return new EvalCont(exp, nenv, vcont);
    }
%%%
\end{qv}
\LARGE
Notice that we don't call the \verb'vcont' \verb'apply' method directly here,
since we expect that the \verb'EvalCont' continuation will ultimately do so
when it jumps onto the trampoline.
\end{minipage}
