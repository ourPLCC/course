\begin{minipage}[t]{\sw}
\slidenumber
\LARGE
{\bf Tokens} (continued)\exx
Writing a scanner is somewhat involved,
so we have provided you with a tool set
that produces a Java scanner automatically
from a file that specifies the tokens using regular expressions.
This tool set, named \verb'PLCC',
consists of a program written in Python 3
along with some support files.
\verb'PLCC' stands for a ``Programming Languages Compiler Compiler''.
You should be able to use this tool set with any system
that supports Python 3 and Java.
The \verb'plcc.py' Python program
and the \verb'Std' subdirectory that contains its support files
are on the RIT Ubuntu lab systems in this directory:
\begin{qv}
/usr/local/pub/plcc/src
\end{qv}
This directory also contains a shell script called \verb'plcc'
that runs the \verb'plcc.py' Python program,
along with a script called \verb'plccmk'
that also compiles the Java programs created by PLCC.
When you are working on one of our Ubuntu lab systems,
you can simply run \verb'plcc' or \verb'plccmk'
to process the various languages we will specify in this course.\exx
See the \verb'HOWTO.html' file in \verb'/usr/local/pub/plcc/tvf'
for {\em important information} about how
to set up your CS account environment
so that PLCC will be able to access the required program files
and library routines on CS servers and lab workstations.
This file also explains how you can use \verb'github'
to set up PLCC on a local Windows or Linux computer system.
If you do this, you are on your own.
\end{minipage}
\clearpage
