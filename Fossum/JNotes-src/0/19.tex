\begin{minipage}[t]{\sw}
\slidenumber
\Large
{\bf Tokens} (continued)\exx
After running the \verb'plccmk' command,
a \verb'Java' subdirectory is created,
populated with the following Java source files
{\Large
\begin{qv}
Token.java
Scan.java
\end{qv}
}
as well as a few other Java support files.
The \verb'plccmk' command also compiles these source files.
When you run
{\Large
\begin{qv}
java -cp Java Scan
\end{qv}
}
you can enter strings from your terminal
and see what tokens are recognized by the scanner.\exx
Examine the \verb'Token.java' file to see
how the token specifications in your \verb'grammar' file are translated
into Java code that associates
the token names with their corresponding regular expression patterns,
and similarly for skip patterns.\exx
When specifying tokens in a \verb'grammar' file,
you can omit the \verb'token' term (but not the \verb'skip' term).
This means that both
{\Large
\begin{qv}
WORD '\S+'
\end{qv}
}
and
{\Large
\begin{qv}
token WORD '\S+'
\end{qv}
}
are considered as equivalent.
We follow this convention in all of our subsequent examples.\exx
\end{minipage}
\clearpage
