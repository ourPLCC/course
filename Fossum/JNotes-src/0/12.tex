\begin{minipage}[t]{\sw}
\slidenumber
\LARGE
{\bf Tokens} (continued)\exx
The lexical specification section of a \verb'grammar' file
uses {\em regular expressions} to specify language tokens.
A regular expression is a formal description of a pattern
that can match a sequence of characters in a character stream.
For example, the regular expression `\verb'd'' matches the letter \verb'd',
the regular expression `\verb'\d'' matches any decimal digit,
and the regular expression `\verb'\d+'' matches one or more decimal digits.
{\bf You should read the Java documentation for the \verb'Pattern' class
for information about how to write regular expressions.}\exx
When specifying tokens, we must identify
what input stream characters do {\em not} belong to tokens
and should be skipped.
Typically, we skip whitespace: spaces, tabs, and newlines.
We identify these skipped characters in the \verb'grammar' file
using a {\em skip specification} line like this :
\begin{qv}
skip WHITESPACE '\s+'
\end{qv}
The regular expression `\verb'\s'' stands for ``space''
(the space character, a tab, or a newline),
and the regular expression `\verb'\s+'' stands for one or more spaces.
We use the symbolic name ``WHITESPACE''
to identify this particular collection of characters to be skipped.
(Any symbolic name would suffice,
but it makes sense to use one that describes its purpose.)
We also typically skip comments.\exx
{\bf Characters to be skipped during lexical analysis are not tokens.}
\end{minipage}
\clearpage
