\begin{minipage}[t]{\sw}
\slidenumber
\LARGE
{\bf Tokens} (continued)\exx
When running \verb'plcc', you need to give the name
of the language specification file on the command line.
For example:
\begin{qv}
plcc grammar
\end{qv}
The \verb'grammar' file is a text file
that defines the tokens of the language
using skip specifications and token specifications
as we have illustrated earlier.
The filename \verb'grammar' is typically used,
but you can name your file anything you wish,
so that both of the following examples are acceptable:
\begin{qv}
plcc mygrammar
plcc foo
\end{qv}
The language specification file can contain
comments starting with a '\#' character
and continuing to the end of the line.
These comments are ignored by \verb'plcc'.\exx
The \verb'plccmk' script runs \verb'plcc'
on a language specification file
whose name must be \verb'grammar'.
In addition to creating a \verb'Java' subdirectory
and depositing the \verb'Scan.java' program in that directory
(along with a few other necessary Java support files),
\verb'plccmk' compiles the Java programs in that directory.
Our first \verb'grammar' files will contain only
token specifications.
Later, we will use \verb'grammar' files
to define language syntax, and then semantics.
For now, we concentrate only on token specifications.

\end{minipage}
\clearpage
