\begin{minipage}[t]{\sw}
\slidenumber
\LARGE
{\bf Language NEED} (continued)\exx
Here's another example illustrating the difference
between call-by-reference and call-by-name/need.
Examine the definition of \verb.seq., which seems
to recurse infinitely but doesn't with call-by-name (why?).
\begin{qv}
define pair = proc(x,y)
  proc(t) if t then y else x
define first = proc(p) .p(0)
define rest = proc(p) .p(1)
define nth = proc(n,lst)    % zero-based
  if n then .nth(sub1(n),.rest(lst)) else .first(lst)
define seq = proc(n) .pair(n,.seq(add1(n)))
define natno = .seq(0)      % all the natural numbers!!
%% The above never terminates with call-by-reference.
%% With call-by-name or call-by-need, we get:
.first(natno)               % => 0 
.first(.rest(natno))        % => 1
.first(.rest(.rest(natno))) % => 2, and so forth ...
.nth(100,natno)             % => 100
\end{qv}
\end{minipage}
\clearpage
