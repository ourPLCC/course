\begin{minipage}[t]{\sw}
\slidenumber
\LARGE
{\bf Order of evaluation} (continued)\exx
You can see that both \verb'evalRands' and \verb'evalRandsRef'
use \verb'for-each' loops (also called enhanced \verb'for' loops)
to traverse and evaluate the expressions in the list of actual parameters.
The traversal is guaranteed by the Java API specification to be ``natural'',
in the sense that the elements of the list
are visited in ascending item number order.
Here is the code for \verb'evalRands' in the \verb'Rands' class:
{\Large
\begin{qv}
public List<Val> evalRands(Env env) {
    List<Val> valList = new ArrayList<Val>();
    for (Exp e : expList)
        valList.add(e.eval(env));
    return valList;
}
\end{qv}
}
We are, of course, not under any obligation
to use Java's \verb'for-each' order of evaluation mechanism
to implement our \verb'evalRands' semantics.
We could, instead, explicitly traverse the \verb'expList'
from last to first if we wished.\exx
The point is that,
if we want to make our language semantics well-defined and unambiguous,
we need to specify the order of evaluation
in our language documentation.
Unless the language specification clearly addresses the issue
of order of evaluation, the language implementor can choose
any evaluation order. {\em Let the buyer beware!}
\end{minipage}
\clearpage
