\begin{minipage}[t]{\sw}
\slidenumber
\LARGE
{\bf Language REF} (continued)\exx
When actual parameter expressions are not themselves variables,
we use value semantics.
To illustrate this, consider the value returned by the following program:
\begin{qv}
let
  x = 3
  p = proc(t) set t = add1(t)
in
  { .p(+(x,0)) ; x }
\end{qv}
Clearly the expressed value
of the actual parameter \verb'+(x,0)'
is the same as that of \verb'x',
but the expression \verb'+(x,0)' is not a variable,
so value semantics apply to this actual parameter.
This means that when we apply the procedure \verb'p',
the formal parameter \verb't' denotes
a {\em new} reference to the value of this expression:
the variables \verb't' and \verb'x' have the same expressed values,
but they have different denoted values,
so modifying \verb't' does not affect the value of \verb'x'.
This expression evaluates to 3.
\end{minipage}
\clearpage
