\begin{minipage}[t]{\sw}
\slidenumber
\LARGE
{\bf Order of evaluation}\exx
\Large
Let's examine the following example:
\begin{qv}
let
  x = 3
in
  let
    y = {set x = add1(x)}
    z = {set x = add1(x)}
  in
    z
\end{qv}
Consider the inner \verb'let'.
We know that the right-hand side expressions
(here written inside curley braces for clarity)
are evaluated before their values are bound
to the left-hand variables.
But our language does not specify the order
in which the right-hand side expressions are evaluated.\exx
In the absence of side-effects,
{\em i.e.} in our early interpreters without \verb'set',
the order of evaluation of the RHS expressions wouldn't matter.
However, when side-effects are possible,
as in our interpreters such as \verb'SET' and \verb'REF',
the order of evaluation does matter.\exx
In the above example, if the second \verb'set' is evaluated first,
then \verb'z' becomes 4 and \verb'y' becomes 5,
so the entire expression evaluates to 4 -- the value of \verb'z'.
If the order of evaluation is reversed,
the entire expression evaluates to 5.
Our language does not specify order of evaluation;
consequently, the value of this expression is ambiguous.
\end{minipage}
\clearpage
