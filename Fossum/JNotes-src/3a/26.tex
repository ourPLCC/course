\begin{minipage}[t]{\sw}
\slidenumber
\LARGE
{\bf Language NAME} (continued)\\[1.5ex]
Consider the following definition:
\Large
\begin{qv}
define while = proc(test?, do, ans)
  letrec loop = proc()
    if test? then {do ; .loop()} else ans
  in .loop()
\end{qv}
\LARGE
Using call-by-name, the expression
\Large
\begin{qv}
let x = 0 sum = 0 in
  .while(
    <=?(x,10),
    { set sum=+(sum,*(x,x)) ; set x = add1(x) },
    sum
  )
\end{qv}
\LARGE
returns the sum
\[\sum_{x=1}^{10} x^2 = 385\]\\
Using call-by-reference as in the language \verb'REF',
the expression never terminates
because the actual parameter expression \verb'<=?(x,10)'
is evaluated only once, to 1 (true) when \verb'x' is initially 0,
and so the \verb'test?' parameter is bound permanently to (a reference to) 1.
Evaluating \verb'test?' repeatedly always returns 1 (true),
so the ``loop'' never terminates.
\end{minipage}
\clearpage
