\begin{minipage}[t]{\sw}
\slidenumber
\LARGE
{\bf Language NEED} (continued)\exx
The following example illustrates the difference
between call-by-name and call-by-need:
\begin{qv}
let
  x = 3
  p = proc(t) {t;t;t}
in
  .p(set x=add1(x))
\end{qv}
\LARGE
With call-by-name,
when we apply the procedure \verb'p',
its formal parameter \verb't' is bound to a thunk
containing the expression \VerbBox{\fbox}{\verb'set x=add1(x)'}.
Each time we evaluate the formal parameter \verb't'
in the body of the procedure \verb'p',
its thunk is dereferenced, resulting
in evaluation of the expression \VerbBox{\fbox}{\verb'set x=add1(x)'}.
So since we evaluate \verb't' three times in the body of \verb'p',
the expression \VerbBox{\fbox}{\verb'set x=add1(x)'}
gets evaluated three times, incrementing \verb'x' from 3 to 6.
Consequently, the entire expression evaluates to 6.\exx
With call-by-need,
the first time we evaluate \verb't' in the body of \verb'p',
its corresponding actual parameter expression
\VerbBox{\fbox}{\verb'set x=add1(x)'}
is evaluated,
which has the side-effect of incrementing the value of \verb'x' to 4
and evaluates to 4.
However, the thunk memoizes the expressed value of 4,
so any further references we make to \verb't' evaluate to 4.
Consequently, the entire expression evaluates to 4.
\end{minipage}
\clearpage
