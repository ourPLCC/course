\begin{minipage}[t]{\sw}
\slidenumber
\LARGE
{\bf Language REF} (continued)\exx
The other change is in the \verb'Rands' code.
In the \verb'SET' language,
the \verb'evalRands' method was used in
the implementation of \verb'eval'
for both a \verb'LetExp' object and an \verb'AppExp' object,
since both created new bindings to values.
In the \verb'REF' langauge, 
an \verb'AppExp' object needs new bindings to values
except for actual parameters which are variables --
a situation that is described in the previous slide.
Therefore, to implement the correct \verb'eval' semantics
for an \verb'AppExp' object,
we need to collect \verb'evalRef' references
instead of \verb'eval' values
to bind them to the formal parameters.
The method \verb'evalRandsRef' in the \verb'Rands' class
does this work for us.
The \verb'eval' method in the \verb'AppExp' class uses
the \verb'evalRandsRef' method
to create the bindings of the formal parameters
to their appropriate references.
The definition for \verb'evalRandsRef' follows:
{\Large
\begin{qv}
    public List<Ref> evalRandsRef(Env env) {
        List<Ref> refList = new ArrayList<Ref>(expList.size());
        for (Exp exp : expList)
            refList.add(exp.evalRef(env));
        return refList;
    }
\end{qv}
}
\end{minipage}
\clearpage
