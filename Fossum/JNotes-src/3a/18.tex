\begin{minipage}[t]{\sw}
\slidenumber
\LARGE
{\bf Language REF} (continued)\exx
The term {\em L-value} refers to an expression
that can be interpreted as a reference.
(It's called an L-value because it is the sort of expression
that can appear to the {\em left} of the \verb'=' in a \verb'set'.)
While a variable \verb'x' can always be considered as an L-value,
the expression \verb'+(x,0)' can only be interpreted as
a value, never a reference.
In Language REF, only variable expressions are L-values.\exx
In summary,
{\bf if an actual parameter is a L-value in Language REF,
(and therefore can be interpreted as a reference),
then the corresponding formal parameter is bound to the same reference.
If an actual parameter is something other than an L-value,
then the corresponding formal parameter is bound
to a {\em new} temporary reference
containing the value of the actual parameter.}
\end{minipage}
\clearpage
