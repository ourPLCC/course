\begin{minipage}[t]{\sw}
\slidenumber
\LARGE
{\bf Static Properties of Variables} (continued)\exx
Determing the value of an expression at run-time is at the heart
of executing a program, particularly so in expression-based languages.
Since most expressions involve variables,
evaluating an expression requires determining the values
of its constituent variables --
in other words, finding the values bound to these variables.\exx
At run-time, how can you find the value bound to a variable?
There are two basic approaches:
\begin{itemize}
\item
if the location of the value bound to a variable can be determined
by {\em where} that variable appears {\em in the text of a program},
we call it {\em static binding}.
\item
if the location of the value bound to a varible can only be determined
by {\em when} the variable is accessed {\em during program execution},
we call it {\em dynamic binding}.
\end{itemize}
Almost all programming languages commonly in use today
use static bindings,
principally because it is easier
to reason (or prove things) about programs that use static bindings.
You will have the opportunity
to explore dynamic binding in your homework.\exx
\end{minipage}
\clearpage
