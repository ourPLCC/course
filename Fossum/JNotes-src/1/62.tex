\begin{minipage}[t]{\sw}
\slidenumber
\LARGE
{\bf Static Properties of Variables} (continued)\exx
Most imperative programming languages are {\em block structured}
and use {\em lexical binding}, another term for static scope rules.
A {\em block} is a region of code
introduced by one or more variable declarations
and continuing to the end of the code
where these declarations are active.
In C, C++, and Java, blocks are delimited by matching pairs
of braces `\verb'{...}''.\\[2ex]
In some languages, blocks may be {\em nested}, in which case
variable bindings at outer blocks may be {\em shadowed}
by bindings in inner blocks.
Consider, for example the following C++ code fragment:
{\Large
\begin{qv}
{ int x = 3;
   { int x = 5;
     cout << x << endl;
   }
   cout << x << endl;
}
\end{qv}
}
This code displays 5 and then 3.\exx
In block structured languages,
a variable in an expression is bound
to the variable with the same name
in the {\em innermost} block that defines the variable.
(Note that Java does not allow the same variable to be defined
both in an outer block and in an inner block.)
\end{minipage}
\clearpage
