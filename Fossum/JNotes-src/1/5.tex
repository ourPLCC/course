\begin{minipage}[t]{\sw}
\slidenumber
\LARGE
{\bf BNF (continued)}\\[2ex]
\emm{\Large\LightBox{\MYlon}}\exx
BNF has some shortcuts {\bf that we do not use}
but that you may encounter in your reading.
These shortcuts are usually called Extended BNF, or simply EBNF.
For example, instead of writing two different formulas
with \verb'<nums>' on the LHS,
one can use {\em alternation} notation ``\verb'|''':
\begin{quote}
\verb'<nums> ::= NUM <nums> | ' $\epsilon$
\end{quote}
{\em Note:} `$\epsilon$' means the empty string.\exx
One could also use the {\em Kleene star} notation
to define \verb'<nums>':
\begin{qv}
<nums> ::= { NUM }*
\end{qv}
{\bf Note:} We use a variant of the Kleene star notation later.
\end{minipage}
\clearpage
