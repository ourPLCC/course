\begin{minipage}[t]{\sw}
\slidenumber
\LARGE
{\bf Environments}\exx
In virtually all programming languages,
programmers create symbols (variables) and associate values with them.
We discussed bindings earlier.
What we want to show now is how to implement bindings.
Our implementation allows us to implement {\em static scope rules},
since this is the most common binding method in programming languages today.\exx
An {\em environment} is a data structure that associates a value
with each element of a finite set of symbols --
that is, it represents a set of bindings.
We could think of an environment as a set of pairs
\Large
\[\{(s_1,v_1),\cdots,(s_n,v_n)\}\]
\LARGE
that encode the binding of symbol $s_1$ to value $v_1$,
$s_2$ to value $v_2$, {\em etc.}
The problem with this simple approach is that
the same symbol may have different bindings
in different parts of the program,
and this approach doesn't make it clear
how to determine which binding is the {\em current} binding.\exx
Instead, we specify an environment as a Java object
having a method called \verb'applyEnv' that,
when passed a symbol (a \verb'String') as a parameter,
returns the current value bound to that symbol.
So if \verb'env' is an environment and \verb'x' is a symbol,
\Large
\begin{qv}
env.applyEnv("x")
\end{qv}
\LARGE
returns the value currently bound to the symbol \verb'x'.\exx
\end{minipage}
\clearpage
