\begin{minipage}[t]{\sw}
\slidenumber
\LARGE
{\bf PLCC -- Syntax (continued)}\exx
PLCC requires that there are no duplicates
of Java class names associated with the given grammar rules.
In particular, if two grammar rules have the same LHS nonterminal name,
then their left-hand sides must have annotations
giving distinct Java class names.
For example, the grammar rules on Slide 1.3
\begin{qv}
<nums> ::= <NUM> <nums>
<nums> ::= 
\end{qv}
would not be acceptable to PLCC.
The following annotations fix this:
\begin{qv}
<nums>:NumsNode ::= <NUM> <nums>
<nums>:NumsNull ::=
\end{qv}
When LHS rules are annotated in this way,
the nonterminal class name (\verb'Nums', in this example)
becomes an abstract Java class,
and the annotated class names are used
to generate classes that extend the abstract class.
In this example,
both the \verb'NumsNode' and \verb'NumsNull' classes
are declared to extend the abstract \verb'Nums' class.
\end{minipage}
\clearpage
