\begin{minipage}[t]{\sw}
\slidenumber
\LARGE
{\bf PLCC -- Parsing}\exx
PLCC generates a unique Java class for every grammar rule.
Each such class has a static \verb'parse' method
that is called with a \verb'Scan' object parameter
and that returns an instance of the class
with the class fields (defined by the grammar rule RHS as described above)
populated with appropriate values.
For an RHS field corresponding to a token,
the field value -- a \verb'Token' -- comes directly
from a lexeme in the input file being parsed.
For an RHS field corresponding to a nonterminal,
the field value comes from calling the \verb'parse' method
on the nonterminal class name.\exx
Similar remarks apply to repeating rules,
where the \verb'parse' method uses a loop
to populate the \verb'List' fields in the class.
The members of the \verb'List' fields appear
in the same order that their corresponding syntax entities
appear in the input file.\exx
An abstract Java class generated by a nonterminal
that appears on the LHS of more than one grammar rules
also defines a static \verb'parse' method.
This method looks at the current token (delivered by the \verb'Scan' object)
and determines which of the RHS grammar rules corresponds to that token.
It then returns the value obtained
by calling the \verb'parse' method on the derived class
corresponding to the selected grammar rule.
The result is an instance of the derived class,
which is also an instance of the given abstract class.
\end{minipage}
\clearpage
