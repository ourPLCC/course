\begin{minipage}[t]{\sw}
\slidenumber
\LARGE
{\bf PLCC -- Syntax}\exx
The second section of a PLCC language specification file
is the syntax section consisting
of grammar rules in the style of Backus-Naur Form (BNF),
as described in Slide Set 1.
Recall that a BNF grammar rule has the following form:
{\Large
\begin{qv}
LHS ::= RHS
\end{qv}
}
where LHS (the {\em Left Hand Side}) is a nonterminal
and the RHS (the {\em Right Hand Side}) is a sequence
of nonterminals and terminals.
The individual parts of a PLCC grammar rule,
including the `\verb'::='' part,
are separated by whitespace.\exx
Nonterminals in PLCC are identifiers enclosed
between angle brackets `\verb'<'' and `\verb'>''.
The identifier must begin with a lowercase letter
and can consist of zero or more additional letters, digits, or underscores.
Identifiers that match Java reserved words should be avoided.\exx
Terminals in PLCC must begin with an uppercase letter
and can consist of zero or more additional uppercase letters, digits,
or underscores.
A terminal must appear as the name of a token
in the lexical specification section.\exx
{\bf PLCC associates every grammar rule
with a unique Java class with a class name derived
from the LHS of the grammar rule by converting its first character
to uppercase.}
Class names that match standard Java class names should be avoided.
\end{minipage}
\clearpage
