\begin{minipage}[t]{\sw}
\slidenumber
\LARGE
{\bf PLCC -- Tokens}\exx
If -- after iterating through all of the lexical specifications --
a token candidate has been identified (first longest match),
the token candidate is used 
to create an instance of the \verb'Token' class.
The \verb'cur()' method advances the input stream
beyond the characters matched,
saves the \verb'Token' object for possible future calls to \verb'cur()',
and returns the \verb'Token' object.
If no token candidate has been identified,
the \verb'cur()' method returns a special \verb'$ERROR' Token object.
The \verb'str' field of this object is a string
of the form `\verb'!ERROR(...)'',
where the `\verb'...'' part is (a representation of)
the current input stream character where the match failed.\exx
If the \verb'cur()' method finds that there is
a non-\verb'null' saved \verb'Token' object from a prior call to \verb'cur()',
it simply returns that \verb'Token' object without any further processing.
Otherwise, processing occurs as described above.\exx
The \verb'adv()' method replaces
the saved \verb'Token' object with \verb'null',
so that a subsequent call to \verb'cur()' will be forced
to get the next token from the input stream.
(If it first detects that the saved token object is already null,
it calls \verb'cur()' to consume the next input token.)\exx
As described earlier, the \verb'cur' method returns a special
\verb'$EOF' token when it detects the end
of the \verb'BufferedReader' input.
\end{minipage}
\clearpage
