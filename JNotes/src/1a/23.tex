\begin{minipage}[t]{\sw}
\slidenumber
\LARGE
{\bf PLCC -- Semantics (continued)}\exx
An \verb'include' feature allows a PLCC language specification file
to include the contents of other files,
making them part of a single specification.
In this way, separately created files
can be combined together to form a single language specification.
The names of include files must be given in the semantics section
of the specification file, and generally appear
at the end of the specification.
Here is an example from a \verb'grammar' file:
\begin{qv}
...
include code
include env
include prim
include val
\end{qv}
In this example, the entire contents of the \verb'code' file will be
read by PLCC as if appended to the \verb'grammar' file,
then the \verb'env' file, and so forth.
The names of the \verb'include' files
will normally be representative of their purposes,
but these names do not otherwise play a role
in the generated Java files.
\end{minipage}
\clearpage
