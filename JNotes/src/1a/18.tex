\begin{minipage}[t]{\sw}
\slidenumber
\LARGE
{\bf PLCC -- Semantics}\exx
For a given ``program'' in the language defined by its specification,
the \verb'parse' method in the {\em start symbol class} --
the class determined by the first nonterminal
in the language BNF grammar rules --
returns an instance of this class.
This instance is the root of the parse tree for the program.
For example, given the BNF rules
{\Large
\begin{qv}
<tree>:Leaf     ::= <NUMBER>
<tree>:Interior ::= LPAREN <SYMBOL> <tree>left <tree>right RPAREN
\end{qv}
}
the \verb'parse' method defined in the \verb'Tree' class
returns an instance of a \verb'Tree' object,
which is the root of the parse tree.
(In what follows, we use the term ``parse tree''
to refer to the root of the parse tree).\exx
{\bf The runtime semantics of any PLCC program is the behavior
obtained by calling the \verb'void $run()' method
on the parse tree.}
For any PLCC language,
the start symbol class
extends a special \verb'_Start' class
generated automatically by \verb'PLCC',
so the parse tree is also an instance of \verb'_Start'.
Because the \verb'_Start' class defines a \verb'$run()' method
whose behavior is to display the \verb'toString' representation
of this object to standard output,
and because the parse tree is an instance of the \verb'_Start' class,
the \verb'$run()' behavior defined on the parse tree
defaults to displaying this \verb'toString' representation.
Here is the code for \verb'$run()' in the \verb'_Start' class:
{\Large
\begin{qv}
    public void $run() {
        System.out.println(this.toString());
    }
\end{qv}
}
\end{minipage}
\clearpage
