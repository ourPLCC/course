\begin{minipage}[t]{\sw}
\slidenumber
\LARGE
{\bf Language INFIX} (continued)\exx
We use semicolons in the syntax of a block to terminate
each of the block's variable definitions (using \verb'def').
As shown in Slide 6.9, 
a \verb'block' is also used to define the body of a procedure.
This means that the body of a procedure can define its own
local variables using \verb'def'.
Since the variable bindings in a block create an extended environment
using the \verb'letrec' strategy,
the RHS expressions in a \verb'def'
can refer to variables defined in the same block,
and procedures can be self-recursive.\exx
As noted above, procedure applications in \verb'INIFX' are expressed
using traditional mathematical notation.
For example, 
the following program evaluates to 120:
{\Large
\begin{qv}
{ def f = proc(x) {if x then x*f(x-1) else one endif};
  def one = 1;
  f(5)
} ;
\end{qv}
}
Observe that body of the the procedure \verb'f' can refer
to itself recursively,
and can even refer to the free variable \verb'one'
before it is defined in the same block,
since the semantics of a \verb'block'
behave as in \verb'letrec'.\exx
Also observe that function calls can be nested, as shown in this example
(which evaluates to 19):
{\Large
\begin{qv}
proc(x) {proc(y) {3*x+y}}(4)(7);
\end{qv}
}
\end{minipage}
