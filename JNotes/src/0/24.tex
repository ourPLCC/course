\begin{minipage}[t]{\sw}
\slidenumber
\LARGE
{\bf Tokens} (continued)\exx
Two special ``tokens'' are defined
in the \verb'Token.java' class:
\begin{qv}
$ERROR
$EOF
\end{qv}
Since the PLCC lexical specification requires that token symbol names
begin with an uppercase letter,
these ``tokens'' cannot be confused
with language-specific token symbols.\exx
The \verb'$ERROR' ``token'' is produced
when the scanner encounters an input character
that does not match the beginning
of any skip or token specification.
The \verb'toString()' value of this token
is of the form \verb'!ERROR(...)',
where the `\verb'...'' part
displays the offending character.
In the context of syntax analysis (which we cover later),
such a character cannot
be part of a syntactically correct program,
so syntax analysis will terminate with an error.\exx
The \verb'$EOF' ``token'' is produced
when the scanner encounters end-of-file on the input stream.
In the context of syntax analysis,
encountering end-of-file signals
that there are no further input tokens to process,
so syntax analysis terminates
(possibly prematurely).
\end{minipage}
\clearpage
