\begin{minipage}[t]{\sw}
\slidenumber
\LARGE
{\bf Lexical Analysis, Syntax analysis, and Semantic analysis}\exx
The {\em syntax} (from a Greek word meaning ``arrangement'')
of a programming language refers
to the rules used to determine the structure
of a program written in the language.
{\em Syntax analysis} is the process
of applying these rules to determine the structure of a program.
A program is {\em syntactically correct}
if it follows the syntax rules defining the language.
Every programming language has syntax rules
(these rules differ from one programming language to another)
that are part of the programming language specification.\exx
Before we can specify the syntax rules of a programming language,
we must specify the {\em lexical} (from a Greek word meaning ``word'')
structure of the language:
the symbols used to construct a program in the language.
These symbols are called {\em tokens}.
{\em Lexical analysis} is the process of applying these rules
by reading program input and isolating its tokens.
Tokens comprise the ``atomic structure'' of a program.\exx
Lexical analysis is also called {\em scanning}.
You can think of scanning as what you do
when you ``scan'' a line of printed text on a page
for the words (tokens) in the text.
Programming language tokens normally consist
of things such as numbers (``23'' or ``54.7''),
identifiers (``foo'' or ``x''),
reserved words (``for'', ``while''),
and punctuation symbols (``.'', ``['').
Every programming language has rules that define
the tokens in the language
(these rules differ from one programming language to another)
that are part of the programming language specification.\exx
\end{minipage}
\clearpage
