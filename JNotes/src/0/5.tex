\begin{minipage}[t]{\sw}
\slidenumber
\LARGE
{\bf Lexical Analysis, Syntax Analysis, and Semantic Analysis (continued)}\exx
When a program produces some output, for example,
the language semantics defines what specific behavior results
from running the program.
For example, the defined semantics of Java dictates
that the following Java program sends,
to the standard output stream, the decimal character 3
followed by a newline:
{\Large
\begin{qv}
public class Div {
    public static void main(String [] args) {
        System.out.println(18/5);
    }
}
\end{qv}
}
This course is about programming language syntax and semantics,
with an emphasis on semantics.
Syntax doesn't matter if you don't understand semantics.\exx
Quick question: what output is produced
by the following snippet of python3 code?
{\Large
\begin{qv}
print(18/5)
\end{qv}
}
Try executing this using the following command:
{\Large
\begin{qv}
python3 -c "print(18/5)"
\end{qv}
}
Does this say anything about how the semantics of Java and Python differ
when it comes to integer division? 
\end{minipage}
\clearpage
