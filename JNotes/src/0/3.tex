\begin{minipage}[t]{\sw}
\slidenumber
\LARGE
{\bf Lexical Analysis, Syntax Analysis, and Semantic Analysis}\exx
A {\em lexical analyzer} is a program or procedure
that carries out lexical analysis for a particular language.
Such a program is also called a {\em scanner},
{\em tokenizer}, or {\em lexer}.
The input to a scanner is a stream (sequence) of characters,
and its output is a stream of tokens.
The behavior of a scanner for a language is defined
by the lexical specification of the language.\exx
A {\em syntax analyzer} is a program or procedure
that carries out syntax analysis for a particular language.
Such a program is also called a {\em parser}.
The input to a parser is a stream of tokens
(produced by a scanner),
and its output is a {\em parse tree}
that is an abstract representation of the structure of the program.
The behavior of a parser for a language is defined
by the syntax specification of the language.\exx
The string of input characters that makes up a token
is called a {\em lexeme}.
For example, when you read a word (token) from printed text on a page,
the particular collection of characters that make up the word is its lexeme.
In this paragraph, the first lexeme is ``the'' (ignoring case),
consisting of the individual letters `t', `h', and `e'.
This lexeme is an instance of an English part of speech called an ``article''.
In this case, ``article'' is the token and ``the'' is the instance.
The other instances of the ``article'' token (in English)
are ``a'' and ``an''.
{\bf A token is an abstraction, and a lexeme 
is an instance of this abstraction}.\exx
\end{minipage}
\clearpage
