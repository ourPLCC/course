\begin{minipage}[t]{\sw}
\slidenumber
\LARGE
{\bf Logic Programming Puzzles}\exx
Continue in this way to define three and more consecutive crossings:
{\Large
\begin{qv}
c3(X,P1,P2,P3,Z) :- c2(X,P1,P2,Y), cross(Y,P3,Z).
c4(X,P1,P2,P3,P4,Z) :- c3(X,P1,P2,P3,Y), cross(Y,P4,Z).
c5(X,P1,P2,P3,P4,P5,Z) :- c4(X,P1,P2,P3,P4,Y), cross(Y,P5,Z).
c6(X,P1,P2,P3,P4,P5,P6,Z) :- c5(X,P1,P2,P3,P4,P5,Y), cross(Y,P6,Z).
c7(X,P1,P2,P3,P4,P5,P6,P7,Z) :- c6(X,P1,P2,P3,P4,P5,P6,Y),
                                cross(Y,P7,Z).
\end{qv}
}
We can determine (by brute force)
that less than seven crossings (always an odd number)
will not get everything to the ending side.
(For example, the query
\begin{qv}
c5(s0000,P1,P2,P3,P4,P5,s1111)?
\end{qv}
gives no results.)
However, \verb'c7' has two matches:
\begin{qv}
c7(s0000, g, x, f, g, c, x, g, s1111)
c7(s0000, g, x, c, g, f, x, g, s1111)
\end{qv}
Reading the letters from left to right
gives the list of passengers (\verb'g'=goose, \verb'x'=no passenger, etc.)
of a solution path from the start to the end state.\exx
Observe that both solution paths have the goose as a passenger three times,
but the fox only once.
\end{minipage}
