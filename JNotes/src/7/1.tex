\begin{minipage}[t]{\sw}
\slidenumber
\LARGE
{\bf Logic Programming}\exx
The term {\em logic programming} refers to
a programming paradigm
that principally embodies the rules of {\em first-order logic}
(also known as {\em predicate logic}).
In first-order logic,
you can express {\em facts} involving non-logical objects,
and {\em relations},
where the relations involve facts, standard logical operations,
variables, and (universal and existential) quantifiers.
Logic programming uses {\em rules of inference}
to ``reason'' about a collection of facts and relations.\exx
In the imperative and functional example programming languages
we have studied so far,
we wanted the result
of applying (or ``running'') a program with a given input.
In logic programming,
we want to determine whether a particular logic statement
evaluates to {\em true} or {\em false};
or, more generally,
to determine all of the values of the variables in a logic expression
that make the expression true.\exx
{\bf Prolog} remains the dominant logic programming language implementation,
though there exist variants that extend (or restrict)
Prolog's functionality and purpose:
{\bf Datalog} is one such variant.
{\bf AbcDatalog}, developed at Harvard University,
is a Java-based implementation of Datalog that we use in these notes.
AbcDatalog is covered by the BSD License
(see \verb'http://abcdatalog.seas.harvard.edu/license.txt').
\end{minipage}
