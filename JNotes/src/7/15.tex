\begin{minipage}[t]{\sw}
\slidenumber
\LARGE
{\bf Logic Programming Puzzles}\exx
However, this doesn't show give an explicit soltuion to the puzzle.
Unlike Prolog, the \verb'ABC' language doesn't have a way to build
the solution using some sort of list structure.
So we do it by brute force!\exx
First, observe that a minimal length solution path
(with fewest crossings)
cannot have two crossings
that repeat a start side state,
otherwise you would be back to a previous state.
Since there are exactly 5 start side states of the form \verb's0???'
that can be used for a crossing
(see Slide 7.14),
a minimum-length path can have
at most nine crossings with different start side states.
(Remember that a solution must have an odd number of crossings
because a solution will always have the boat on the ending side.)\exx
Define a premise \verb'c1' for a single crossing:
\begin{qv}
c1(X,P1,Z) :- cross(X,P1,Z). % P1 is the passenger
\end{qv}
Then define a premise \verb'c2' for two consecutive crossings:
\begin{qv}
c2(X,P1,P2,Z) :- c1(X,P1,Y), cross(Y,P2,Z). % P1 then P2
\end{qv}
\end{minipage}
