\begin{minipage}[t]{\sw}
\slidenumber
\LARGE
{\bf Logic Programming}\exx
We similarly define a relation \verb'rel' on \verb'A' to be {\em symmetric}
if \verb'rel(x,y)' implies \verb'rel(y,x)'.
Just like we did for a reflexive closure,
we can create the {\em symmetric closure} of the relation \verb'rel'
in the \verb'ABC' language as follows:
{\Large
\begin{qv}
rel(X,Y) :- rel(Y,X).
\end{qv}
}
Finally, we define a relation \verb'rel' on \verb'A' to be {\em transitive}
if \verb'rel(x,y)' and \verb'rel(y,z)' implies \verb'rel(x,z)'.
We can create the {\em transitive closure} of the relation \verb'rel'
in the \verb'ABC' language as follows:
{\Large
\begin{qv}
rel(X,Z) :- rel(X,Y), rel(Y,Z).
\end{qv}
}
Notice that this is exactly what we did
with the \verb'larger' relation on the planets.\exx
One can create any one of these closures individually,
or all of them, if you wish.
If your resulting relation is reflexive, symmetric, and transitive,
it is called an {\em equivalence relation},
an important construct in first-order logic and set theory.\exx
(Notice that the rule defining the reflexive closure assumes
that everything in the set \verb'A' is related to {\em something}.)
\end{minipage}
