\begin{minipage}[t]{\sw}
\slidenumber
\LARGE
{\bf Logic Programming}\exx
A {\em binary relation} on a set $A$
is defined to be a subset of the cartesian product $A \times A$.
In the \verb'ABC' language, we can represent a binary relation
as a set of facts.
For example, consider the set \verb'A = {a,b,c,d}'.
Define a binary relation called \verb'rel' on \verb'A'
in the \verb'ABC' language as follows,
which you would enter as facts using the Java \verb'Rep' program:
{\Large
\begin{qv}
rel(a,b).
rel(a,d).
rel(d,a).
rel(c,c).
\end{qv}
}
If you were to enter the query \verb'rel(X,Y)?',
you would get a response back consisting
of exactly the same four facts (in some order).\exx
We use the term {\em reflexive}
to describe a binary relation \verb'rel' on \verb'A' having the property
that if \verb'x' is any element in the set \verb'A',
then \verb'rel(x,x)' is true ({\em i.e.}, is a fact).
You can see that the \verb'rel' binary relation
is definitely not reflexive,
but we can add a rule to \verb'rel' to make it reflexive,
as follows:
{\Large
\begin{qv}
rel(X,X) :- rel(X,_).
rel(X,X) :- rel(_,X).
\end{qv}
}
We call the resulting modification to the relation \verb'rel'
its {\em reflexive closure}.
Note that we are using the `\verb'_'' variable as a place holder
that can match anything already in the \verb'rel' relation.
\end{minipage}
