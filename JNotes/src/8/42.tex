\begin{minipage}[t]{\sw}
\slidenumber
\LARGE
{\bf Concurrency} (continued)\exx
In general, we want to define a mechanism
that can carry out a simulation of parallel execution
of multiple continuations.
First, we build a queue that holds all of the continuations
to be executed in parallel.
Then we create a ``wrapper'' continuation
that takes a single continuation from the queue,
calls the dequeued continuation's \verb'apply()' method,
and puts the result back on the end of the queue.
Once all of the continuations have completed,
the wrapper continuation returns
the next continuation step in the evaluation
(such as primitive application, procedure application,
or \verb'let' body evaluation).
The wrapper continuation uses the trampoline
to carry out each of its dequque steps.
A \verb'ConcurrentCont' class,
shown on the next slide, serves this purpose.\exx
This approach does not achieve true parallelism,
since we are still applying the continuation steps
one at a time (using the trampoline),
but in the presence of suitable hardware,
it would not be difficult to dispatch the application
of the queued continuations to separate threads.
\end{minipage}
