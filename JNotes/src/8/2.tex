\begin{minipage}[t]{\sw}
\slidenumber
\LARGE
{\bf Continuations}\exx
An \verb'eval' method call is intended
to carry out some computation --
one that may be used, for example,
to evaluate an actual parameter expression in a procedure application
or a test expression in an \verb'if' expression.
A stack frame gets created implicitly
by the Java Runtime Environment (JRE)
upon each \verb'eval' method call.
This stack frame consists of information
including the method arguments,
where to find non-local variables ({\em i.e.}, an environment),
and a ``return address'' that indicates
where the JRE should execute next when the method finishes.
This information can be considered as an ``execution context''.
Once the method finishes, this context is discarded
(popped off the stack)
and the execution context of the caller takes over.\exx
One way to avoid stack overflow
is to maintain execution context explicitly.
Instead of using the JRE stack to hold this information,
we pass along an execution context to the \verb'eval' method
that is used by the method
to determine what should be executed next
when the method finishes.
Such an execution context is called a {\em continuation}.
The idea is that a continuation determines
how the overall compution should continue
once the current computation is finished.
\end{minipage}
