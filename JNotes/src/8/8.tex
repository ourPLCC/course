\begin{minipage}[t]{\sw}
\slidenumber
\LARGE
{\bf Continuations} (continued)\exx
Things start off by creating an initial
expression evaluation continuation \verb'acont'
and then jumping onto the trampoline:
\Large
\begin{qv}
acont.trampoline();
\end{qv}
\LARGE
In this code, the trampoline loops
until one of the calls to \verb'apply' throws a \verb'ContException'.
In order for expression evaluation to terminate,
some continuation must therefore throw a \verb'ContException'
that jumps off the trampoline.
Since the purpose of expression evaluation is
to produce a value, 
this exception needs to have a \verb'Val' field
that can be used to return a value to the top-level REP loop.\exx
We choose a special ``halt continuation'' \verb'HaltCont' to do this. 
\verb'HaltCont' extends the \verb'VCont' class,
so its \verb'apply' method takes a \verb'Val' parameter.
The \verb'apply' method in this class --
the one that actually jumps off the trampoline --
is now trivial.
\Large
\begin{qv}
public class HaltCont extends VCont {

    public ACont apply(Val val) {
        throw new ContException(val);
    }

}
\end{qv}
\LARGE
This continuation is created once,
at the top level of expression evaluation.
\end{minipage}
