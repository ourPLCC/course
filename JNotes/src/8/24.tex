\begin{minipage}[t]{\sw}
\slidenumber
\LARGE
{\bf Continuations} (continued)\exx
To evaluate a procedure application,
we need to evaluate the procedure expression
(it must evaluate to a \verb'ProcVal'),
the actual parameter expressions using reference semantics,
bind the actual parameter references to their formal parameter names,
use these bindings to extend the environment captured by the procedure,
and finally evaluate the body of the procedure in this extended environment.
The \verb'eval' method in the \verb'AppExp' class
is given here:
\Large
\begin{qv}
AppExp
%%%
public Cont eval(Env env, Cont cont) {
    return new EvalCont(exp,
                        env,
                        new AppCont(rands, env, vcont));
}
%%%
\end{qv}
\LARGE
The \verb'AppCont' continuation, shown on the next slide,
gets an expression that must evaluate to a \verb'ProcVal',
evaluates the reference parameters,
and passes the reference parameters along
with the \verb'vcont' continuation to the \verb'ProcVal'
to evaluate the procedure body
and pass its value along for further processing.
\end{minipage}
