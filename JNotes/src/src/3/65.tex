\begin{minipage}[t]{\sw}
\slidenumber
\LARGE
{\bf Language V4} (continued)\exx
The only thing we have left is to implement
the behavior of the \verb'apply' method in the \verb'Val' class.
Since we want \verb'apply' only to be meaningful for a \verb'ProcVal' object,
we define a default behavior in the (abstract) \verb'Val' class
to throw an exception for anything but a \verb'ProcVal':
{\Large
\begin{qv}
public Val apply(List<Val> args, Env e) {
    throw new PLCCException("Cannot apply " + this): 
}
\end{qv}
}
For a \verb'ProcVal', here's the implementation of \verb'apply'.
Notice that this implementation carries out steps 2a, 2b, and 3
in the semantics for evaluating a procedure application (Slide 3.60).
{\Large
\begin{qv}
public Val apply(List<Val> args, Env e) {
    // bind the formals to the arguments (step 2a)
    Bindings bindings = new Bindings(formals.varList, args);
    // extend the captured environment with these bindings (step 2b)
    Env nenv = env.extendEnv(bindings);
    // and evaluate the body in this new environment (step 3)
    return body.eval(nenv);
}
\end{qv}
}
Language V4 also checks for duplicate identifiers
while parsing the \verb'Formals' in a procedure definition.
This code is inserted into the \verb'Formals' class constructor
using the \verb':init' hook.
\end{minipage}
\clearpage
