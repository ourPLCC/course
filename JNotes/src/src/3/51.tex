\begin{minipage}[t]{\sw}
\slidenumber
\LARGE
{\bf Language V3} (continued)\exx
{\large
\emm\VerbBox{\colorbox{light}}{\begin{tabular}{@{}ll}
\verb'<exp>:LetExp' & \verb'::= LET <letDecls> IN <exp>'\\
  & \VerbBox{\fbox}{\verb'LetExp(LetDecls letDecls, Exp exp)'}\\
\verb'<letDecls>' & \verb'**= <VAR> EQUALS <exp>'\\
  & \VerbBox{\fbox}{\verb'LetDecls(List<Token> varList, List<Exp> expList)'}\\
\end{tabular}}
\exx
}
Here is another observation you should pay attention to.
In the \verb'<letDecls>' rule, each \verb'<VAR>' symbol
is called the {\em left-hand side} (LHS) of the binding
and the corresponding \verb'<exp>'
is called its {\em right-hand side} (RHS).
(Don't confuse this with the LHS and RHS of the grammar rule itself.)
All of the RHS expressions in a \verb'LetDecls' are evaluated
in the enclosing environment.
{\em The LHS \verb'<VAR>' variables become bound
to their corresponding RHS expression values
{\bf after} all of the RHS expressions have been evaluated}.
Thus the following expression
\begin{qv}
let p = 4
in
  let
    p = 42
    x = p
  in
    x
\end{qv}
evaluates to 4.
\end{minipage}
\clearpage
