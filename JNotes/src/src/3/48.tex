\begin{minipage}[t]{\sw}
\slidenumber
\LARGE
{\bf Language V3} (continued)\exx
{\Large
\emm\VerbBox{\colorbox{light}}{%
\begin{tabular}{@{}ll}
\verb'<exp>:LetExp' & \verb'::= LET <letDecls> IN <exp>'\\
  & \VerbBox{\fbox}{\verb'LetExp(LetDecls letDecls, Exp exp)'}\\
\verb'<letDecls>' & \verb'**= <VAR> EQUALS <exp>'\\
  & \VerbBox{\fbox}{\verb'LetDecls(List<Token> varList, List<Exp> expList)'}\\
\end{tabular}}%
}\exx
A \verb'LetDecls' object
has two instance variables:
\verb'varList' is a list of \verb'Token' objects
representing the \verb'<VAR>' part of the BNF grammar rule,
and \verb'expList' is a list of expressions
representing the \verb'<exp>' part of the BNF grammar rule.
(The reason that these are \verb'List's is because
the \verb'letDecls' BNF grammar rule is repeating.)
Our plan for defining the \verb'addBindings' method
in the \verb'LetDecls' class
involves evaluating each of the expressions in \verb'expList'
in the enclosing environment and binding these values
to their corresponding token strings in \verb'varList'.
We then use these bindings to extend the enclosing environment
given by the \verb'env' parameter,
and we return this new environment to the \verb'eval' method
in the \verb'LetExp' class.\exx
The \verb'LetDecls' constructor throws an exception
if it finds duplicate identifiers in its \verb'varList'.
This means that a \verb'let' expression cannot have
two instances of the same LHS idenfier.
The code to check for duplicates is inserted
into the \verb'LetDecls' class constructor
using the ({\em context-sensitive}) \verb':init' hook,
so the check for duplicates occurs during parsing,
not during expression evaluation.
\end{minipage}
\clearpage
