\begin{minipage}[t]{\sw}
\slidenumber
\LARGE
{\bf Language V1} (continued)\exx
The definitions of \verb'apply' for the classes
\verb'SubPrim', \verb'MulPrim', and \verb'DivPrim'
(the latter two are added in V1)
have obvious implementations, except that in \verb'DivPrim',
the \verb'apply' method throws an exception
if it detects an attempt to divide by zero:
this code is not shown here.\exx
For the \verb'Add1Prim' class,
the \verb'apply' method expects only one value,
which is passed as element zero of the \verb'va' array.
{\Large
\begin{qv}
Add1Prim
%%%
    public Val apply(Val [] va) {
        if (va.length != 1)
            throw new PLCCException("one argument expected");
        int i0 = va[0].intVal().val;
        return new IntVal(i0 + 1);
    }
%%%
\end{qv}
}
Again, the definition of \verb'apply' for the \verb'Sub1Prim' class
is entirely similar.
The definition of \verb'apply' for the \verb'ZeropPrim' class
returns an \verb'IntVal' of 1 (true) for a zero argument
and an \verb'IntVal' of 0 (false) for a nonzero argument.
\end{minipage}
\clearpage
