\begin{minipage}[t]{\sw}
\slidenumber
\LARGE
{\bf Language V0} (continued)\exx
We follow the method described in Slide Set 1
to redefine the default behavior of the \verb'$run()' method
in the \verb'Program' class.
{\em In all of our languages,
the observable {\bf semantics} of a program
is the output produced by the \verb'$run()' method
applied to the root of the parse tree of the program.}\exx
Recall that to add Java code (fields and method definitions)
to a class such as \verb'Program',
use the following template:
\Large
\begin{qv}
Program
%%%
...Java code...
%%%
\end{qv}
\LARGE
Since PLCC inserts this code verbatim
into the body of the \verb'Program' class,
any methods defined in this code
can access all of the instance variables in the class.
So for the \verb'Program' object,
these methods can refer to the \verb'exp' instance variable
of type \verb'Exp'.
Since the RHS of the \verb'<program>' grammar rule is just \verb'<exp>',
the \verb'$run()' method of the \verb'Program' class is simple:
{\Large
\begin{qv}
Program
%%%
    public void $run() {
        System.out.println(exp.toString());
    }
%%%
\end{qv}
}
\end{minipage}
\clearpage
