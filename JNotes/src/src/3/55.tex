\begin{minipage}[t]{\sw}
\slidenumber
\LARGE
{\bf Language V4} (continued)\exx
Now consider the following examples, all of which evaluate to 5:
\large
\begin{qv}
let
  app = proc(f,x) .f(x)
  add2 = proc(y) add1(add1(y))
in
  .app(add2,3)

let
  app = proc(f,x) .f(x)
in
  .app(proc(y) add1(add1(y)), 3)

.proc(f,x) .f(x) (proc(y) add1(add1(y)), 3)
\end{qv}
\LARGE
In the first example,
observe that we can pass a procedure (in this case \verb'add2')
as a parameter to another procedure.
This \verb'app' procedure takes two parameters
and returns the result of applying the first actual parameter
to the second.
Of course, the first parameter had better be bound to a procedure
for this to work.
(If it isn't, an attempt to apply it throws an exception.)\exx
In the second example,
we have eliminated the identifier \verb'add2'
and instead simply replaced \verb'add2' in the application \verb'.app(add2,3)'
with the nameless procedure \verb'proc(y) add1(add1(y))'
that used to be called \verb'add2'.\exx
In the third example,
we have even eliminated the identifier \verb'app'.
\end{minipage}
\clearpage
