\begin{minipage}[t]{\sw}
\slidenumber
\LARGE
{\bf Language V3} (continued)\exx
The languages we have discussed do not allow mutation of variables,
although you might be tempted to think that this Language V3 program
is doing something akin to mutation:
\Large
\begin{qv}
let
  x = 3
in
  let
    x = add1(x)
  in
    +(x, x)
\end{qv}
\LARGE
This program evaluates to \verb'8' (which is not surprising),
but in the scope of the outer \verb'let',
the variable \verb'x' is still bound to \verb'3'.
To see this, consider the following variant of this program:
\Large
\begin{qv}
let
  x = 3
in
  +(let x = add1(x) in x, x)
\end{qv}
\LARGE
The last occurrence of \verb'x' in this expression evaluates to \verb'3'
because the variable \verb'x' in the inner \verb'let'
has scope only through the inner \verb'let' expression body.
Outside of the inner \verb'let' expression body,
the binding of \verb'x' to \verb'3' remains unchanged.
Thus the entire expression evaluates to \verb'7'.
\end{minipage}
\clearpage
