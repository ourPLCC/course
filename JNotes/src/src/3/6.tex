\begin{minipage}[t]{\sw}
\slidenumber
\LARGE
{\bf Language V0} (continued)\exx
Example ``programs'' in this language:
\begin{qv}
3
x
+(3, x)
add1( +(3,x) )
+(4, -(5,2))
\end{qv}
Observe that in Language V0 --
and in most of the other languages
you will see in these class notes --
we write arithmetic expressions in {\em prefix form},
where the arithmetic operator (such as `\verb'+'' or `\verb'-'')
precedes its operands.
(A prefix form expression like `\verb'+(3,x)'' would normally be written
mathematically as `\verb'3+x'', using {\em infix form}.
It turns out that prefix form expressions are easier
to parse and evaluate
than infix form expressions,
which is why we use prefix form in our languages.
See the \verb'INFIX' language in Slide Set 6
for a further discussion of infix form.)\exx
Prefix form is not entirely unusual:
languages in the Lisp family (including Scheme) use prefix form.
Contrast this to languages such as C, Java, and Python,
where arithmetic operators appear principally in infix form.
\end{minipage}
\clearpage
