\begin{minipage}[t]{\sw}
\slidenumber
\LARGE
{\bf Logic Programming Puzzles}\exx
For example, in the first river crossing
(starting from state \verb's0000'),
the farmer takes the boat and the goose
from the starting side to the ending side.
We represent this in the \verb'ABC' language by the following fact:
\begin{qv}
cross(s0000, g, s1010). % carry the goose in the boat
\end{qv}
You can see that this is the {\em only} legal crossing
starting from the initial state \verb's0000'.
We use the letter \verb'x' to mean a river crossing
with no ``passenger'' in the boat.\exx
Observe that for every legal crossing of the form
\begin{qv}
cross(s0..., ?, s1...).
\end{qv}
(where the \verb'?' can be any of \verb'fgcx')
with the boat (and farmer) going from the starting side
to the ending side of the river,
there is a corresponding legal crossing of the form
\begin{qv}
cross(s1..., ?, s0...).
\end{qv},
and {\em vice versa}.
This means that we need only consider the legal crossings
where the boat is on the starting side of the river,
and use the following rule to generate the others:
\begin{qv}
cross(X, P, Y) :- cross(Y, P, X).
\end{qv}
\end{minipage}
