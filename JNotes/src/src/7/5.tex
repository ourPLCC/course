\begin{minipage}[t]{\sw}
\slidenumber
\LARGE
{\bf Logic Programming}\exx
Continuing the above example,
the query `\verb'larger(X,Y)?''
produces the following result:
{\Large
\begin{qv}
2 matches:
larger(jupiter, earth)
larger(earth, mercury)
\end{qv}
}
Can we logically deduce that jupiter is larger than mercury?
The answer is `no', since there is no fact that says this:
if we were to try the query `\verb'larger(jupiter,mercury)?'',
we would get no matches.\exx
To make the \verb'larger' relation transitive,
we build a {\em rule} that says that
if \verb'X' is larger than \verb'Y',
and if \verb'Y' is larger than \verb'Z',
then \verb'X' is larger than \verb'Z'.
We do so as follows:
{\Large
\begin{qv}
larger(X,Z) :- larger(X,Y), larger(Y,Z).
\end{qv}
}
The LHS of this rule (the part to the left of `\verb':-'')
is called an {\em Atom},
and the RHS of this rule is a sequence of {\em Premises}
separated by commas.
You can read this rule as saying
{\em the LHS is true if all of the Premises on the RHS are true}.
The result is all of the possible substitutions
of \verb'X', \verb'Y', and \verb'Z'
that make the rule true.
\end{minipage}
