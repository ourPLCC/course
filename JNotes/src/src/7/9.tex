\begin{minipage}[t]{\sw}
\slidenumber
\LARGE
{\bf Logic Programming Puzzles}\exx
Consider the Fox/Goose/Corn river crossing puzzle
(also known as the Wolf/Goat/Cabbage puzzle):\exx
A farmer, a fox, a goose, and a bag of corn are
on one side of a river.
The farmer has a boat that can hold {\em at most one} other ``passenger'' --
either the fox, the goose, the bag of corn, or nothing at all.
The problem is to transport everything
to the other side of the river using the boat.
One constraint is that the farmer cannot
leave the fox and the goose together on one side
with the farmer on the other side --
otherwise the fox, unsupervised, will eat the goose.
Another constraint is that the farmer cannot
leave the goose and the bag of corn together on one side
with the farmer on the other side --
otherwise the goose, unsupervised, will eat the corn.
A final constraint is that the farmer and the boat
will always be on the same side of the river together.
{\em Can the farmer transport everything
from one side of the river to the other
using the boat, while maintaining the constraints?}\exx
We will model this problem by considering all possible {\em states},
where a state is a string of the form \verb'sbfgc'.
Here, \verb's' is just the letter `s',
which stands for {\em state}.
The \verb'bfgc' positions will be digits \verb'0' or \verb'1',
referring to which side of the river the boat (and the farmer),
the fox, the goose, and the corn are on, respectively.
We will use the digit \verb'0' to mean the starting side
and the digit \verb'1' to mean the ending side.
So the problem is to move from state \verb's0000' to state \verb's1111'
by legal river crossings.\exx
\end{minipage}
