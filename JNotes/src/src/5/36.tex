\begin{minipage}[t]{\sw}
\slidenumber
\LARGE
{\bf Language OBJ} (continued)\exx
Unlike the \verb'new' operator in Java,
Our \verb'new' operator does not take any arguments,
and all of the fields are initialized to \verb'nil'.
We can, of course, initialize fields by calling a method.
Here's an example:
\Large
\begin{qv}
let
  c = class
        field x
        field y
        method init = proc(a,b) { set x=a ; set y=b ; self }
      end
  in
    let
      o = .<new c>init(3,4)
    in
      <o>+(x,y) % => 7
\end{qv}
\LARGE
Since the \verb'init' method returns \verb'self',
the value of the expression \verb'.<new c>init(3,4)'
is the {\em same} object as the one created by the expression \verb'new c',
except that its fields \verb'x' and \verb'y' are set
to values 3 and 4, respectively.\exx
You might be inclined to think that \verb'.<new c>init(a,b)'
is the same as \verb'<new c>{ set x=a ; set y=b ; self}',
but the bindings for a and b may be different in these two expresions.

\end{minipage}
