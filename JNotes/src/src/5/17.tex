\begin{minipage}[t]{\sw}
\slidenumber
\LARGE
{\bf Language OBJ} (continued)\exx
The \verb'makeObject' method for \verb'EnvClass' is simple, since
it's always the last superclass object that needs to be constructed.
It extends the EnvClass environment, namely the top-level environment,
with a single field binding of \verb'self' to (a reference to) the object
being created. This binding is ``deep'', in the sense that the \verb'objRef'
reference may ultimately refer to an object defined in a deeply nested
subclass.
As shown in the \verb'NewExp' code for \verb'eval' on Slide 5.13,
once the objects are created in the chain of superclasses,
\verb'objRef' is finally bound to to the object
at the beginning of the chain.

\Large
\begin{qv}
public ObjectVal makeObject(Ref objRef) {
    // start with the static (top-level) env. of this class
    Env env = staticEnv;
    // add the field binding 'self' to refer to
    // the object being created (objRef)
    Bindings fieldBindings = new Bindings();
    fieldBindings.add("self", objRef);
    env = env.extendEnvRef(fieldBindings);
    return new ObjectVal(env);
}
\end{qv}
\LARGE
Observe that an \verb'ObjectVal' can access the static environment
of the top-level class, which is the top-level program environment.
\end{minipage}
