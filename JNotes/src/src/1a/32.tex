\begin{minipage}[t]{\sw}
\slidenumber
\LARGE
{\bf PLCC -- Hooks (continued)}\exx
You use the final hook in an automatically-generated Java source file
to add method definitions (and sometimes field declarations)
that will be appended to the class definition.
We have already seen how this is used,
for example, in Language LON2 on Slide 1.48,
which we repeat here:
{\Large
\begin{qv}
Lon
%%%
    public void $run() {
        System.out.print("( ");
        for (Token tok: nums.numList)
            System.out.print(tok.toString() + " ");
        System.out.println(")");
    }
%%%
\end{qv}
}
For temporary testing purposes,
you may disable adding the \verb'$run()' method definition
into the \verb'Lon.java' source file
by replacing the \verb'Lon' line with the following:
\begin{qv}
Lon:ignore!
\end{qv}
This is simpler (and easier to undo) than 
``commenting out'' all of these lines
(including the `\verb'%%%'' lines)
by turning them into PLCC `\verb'#'' comments.
In this case, PLCC reads the lines
in the \verb'grammar' file bracketed by the `\verb'%%%'' lines
but otherwise ignores them, making no changes
to \verb'Lon.java'.
\end{minipage}
\clearpage
