\begin{minipage}[t]{\sw}
\slidenumber
\LARGE
{\bf PLCC -- Semantics (continued)}\exx
As we observed above,
PLCC automatically generates a Java source file for each of the classes --
both abstract and non-abstract --
that are derived from the BNF grammar rules in the syntax section.
In the semantics section of the language specification file,
if PLCC encounters an entry of the form
\begin{qv}
ClassName
%%%
...
%%%
\end{qv}
where \verb'ClassName' stands for a class that is {\em not} one
of the automatically generated classes,
then PLCC generates a new file \verb'ClassName.java'
containing the code bracketed by the `\verb'%%%'' lines.
This makes it possible for PLCC to generate Java source files
that can be used to augment the semantics of the language.
For example, we will use this to implement {\em environments},
which we describe later.\exx
In this situation, there is no automatically generated source file,
so the Java code bracketed by the `\verb'%%%'' lines
must be a complete Java source file, not just a method.
\end{minipage}
\clearpage
