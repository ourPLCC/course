\begin{minipage}[t]{\sw}
\slidenumber
\LARGE
{\bf PLCC -- Syntax (continued)}\exx
Repeating grammar rules --
ones that have `\verb'**='' instead of `\verb'::='' --
result in RHS fields similar
to non-repeating grammar rules,
except that their RHS fields are \verb'List's of the appropriate type,
and their field names have the string \verb'List' appended.\exx
The following grammar rule (see Slide 1.36)
{\Large
\begin{qv}
<nums> **= <NUM>
\end{qv}
}
coressponds to the Java class \verb'Nums',
having a single field \verb'numList' of type \verb'List<Token>'.\exx
Similarly, the following grammar rule (we will encounter this later)
{\Large
\begin{qv}
<letDecls> **= LET <VAR> EQUALS <exp>
\end{qv}
}
corresponds to the Java class \verb'LetDecls'
having one field \verb'varList' of type \verb'List<Token>'
and another field \verb'expList' of type \verb'List<Exp>'.\exx
{\bf A repeating rule cannot be the first rule in a grammar file.
A non base (annotated) class cannot define a repeating rule.
A repeating rule cannot have an empty RHS.}
\end{minipage}
\clearpage
