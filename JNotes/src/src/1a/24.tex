\begin{minipage}[t]{\sw}
\slidenumber
\LARGE
{\bf PLCC -- Java predefined/reserved name conflicts}\exx
The Java class names generated by the LHS of rules
in the grammar section of a PLCC file
must not conflict with standard Java class names.
This means that with a grammar rule such as
{\Large
\begin{qv}
<string> ::= ...
\end{qv}
}
PLCC creates a Java class named \verb'String',
which results in a Java compile-time error
since `\verb'String'' is a reserved class name.\exx
Similarly, the names of fields in the RHS of rules
in the grammar section must not conflict
with Java reserved words or predefined identifiers.
This means with a grammar rule such as
{\Large
\begin{qv}
<foo> ::= <IF> <blah>null
\end{qv}
}
PLCC creates a field named \verb'if' in the \verb'Foo' class,
which results in a Java compile-time error
since \verb'if' is a reserved word in Java.
PLCC itself does not attempt to identify these errors,
so these errors will only show up during compilation.
\end{minipage}
\clearpage
