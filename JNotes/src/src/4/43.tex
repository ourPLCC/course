\begin{minipage}[t]{\sw}
\slidenumber
\LARGE
{\bf Language TYPE1} (continued)\\[0ex]
\Large
We implement top-level declarations as follows:
{\large
\begin{qv}
Declare
%%%
    // calling $run may trigger a modificaiton
    // of the initial type environment
    public void $run() {
        TypeEnv tenv = Program.tenv; // top-level type environment
        Type tv; // variable's declared type
        String sym = var.toString(); // the LHS symbol
        try {
            // look up the variable in the initial type environment
            tv = tenv.applyTypeEnv(sym);
        } catch (PLCCException e) {
            // no type binding -- must be a new type declaration
            // that we add to the top-level type environment
            tv = typeExp.toType();
            tenv.add(new TypeBinding(sym, tv)); // type binding
            System.out.println(sym + ":" + tv);
            return;
        }
        throw new PLCCException(sym + ": duplicate variable declaration");
    }
%%%
\end{qv}
}
If a type binding
(either through a declaration of definition)
already exists, it is flagged as a duplicate declaration.
Otherwise a top-level binding is made
between the identifier and the type represented
by its type expression.
The \verb'$run' behavior of a variable declaration/definition
is to display the variable being declared/defined
along with its declared type.
\end{minipage}
