\begin{minipage}[t]{\sw}
\slidenumber
\LARGE
{\bf Language TYPE1} (continued)\exx
\Large
The definitions of the classes relating to type environments
is given in the \verb'tenv' file.\exx
Since most of our type checking rules relate to type equality,
we must define what it means for two types to be equal.
There are only three basic types,
and for each of them we define a \verb'checkEquals' method.
This \verb'void' method silently returns
if the types are the same,
otherwise it calls the static \verb'typeMismatch' method
in the \verb'Type' class;
this method throws a runtime exception
with an appropriate message.\exx
The two primitive types are simple.  For the \verb'BoolType' class,
the \verb'checkEquals' method is defined as follows:
\large
\begin{qv}
public void checkEquals(Type t) {
    t.checkBoolType(this);
}

public void checkBoolType(BoolType t) {
    // if we get here, this must be a BoolType
}
\end{qv}
\Large
The default \verb'checkBoolType' method in the \verb'Type' class --
defined for all other types except for \verb'BoolType' --
throws a type mismatch exception.\exx
A similar definition works for the \verb'IntType'.\exx
We are left to define \verb'checkEquals' for the \verb'ProcType' class.
This is easy: two \verb'ProcType' objects are equal
if they both have the same number of formal parameter types
and these formal parameter types are pairwise equal (in the proper order),
and if they both have the same return type.
\end{minipage}
