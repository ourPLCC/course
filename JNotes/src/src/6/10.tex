\begin{minipage}[t]{\sw}
\slidenumber
\LARGE
{\bf Language INFIX} (continued)\exx
The \verb'INFIX' language also supports the assignment
of values to variables in a way that is similar
to the \verb'set' expression in Language \verb'SET'
but without using the token \verb'set'.
The right-hand-side of a such an assignment
must be an atom instead of an arbitrary expression.
The \verb'<assign>' grammar rule defines its syntax
(as shown on Slide 6.9).
Because the RHS in an \verb'Assign1' rule is an atom,
assignment has higher precedence than any infix arithmetic operations.
{\Large
\begin{qv}
<atom>:VarAtom         ::= <VAR> <assign>
<assign>:Assign0       ::=
<assign>:Assign1       ::= EQUALS <atom>
\end{qv}
}
For example, in the following code,
the expression \verb'x*y' evaluates to 64:
{\Large
\begin{qv}
define x=5;
define y=11;
x=(y-3);     % => 8
y=x+5;       % => assigns 8 to y, evaluates to 13
x*y;         % => 64
\end{qv}
}
We borrow the syntax of \verb'letrec' expressions in Language V5
to create a new atom called a \verb'<block>',
with the following grammar rules:
{\Large
\begin{qv}
<atom>:BlockAtom  ::= <block>
<block>           ::= LBRACE <blockDecls> <exp> <exps> RBRACE
<blockDecls>      **= DEF <VAR> EQUALS <exp> SEMI
<exps>            **= SEMI <exp>
\end{qv}
}
\end{minipage}
