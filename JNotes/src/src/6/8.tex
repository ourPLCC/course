\begin{minipage}[t]{\sw}
\slidenumber
\LARGE
{\bf Language INFIX} (continued)\exx
Slide 6.4 is a simplified version
of the actual grammar for Language \verb'INFIX'.
For example,
we want to include a ``unary minus'' primitive
that acts like the \verb'NEG' primitive in Assignment A3
but that uses the traditional unary minus prefix operator
and that has precedence higher than addition/subtraction
and multiplication/division.
Here are grammar rules that support this,
where \verb'SUBOP' is the token '\verb'-''
{\Large
\begin{qv}
<factor>:Prim2Factor  ::= <prim2> <factor>
<prim2>:NegPrim2      ::= SUBOP
\end{qv}
}
An expression like
{\Large
\begin{qv}
-3+5;
\end{qv}
}
evaluates to 2.\exx
In the full \verb'INFIX' language,
a \verb'<factor>' BNF rule also has a RHS \verb'<atom>',
which defines rules for 
\verb'if' expressions,
procedure definitions,
procedure applications,
blocks,
variables,
and literals
(which we showed in simplified form as a \verb'<factor>'
on Slide 6.4).
We call these {\em atoms}
because they are at the top of the precedence hierarchy.
The \verb'<atom>' nonterminal in our \verb'INFIX' grammar rules
serves to define their syntax.
\end{minipage}
