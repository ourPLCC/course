\begin{minipage}[t]{\sw}
\slidenumber
\LARGE
{\bf Language INFIX} (continued)\exx
Language \verb'INFIX' defines a unary minus primitive
(defined in class \verb'NegPrim2' that extends the \verb'Prim2' class)
that acts as a prefix operator.
Its behavior is to (arithmetically) negate its \verb'Factor' operand.\exx
Language \verb'INFIX' defines four non-infix primitives,
all of which are instances of the \verb'Prim' class.
Three of them are taken directly from Language \verb'SET':
their syntax and semantics are unchanged from Language \verb'SET'.
The \verb'PospPrim' implementation (\verb'pos?')
is drawn from Assignment A8.\exx
All of the \verb'<atom>' grammar rules
and related semantics in Language \verb'INFIX' are similar
to those in Language \verb'SET'.
As shown in the \verb'ProcAtom' BNF rule on Slide 6.9,
the body of a \verb'proc' definition must be a \verb'<block>'
(in Language \verb'SET', it's an \verb'exp').
The \verb'ProcVal' class is mostly unchanged
from that of Language \verb'SET',
the only exception being that its \verb'apply' method
does not have an \verb'Env' formal parameter.
The \verb'apply' method body is unchanged.\exx
Language INFIX, as given in the Code directory,
has skeleton definitions of the \verb'eval' semantics for expressions.
In an assignment, you are asked to complete these definitions
for a full implementation of Language \verb'INFIX'.
\end{minipage}
