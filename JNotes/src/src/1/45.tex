\begin{minipage}[t]{\sw}
\slidenumber
\LARGE
{\bf Parsers} (continued)\exx
\emm\LightBox{\MYlonGrammarKleene}\exx
How exactly can we have the \verb'Rep' program access and display
the values of the \verb'NUM' fields from the parse tree?
The \verb'Rep' program first parses the program,
yielding an instance of the start symbol class
-- an instance of \verb'Lon', in this case.
It then runs (evaluates) this instance,
which as we have seen defaults
to displaying something like `\verb'Lon@...''.\exx
This default behavior resides in the Java class \verb'_Start'.
This class defines a \verb'void' method named \verb'$run()'
that simply displays the \verb'toString()' value of its instance
(see the Java code for \verb'_Start.java'
in the Java subdirectory of Language \verb'LON').
Since the Java class \verb'Lon' extends the \verb'_Start' class,
evauating the \verb'$run()' method of a \verb'Lon' object
defaults to evaluating the \verb'$run()' method of its superclass,
which gives us the behavior we have already seen.
\verb'Rep' simply calls the \verb'$run()' method
on the \verb'Lon' instance obtined by parsing the program.
\end{minipage}
\clearpage
