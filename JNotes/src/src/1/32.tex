\begin{minipage}[t]{\sw}
\slidenumber
\LARGE
{\bf Parsers} (continued)\exx
\Large
%\emm\LightBox{\MYlonGrammarOnly}\exx
To make the connection clear
between a grammar rule and its PLCC-generated Java class,
in these class notes we display the grammar rules
in the following way:\exx
\emm\begin{tabular}{ll}
\verb'<lon>' & \verb'::= LPAREN <nums> RPAREN'\\
    & \VerbBox{\fbox}{\verb'Lon(Nums nums)'}\\
\verb'<nums>:NumsNode' & \verb'::= NUM <nums>'\\
    & \VerbBox{\fbox}{\verb'NumsNode(Nums nums)'}\\
\verb'<nums>:NumsNull' & \verb'::= '\\
    & \VerbBox{\fbox}{\verb'NumsNull()'}\\
\end{tabular}\exx
The item in the box following a grammar rule
is the {\em Java signature of the Java class constructor}
corresponding to the class PLCC generates from the grammar rule.
So the box\exx
\emm\VerbBox{\fbox}{\verb'Lon(Nums nums)'}\exx
means that the constructor for the PLCC-generated class \verb'Lon'
has a single parameter \verb'nums' of type \verb'Nums'.\exx
{\bf There is a one-to-one correspondence
between the types and formal parameters in the constructor
and the types and field names in the class.}
More specifically, when the constructor is invoked,
the constructor body simply copies the values of its parameters
into the corresponding field names of the instance being constructed.\exx
As you can see from above,
the \verb'NumsNull' constructor takes no parameters,
and the \verb'NumsNull' class has no corresponding fields.
\end{minipage}
\clearpage
