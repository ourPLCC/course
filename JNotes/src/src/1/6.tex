\begin{minipage}[t]{\sw}
\slidenumber
\LARGE
{\bf Parsing (syntatic derivation):}\exx
\emm{\Large\LightBox{\MYlon}}\exx
A BNF grammar defines
the set of all ``legal'' token sequences
that conform to the grammar rules.
This set is the {\em language} of the grammar,
where we use the term ``language'' in the sense
of computational theory.
In the context of programming languages,
these legal token sequences are called ``programs''.
We also use the term {\em syntax rules} to refer
to the rules given by a BNF grammar,
and we use the term {\em syntactically correct} to refer
to token sequences that conform to the grammar rules.
Finally, when there is little chance for confusion,
we use the term {\em sentence} (in the sense of computational theory)
to refer to a finite length sequence of lexemes.\exx
If we had a set of grammar rules
for the Java programming language, for example,
then the set of all sentences that conform to this grammar
would be the set of all syntactically correct Java programs.\exx
{\bf In these notes,
we casually use the term {\em token}
to refer both to the symbolic name
of the abstraction (like `\verb'NUM'') used in a BNF rule
and to its corresponding {\em lexeme} (like `\verb'23'')
used in a program.}
In most instances, you should not have difficulty
understanding which meaning is intended.
\end{minipage}
\clearpage
