\begin{minipage}[t]{\sw}
\slidenumber
\LARGE
{\bf The Lambda Calculus -- OPTIONAL SECTION}\exx
\begin{SaveVerbatim}{\MYlambda}
<exp> ::= <SYMBOL>
<exp> ::= PROC LPAREN <SYMBOL> RPAREN LBRACE <exp> RBRACE
<exp> ::= DOT <exp> LPAREN <exp> RPAREN
\end{SaveVerbatim}
The following grammar defines a formal language
called ``the lambda calculus''.
This language plays an important role
in the foundations of computer science
(similar to Turing Machines).
The \verb'PROC' token is the string \verb'proc',
and the \verb'SYMBOL', \verb'LPAREN', \verb'RPAREN',
\verb'LBRACE', \verb'RBRACE', and \verb'DOT'
tokens are straight-forward -- see the examples below.\exx
\emm{\Large\LightBox{\MYlambda}}\exx
Consider the sentential form (remember what that means?) in this language
obtained from the second grammar rule,
where \verb's' replaces \verb'<SYMBOL>':
\begin{qv}
proc(s) { <exp> }
\end{qv}
The occurrence of the symbol \verb's' in this expression
is called a {\em variable declaration}
that {\em binds} all occurrences
of \verb's' that appear in \verb'<exp>'
unless some intervening declaration of the same symbol \verb's'
occurs in \verb'<exp>'.
We say that the expression \verb'<exp>'
is the {\em scope} of the variable declaration for \verb's'.\exx
\end{minipage}
\clearpage
