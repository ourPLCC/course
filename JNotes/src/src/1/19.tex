\begin{minipage}[t]{\sw}
\slidenumber
\LARGE
{\bf Parsers} (continued)\exx
\Large
\emm\LightBox{\MYlon}\vspace{1.5ex}\\
However, there are two grammar rules with \verb'<nums>'
as the LHS nonterminal.
PLCC generates the class name \verb'Nums' automatically
(by converting the first character of the LHS nonterminal name to uppercase),
but {\bf PLCC must generate a
{\em unique} Java class name for each grammar rule.}
We accomplish this by annotating the LHS nonterminal
on each of these lines with a Java class name
that is different from \verb'Nums'
and that distinguishes one from the other.
We modify these grammar rules with annotations as follows:
\begin{qv}
<nums>:NumsNode  ::= NUM <nums>
<nums>:NumsNull  ::=
\end{qv}
(Any Java class names are OK for these annotations,
but good naming conventions should prevail,
and the names must be unique
among the Java class names that PLCC generates.)
A colon is used to separate the nonterminal
from its annotated Java class name.
The RHS entries of these grammar rules are unchanged.\exx
With these modifications, PLCC generates
two new classes, \verb'NumsNode' and \verb'NumsNull'.
Both of these classes are declared to extend the \verb'Nums' class,
so an instance of a \verb'NumsNode' class, for example,
is also automatically an instance of its parent \verb'Nums' class.\exx
The \verb'NumsNode' class has a field named \verb'nums',
which is an instance of the \verb'Nums' class
(capitalize the first letter of \verb'<nums>').
The \verb'NumsNull' has no fields.
\end{minipage}
\clearpage
