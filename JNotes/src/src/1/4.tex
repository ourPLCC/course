\begin{minipage}[t]{\sw}
\slidenumber
\LARGE
{\bf BNF: (continued)}\exx
\emm{\Large\LightBox{\MYlon}}\exx

Every BNF formula has the form

\vspace{1ex}
\emm%
\begin{tabular}{@{}l@{\tt ~::=~}l}
{\em LHS} & {\em RHS} \\
\end{tabular}

\vspace{1ex}
The {\em LHS} (Left-Hand Side) of a BNF formula always has the form
\verb'<nonterm-symbol>' where \verb'nonterm-symbol' is an identifier
(usually letters, digits, and underscores).
PLCC requires that the first character
of this identifier be a lowercase letter.
A \verb'<nonterm-symbol>' expression is called a {\em nonterminal}.
In the example above,
the nonterminals are \verb'<lon>' and \verb'<nums>'.\exx
The {\em RHS} (Right-Hand Side) of a BNF formula
is a (possibly empty) ordered list
of token names and nonterminals.\exx
{\bf Notes:} The term {\em syntactic category} is sometimes used instead
of the term {\em nonterminal},
and the term {\em terminal} is sometimes used instead of {\em token name}.
Instead of using a token name such as \verb'LPAREN',
some languages use BNF formulas that use
the corresponding actual character string such as `\verb'(''.
In the examples on slide 1.2,
you can see that we have also skipped whitespace.
\end{minipage}
\clearpage
