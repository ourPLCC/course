\begin{minipage}[t]{\sw}
\slidenumber
\LARGE
{\bf Parsing (continued):}
\begin{enumerate}
\itemsep -0.5ex
\item [3.]
  \begin{enumerate}
  \item
    If the leftmost unmatched term in the sentential form is a {\bf token name},
    match it with the leftmost string in the unmatched sentence.
    [Here, {\em match} means that the token name
    in the sentential form describes exactly
    the leftmost string to be matched.
    For example, the token name {\tt LPAREN} matches the string `\verb'(''
    and {\tt NUM} matches the string `\verb'42''.]
    If there is no match, return failure.
    If there is a match, replace the leftmost token name in the
    sentential form with its matching string (its lexeme)
    from the unmatched sentence
    and remove the matched string from the unmatched sentence.
  \item
    If the leftmost unmatched term
    in the sentential form is a {\bf nonterminal},
    choose a rule from the grammar with this nonterminal as its LHS
    and {\em replace} the nonterminal with the (possibly empty) RHS
    of the chosen rule.
    [Which rule to choose depends on
    finding a rule that is most likely to complete the derivation.
    For some grammars, there is only one choice: these grammars
    are said to be {\em predictive}.  All of the grammars we
    use in this course are predictive.]
    If no rule can apply, return failure.
  \end{enumerate}
\item [4.]
  If the unmatched sentence is empty, return success:
  the target sentence has been successfully parsed.
  Otherwise, return failure.
\end{enumerate}
It is possible, for some grammars, that Step 3 loops indefinitely.
However, for all of the grammars that we encounter in this course,
this step exits in a finite number of iterations.
\end{minipage}
\clearpage
