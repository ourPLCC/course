\begin{minipage}[t]{\sw}
\slidenumber
\LARGE
{\bf Order of evaluation} (continued)\exx
Order of evaluation does not matter in languages without side-effects,
which makes functional languages immune to order of evaluation issues.
See \verb'http://en.wikipedia.org/wiki/Evaluation_strategy'
for more information about order of evaluation.\exx
Another way to avoid order of evaluation problems
is to require that all procedures have
at most one formal parameter.
In languages that use this approach,
coupled with with call-by-need,
there is never an ``order of evaluation'' issue
because there is never more than one actual parameter to evaluate.\exx
While you may think that a language with procedures having
only one formal parameter might be limited,
it's possible for such a language
to behave like having multiple formal parameters
using an approach called ``Currying'',
as employed in the Haskell programming language
-- named after Haskell Curry.
The following slide gives an example.
\clearpage
\end{minipage}
