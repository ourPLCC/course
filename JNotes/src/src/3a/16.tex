\begin{minipage}[t]{\sw}
\slidenumber
\LARGE
{\bf Language REF}\exx
A parameter passing approach
that evaluates actual parameters using value semantics
and that binds the formal parameters
to these actual parameter values
is called {\em call-by-value}.
This is what we use in languages \verb'V1' to \verb'V6'.\exx
In the language \verb'SET',
where bindings are to references instead of values,
we wrap the actual parameter values
into {\em new} references,
and these references are bound to the formal parameters.
Though denoted values are references in Language SET,
the parameter passing approach is still call-by-value.\exx
Considering the illustration in the previous slide,
Suppose we {\em want} a behavior
that binds the formal parameter \verb't'
to the {\em same} reference that is bound to \verb'x'
instead of a new reference containing a copy of the value.
The following diagram shows how the bindings
in the previous diagram change
when \verb't' is bound to the same reference as \verb'x':
\cfigw{3a.14.eps}{6in}
Such a parameter passing semantics is called
{\em call-by-reference}.
We explore call-by-reference next,
along with variants on this theme.
\end{minipage}
\clearpage
