\begin{minipage}[t]{\sw}
\slidenumber
\LARGE
{\bf Language REF} (continued)\exx
The term {\em L-value} refers
to a semantic entity that can be considered as a reference.
It's called an L-value because it is the sort of thing
that can appear to the {\em left} of the `\verb'=''
in a \verb'set' expression.
In Languages SET and REF,
a variable like \verb'x' can be considered as an L-value
(because variables are always bound to references)
but an expression like \verb'+(x,0)' can only be considered
as a value, never a reference.\exx
Whether we consider a semantic entity
as an L-value depends on where it occurs.
In the expression
{\Large
\begin{qv}
set x = 3
\end{qv}
}
the occurrence of \verb'x' is considered as an L-value.
On the other hand, in the expression
{\Large
\begin{qv}
set y = x
\end{qv}
}
the occurrence of \verb'x' is not considered as an L-value.\exx
In Languages SET and REF, only variables can be considered as L-values.
In Language SET, actual parameter expressions (even variables)
{\em in procedure applications} are never considered as L-values.
In Language REF, only actual parameter expressions that are variables
are considered as L-values.
\end{minipage}
\clearpage
