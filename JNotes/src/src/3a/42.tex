\begin{minipage}[t]{\sw}
\slidenumber
\LARGE
{\bf Order of evaluation}\exx
\Large
Let's examine the following example in Language SET (or REF):
\begin{qv}
let
  x = 3
in
  let
    y = {set x = add1(x)}
    z = {set x = add1(x)}
  in
    z
\end{qv}
Consider the inner \verb'let'.
We know that the right-hand side expressions
(here written inside curley braces for clarity)
are evaluated before their values are bound
to the left-hand variables.
Languages SET and REF {\em specify} the order
in which the right-hand side expressions are evaluated,
namely first to last.
(In Language V6, the RHS expressions in a \verb'let'
cannot produce side effects, so the order of evaluation
of RHS expressions wouldn't matter.
However, when side-effects are possible,
the order of evaluation does matter.)\exx
Suppose we did not specify this order of evaluation.
In the above example,
if the second \verb'set' were to be evaluated first,
then \verb'z' becomes 4 and \verb'y' becomes 5,
so the entire expression evaluates to 4 -- the value of \verb'z'.
If the order of evaluation were reversed,
the entire expression evaluates to 5.
Since Languages SET and REF specify the order of evaluation,
we have an unambiguous interpretation
of the value of the above expression.
In the absence of such a specification,
the value of this expression is ambiguous.
{\em Do you know what your favorite language does?}
\end{minipage}
\clearpage
