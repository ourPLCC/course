\begin{minipage}[t]{\sw}
\slidenumber
\LARGE
{\bf Language REF} (continued)\exx
To repeat:
\begin{itemize}
\item
  The parameter passing semantics
  that we have been using up to now
  is called {\em call-by-value}.
  In call-by-value semantics --
  which we have referred to as {\em value semantics},
  when an actual parameter expression
  in a procedure application is a variable,
  the procedure's corresponding formal parameter denotes
  a new reference to the expressed value
  of the actual parameter.
\item
  In {\em call-by-reference} semantics --
  which we have referred to as {\em reference semantics},
  when an actual parameter expression
  in a procedure application is a variable,
  the procedure's corresponding formal parameter denotes
  the {\em same reference} as the actual parameter.
\end{itemize}
The differences between call-by-value and call-by-reference semantics
only apply when the actual parameter expression is a variable.
When the actual parameter expression is not a variable,
the corresponding formal parameter denotes
a {\em new} reference to the expressed value
of the actual parameter, just as in Language SET.\exx
Observe that in \verb'let' and \verb'letrec' expressions,
we always use value semantics for the variable bindings.
This means that each LHS variable
in a \verb'let/letrec' expression always denotes
a new reference to the expressed value
of its corresponding RHS expression.
\end{minipage}
\clearpage
