\begin{minipage}[t]{\sw}
\slidenumber
\LARGE
{\bf Language SET} (continued)\exx
Let's look the three lines in the \verb'eval' method shown
on the previous slide:
{\Large
\begin{qv}
        Val val = exp.eval(env); // the RHS expression value
        Ref ref = env.applyEnvRef(var); // the LHS reference
        return ref.setRef(val);  // sets the ref and returns val
\end{qv}
}
Notice that the \verb'exp.eval(env)' expression
returns a \verb'Val' object (a {\em value}),
whereas the \verb'env.applyEnvRef(var)' expression
returns a \verb'Ref' object (a {\em reference}).
We use the terms {\bf value semantics} to refer
to obtaining the value of something
and {\bf reference semantics} to refer
to obtaining a reference to something.
The code above shows that {\em we use value semantics
for the RHS of a \verb'set' expression
and reference semantics for its LHS}.\exx
Because we need to modify the value denoted by
the LHS variable using \verb'setRef' (as shown above),
we must use reference semantics for the LHS.
Value semantics would have been useless here,
since we regard ``values'' -- instances of the \verb'Val' class --
as immutable: things that cannot be modified.\exx
In a \verb'set' expression, the LHS variable must exist
in the environment where the \verb'set' expression occurs,
otherwise we could find no reference to modify.
Contrast this to the LHS variables occurring in \verb'let' expressions.
A \verb'let' expression creates {\em new} bindings
to the LHS variables:
we don't {\em modify} these variables, we {\em create} them!
\end{minipage}
\clearpage
