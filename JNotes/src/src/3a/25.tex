\begin{minipage}[t]{\sw}
\slidenumber
\LARGE
{\bf Language REF} (continued)\exx
We want literals (\verb'LIT's) always to have value semantics.
For example, a `4' that appears in an expression should always evaluate
to the integer 4. Consider the following code in Language REF:
{\Large
\begin{qv}
define square = proc(x) set x=*(x,x)
define four = 4
{ .square(four) ; four }
\end{qv}
}
By the time the variable \verb'four' gets evaluated the second time
in the sequence expression, its value has changed to 16,
because Language REF uses reference semantics
for the actual parameter \verb'four'.
Thus the value of the sequence expression is 16.
Consider now what happens if we replace the sequence expression above
with the following:
{\Large
\begin{qv}
{ .square(4) ; 4 }
\end{qv}
}
Of course, this sequence expression evalautes to 4,
because Language REF uses value semantics for everything but variables.
However, languages such as FORTRAN IV (in the 1970s)
treated numeric literals (like `4')
as variables and used reference semantics
when passing them to procedures.
The equivalent sequence expression, if written FORTRAN IV,
would evaluate to 16.
Furthermore, subsequent statements such as
{\Large
\begin{qv}
IF 4 = 16 THEN CALL YIKES 
\end{qv}
}
(not really a legal FORTRAN IV statement)
would end up calling \verb'YIKES'. 
Yikes!
\end{minipage}
\clearpage
