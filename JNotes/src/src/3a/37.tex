\begin{minipage}[t]{\sw}
\slidenumber
\LARGE
{\bf Language NEED} (continued)\exx
Implementing call-by-need is easy, starting
from the call-by-name interpreter.
The principal change is to have a \verb'Val' field named \verb'val'
in the \verb'ThunkRef' class
that is used to memoize the value of the body of the thunk.
This field is initialized to \verb'null' when the thunk is created.
When the thunk's \verb'deRef' method is invoked,
it checks to see if the \verb'val' field has been memoized
({\em i.e.}, is non-null).
If so, the \verb'deRef' method simply returns the memoized value.
Otherwise, it evaluates the body of the thunk,
saves the value in the \verb'val' field (thereby memoizing it),
and then returns that value;
subsequent \verb'deRef' calls simply use the resulting memoized value.\exx
{\em In the NEED language,
the \verb'ThunkRef' constructor initializes
the \verb'val' field to \verb'null',
indicating that the thunk has not been memoized.
This field is modified when the thunk's \verb'deRef' method is called.}
\end{minipage}
\clearpage
