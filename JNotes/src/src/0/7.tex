\begin{minipage}[t]{\sw}
\slidenumber
\LARGE
{\bf Syntax and semantics} (continued)\exx
This course is about programming languages,
and particularly about {\em specifying} programming languages.
A programming language {\em specification} is a document that defines:
\begin{enumerate}
\item the lexical structure of the language (its tokens);
\item the syntax of the language; and
\item the behavior of a program when it is run.
\end{enumerate}
In particular,
we give examples of language specifications
that describe the lexical and syntax structure
of a number of languages
and how to implement their run-time behaviors (semantics).
We show how variables are bound to values,
how to define functions,
and how parameters are passed when functions are called.\exx
Because a program in a particular language must be syntactically correct
before its semantic behavior can be determined,
part of this course is about syntax.
But in the final anaysis, semantics is paramount.
\end{minipage}
\clearpage
