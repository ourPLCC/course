\begin{minipage}[t]{\sw}
\slidenumber
\LARGE
{\bf Tokens}\exx
Assume that we have a program written in some programming language.
(Think of languages such as C, Java, Python, and so forth.)
The first step in analyzing the structure of a program
is to examine its lexical structure: the ``atoms''.\exx
A program is, at the lowest level, a stream of characters.
But some characters are typically ignored
(for example, ``whitespace'', including spaces, tabs, and newlines),
while some characters group together to form things
that can be interpreted (for example) as integers, floats, and identifiers.
Some specific character sequences are meaningful in the language,
such as `\verb'class'' and `\verb'for'' in Java.
Some individual characters are meaningful,
such as parentheses, brackets, and the equals symbol,
while some pairs or characters are meaningful
such as `\verb'++'' and `\verb'<=''.
We use the term {\em token} to refer to such atoms.\exx
A {\em token} in a programming language is an abstraction
that considers a string of one or more characters in the character stream
as having a particular meaning in the language --
a meaning that is more than the individual characters
that make up the string.
The term {\em lexical analysis} refers
to the process of taking a stream of characters representing a program
and converting it into a stream of {\em tokens}
that are meaningful to the language.
\end{minipage}
\clearpage
