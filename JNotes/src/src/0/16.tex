\begin{minipage}[t]{\sw}
\slidenumber
\LARGE
{\bf Tokens} (continued)\exx
To use the PLCC tool set, follow these steps:
\begin{enumerate}
\itemsep -0.2ex
\item
    Create a working directory.
    This directory will be specific
    to the language you want to implement.
\item
    In this working directory,
    put your \verb'grammar' language specification file.
    As we describe later,
    you may also put additional \verb'grammar'-related files
    in this directory.
\item
    If you wish, create subdirectories containing test files
    that you can use to exercise your language implementation.
\item
    Run \verb'plccmk' in your working directory.
    This will create a \verb'Java' subdirectory
    containing Java source files for an interpreter
    for your language.
\item
    If \verb'plccmk' produces any PLCC or Java compile errors,
    edit your \verb'grammar' file to fix these errors
    and repeat the above step.
\end{enumerate}
The \verb'grammar' file is a text file
that defines the tokens of the language
using skip specifications and token specifications
as we have illustrated earlier.
The language specification file can contain
comments starting with a `\verb'#'' character
and continuing to the end of the line.
These comments are ignored by the PLCC tool set.
\end{minipage}
\clearpage
