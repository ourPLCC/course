\begin{minipage}[t]{\sw}
\slidenumber
\LARGE
{\bf Lexical Analysis, Syntax Analysis, and Semantic Analysis (continued)}\exx
The {\em semantics} (from a Greek word meaning ``meaning'')
of a programming language refers to
the rules used to determine the meaning
of a program written in the language.
Here ``meaning'' refers to
(a) whether the program makes sense, and 
(b) what the program does when it is run.
In languages such as English,
a sentence can make sense (such as ``the dog ate the bone'')
but it may not have any meaning in terms of what it ``does''.
Programs are expected to ``do'' something,
and what they ``do'' is part of their semantics.
{\em Semantic analysis} is the process
of applying these rules to a program written in the language
to determine its meaning.\exx
A {\em semantic analyzer} is a program or procedure
that carries out semantic analysis.
The input to a semantic analyzer is a parse tree.
The output is either a direct execution of the resulting program
(as defined by what the program does when it is run)
or some intermediate form
(such as machine code) that can be run at some other time.
Direct execution is called {\em interpretation},
whereas producing an intermediate form is called {\em compilation}.
We will use the interpretation approach in these notes,
though the techniques we describe apply as well
to a compilation approach.\exx
\end{minipage}
\clearpage
