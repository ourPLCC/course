\begin{minipage}[t]{\sw}
\slidenumber
\LARGE
{\bf Tokens} (continued)\exx
Writing a scanner is somewhat involved,
so our PLCC tool set produces
a Java scanner \verb'Scan.java' automatically
from a \verb'grammar' language specification file.
The PLCC tool set consists
of the program \verb'plcc.py' written in Python 3
along with a collection of Java support files.
This tool set works with with any system
that supports Python 3 and Java.\exx
The \verb'plcc.py' Python program
and the \verb'Std' subdirectory that contains its support files
are on the RIT Ubuntu lab systems in this directory:
{\Large
\begin{qv}
/usr/local/pub/plcc/src
\end{qv}
}
This directory also contains a shell script called \verb'plccmk'
that invokes the \verb'plcc.py' program with input
from a \verb'grammar' language specification file.
Normally you will run \verb'plccmk' in a directory
that contains this specification file.
The \verb'plcc.py' program produces a collection of Java programs
in a subirectory named \verb'Java'.
The \verb'plccmk' script then compiles these Java programs
(using the Java compiler).
When you are working on one of our Ubuntu lab systems,
you can simply run \verb'plccmk'
to process the various languages we will specify in this course.\exx
See the \verb'HOWTO.html' file in \verb'/usr/local/pub/plcc/tvf'
for {\em important information} about how
to set up your CS account environment
so that PLCC will be able to access the required program files
and library routines on CS servers and lab workstations.
\end{minipage}
\clearpage
