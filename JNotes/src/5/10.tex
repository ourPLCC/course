\begin{minipage}[t]{\sw}
\slidenumber
\LARGE
{\bf Language OBJ} (continued)\exx
Before we build an object from a base class,
we first build an object that is an instance
of the superclass of the base class.
This superclass object has its own environment, namely \verb'objectEnv'.
We then extend this superclass object environment
by adding bindings to the statics, fields, and methods of the class,
and use this extended environment
to create the instance of the base class.\exx
Since creating the superclass object may itself involve
creating an instance of {\em its} superclass,
object creation continues up the class hierarchy
until the top-level \verb'EnvClass' class is found,
at which point there is no further superclass object to create.\exx
At the top of the chain of superclass objects,
we add the identifier \verb'self'
to the environment of this top-level superclass object,
binding it to a reference to the (original)
base class object being created.
Since all of the objects created
by going up the superclass chain have environments
that ultimatelly extend the top-level object,
all of these objects can refer to the base class object being created
using the \verb'self' field identifier.
(In Java, we do the same thing using \verb'this' instead of \verb'self'.)
Methods declared in superclasses
that refer to \verb'self' will ``see'' the base class object,
allowing for dynamic dispatch of method calls,
an important feature of object-oriented languages.
We call this binding of \verb'self' a {\em deep} binding.
\end{minipage}
