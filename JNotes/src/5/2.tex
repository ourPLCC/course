\begin{minipage}[t]{\sw}
\slidenumber
\LARGE
{\bf Language OBJ}\\
\vspace{-1ex}\\
All classes of an \verb'OBJ' program belong to a {\em class hierarchy},
which is tree structure with an unnamed class at the root of the tree
and with program-created classes at the other nodes of the tree.\exx
In the class hierarchy,
a class \verb'X' that occurs as a child node of class \verb'Y'
in the class hierarchy
is called a {\em subclass} of \verb'Y',
and \verb'Y' is called a {\em superclass} of \verb'X'.
In this case, we also say that \verb'X' {\em extends} \verb'Y'.\exx
When an object of class \verb'X' is instantiated,
instances of each of the classes that lie
on the path from \verb'X' to the root of the tree
are created, and the combination of all those instances
is considered as the resulting object.\exx
If \verb'Y' is the superclass of \verb'X',
then an object \verb'x' created from class \verb'X' ``contains''
an object \verb'y' created from \verb'Y'.
The object \verb'y' is called the {\em parent} of \verb'x',
and likewise \verb'x' is called the {\em child} of \verb'y'.\exx
A class may also have \verb'static' variables
whose values are shared among all instances of the class.\exx
If \verb'x' is an object and \verb'f' is a field,
the expression \verb'<x>f' evaluates to the value bound to \verb'f'
in object \verb'x'.
In languages like C++ and Java,
this would be written instead as \verb'x.f'.
\end{minipage}
