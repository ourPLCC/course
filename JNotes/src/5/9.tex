\begin{minipage}[t]{\sw}
\slidenumber
\LARGE
{\bf Language OBJ} (continued)\exx
Although counter-intuitive,
objects are actually simpler than classes,
because an object is essentially a wrapper
for an instance of \verb'Env'!\exx
An \verb'ObjectVal' is a Java class
that extends the \verb'Val' class.
It has a single instance variable:
\begin{qv}
    public Env objectEnv;
\end{qv}
The \verb'new' operator in our source language
takes a class expression and returns
a Java \verb'ObjectVal' instance
that essentially couples the \verb'static' bindings of the class
with bindings for the class fields and methods.\exx
Since our language does not define
an explicit constructor in class expressions,
we initially bind object fields to (a reference to) \verb'nil'.\exx
The method variable names in the class definition are bound
to procedure closures that capture the environment
that includes bindings for the class \verb'static' variables
(from the \verb'staticBindings' field), 
along with bindings for the fields as described above.
The method closures are created as in \verb'letrec',
so they can refer to themselves recursively.
As with static and field definitions,
we disallow duplicate method variable names.
\end{minipage}
