\begin{minipage}[t]{\sw}
\slidenumber
\LARGE
{\bf Language V1} (continued)\exx
Three classes extend the \verb'Exp' class: they are
\verb'LitExp', \verb'VarExp', and \verb'PrimappExp'.
We'll start with \verb'LitExp'.
Here is the code part of the grammar file
that defines the \verb'eval' behavior of a \verb'LitExp' object.
(The \verb'eval' behavior can coexist with the \verb'toString' behavior
that we defined in Language V0,
but we do not show this here.)
\begin{qv}
LitExp
%%%
    public Val eval(Env env) {
        return new IntVal(lit.toString());
    }
%%%
\end{qv}
Remember that a \verb'LitExp' has a \verb'Token' field named \verb'lit'.
When we apply the \verb'toString()' method to this field,
we get the string of decimal digits
that came from the part of the program text we are parsing.
The \verb'IntVal' constructor converts this into a real Java \verb'int'
that becomes part of the \verb'IntVal' instance.
Obviously an environment doesn't have anything to do
with the value of a numeric literal --
a literal \verb'10' evaluates to the integer value 10
no matter what environment you have --
so the \verb'eval' routine for a \verb'LitExp'
simply returns the appropriate \verb'IntVal' object.
\end{minipage}
\clearpage
