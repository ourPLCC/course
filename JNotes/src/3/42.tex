\begin{minipage}[t]{\sw}
\slidenumber
\LARGE
{\bf Language V2} (continued)
\Large
\begin{qv}
if 1 then 3 else 4
  % => 3

if 0 then 3 else 4
  % => 4

if
  if 1 then 0 else 11
then
  42
else
  15
  % => 15

+(3, if -(x,x) then /(5,0) else 8)
  % => 11 (note that the /(5,0) expression is not evaluated!
\end{qv}
\LARGE
You must understand that
{\em an \verb'if' expression is an expression
and therefore it evaluates to a value.}
It is entirely unlike \verb'if' statements
in imperative languages such as Java and C++,
where the purpose of an \verb'if' statement
is to {\em do} one thing or another, not to return a value.
Also observe that an \verb'if' expression in our source languages
must have both a \verb'then' part and an \verb'else' part,
even though only one of these expressions ends up being evaluated.
\end{minipage}
\clearpage
