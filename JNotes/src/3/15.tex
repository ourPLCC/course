\begin{minipage}[t]{\sw}
\slidenumber
\LARGE
{\bf Language V0} (continued)\exx
Recall that repeating grammar rules
have fields that are Java lists.
For example, our Language V0 grammar has the following repeating rule:
\begin{qv}
<rands> **= <exp> +COMMA
\end{qv}
This rule says that the \verb'<rands>' nonterminal
can derive zero or more \verb'<exp>' entries,
separated by commas.
The following sentences would match the \verb'<rands>' nonterminal:
\begin{qv}
a,b,c     <-- <exp> COMMA <exp> COMMA <exp>
1, +(2,3) <-- <exp> COMMA <exp>
add1(x)   <-- <exp>
          <-- empty string
\end{qv}
The class defined by this rule is named \verb'Rands'.
Its RHS shows only one nonterminal \verb'<exp>',
so its corresponding Java class \verb'Rands'
has a field \verb'expList' of type \verb'List<Exp>'.\exx
You might wonder how we chose a name like ``rands''.
It's actually a shortened form of the term ``operands''.
In mathematics and in programming,
operands are the things being operated on.
For example, given the expression \verb'+(2,3)',
the operator is `\verb'+'' and its operands are \verb'2' and \verb'3'.
(Similarly, some language designs use the term ``rator''
as a shortened form of the term ``operator''.)
\end{minipage}
\clearpage
