\begin{minipage}[t]{\sw}
\slidenumber
\LARGE
{\bf Environment-Passing Interpreters} (continued)\exx
\Large
Most programming languages
have grammar rules defining an {\em expression}.
In Java, for example, an expression typically involves values
(like variables, integers, and the results of method calls) and operators
(like addition and multiplication).
Example Java expressions are `\verb'2+3'' and `\verb'foo(11) && toggle''.
In all of the languages we discuss in this class,
every program consists of evaluating expressions.
Such languages are called ``expression-based languages''.\exx
An {\em expressed value} is the value of an expression
as specified by the language semantics;
for example,
the expressed value of the Java expression `\verb'2+3'' is 5.
A {\em denoted value} is the value bound to a symbol.
Denoted values are internal to the interpreter,
whereas expressed values are values
of expressions that can be seen ``from the outside''.\exx
For a symbol, say \verb'x', you normally think
that the value of the expression \verb'x' is the same
as the denoted value of \verb'x'.
But what about a language such as Java?
In Java, the denoted value of a non-primitive variable
is a {\em reference} to an object,
whereas the expressed value of the variable
is the object itself.
This may seem like a subtle distinction,
but you will see its importance later.\exx
In summary, for a symbol,
its expressed value is what gets displayed when you print it
(using its \verb'toString' representation, for example),
and its denoted value is the value bound
to the symbol in its environment.
In our early languages, the denoted values and expressed values
will be the same.
In our later languages,
we will see why we need to separate denoted values
from expressed values to implement language features such as mutation.
\end{minipage}
\clearpage
