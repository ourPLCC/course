\begin{minipage}[t]{\sw}
\slidenumber
\LARGE
{\bf Language V5} (continued)\exx
The \verb'letrec' construct allows us
to define {\em mutually recursive procedures} --
two or more procedures that call each other in a recursive fashion.
Here's a classic example:
{\Large
\begin{qv}
letrec
  even? = proc(x) if zero?(x) then 1 else .odd?(sub1(x))
  odd? = proc(x) if zero?(x) then 0 else .even?(sub1(x))
in
  .even?(11) % => 0 (false)
\end{qv}
}
[{\bf Exercise} (not to hand in):
See if you can define the \verb'odd?()/even?()'
mutually recursive procedures
in Language V4 without \verb'letrec'.]\exx
Notice that we have used \verb'?'
in the variable names for the \verb'even?' and \verb'odd?' procedures
to suggest that these procedures should be considered
as {\em predicates} that return true (1) or false (0).
This is a lexical feature we have added
to Languages V5 and beyond.
\end{minipage}
\clearpage
