\begin{minipage}[t]{\sw}
\slidenumber
\LARGE
{\bf Environment-Passing Interpreters} (continued)\exx
You should also distinguish
between a {\em source language}
and its {\em implementation language}.
A source language is a language to be interpreted,
and its implementation language is the language
in which the interpreter is written.
(The term {\em defined language} is often used
to refer to a source language.
Similarly, the term {\em defining language} is often used
to refer to its implementation language.)\exx
In the rest of this course,
our source languages will be a collection of artificial languages used
to illustrate the various stages of language design,
and our implementation language will be Java.
Don't be disappointed by the term `artificial' here:
the languages we define have significant computational power,
and they serve to illustrate
a number of core ideas that are present in all programming languages.\exx
We start with a language we call ``Language V0'' --
think of this as ``Language Version Zero''.
Its \verb'grammar' specification file appears on the next slide.
\end{minipage}
\clearpage
