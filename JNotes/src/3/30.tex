\begin{minipage}[t]{\sw}
\slidenumber
\LARGE
{\bf Language V1} (continued)\exx
We {\em specify} that the expressions
in an \verb'evalRands' method call be evaluated
in first-to-last (or left-to-right, depending on how you are looking at it) order.
This is not necessarily the case for all programming languages.
In particular, 
see Slide 3a.38 {\em et seq.} for a more complete discussion
of order of evaluation.\exx
The \verb'evalRands' method returns a {\em list} of \verb'Val's.
In order to access these values easily and
to apply normal arithmetic operations to them,
we convert them into an {\em array} of \verb'Val' objects.
The utility method named \verb'toArray' in the \verb'Val' class
accomplishes this.\exx
{\em The expressions appearing in an application of a primitive
are called its {\bf operands},
also called {\bf actual parameters};
the values of these expressions are called its {\bf arguments}.}\exx
As a careful reader of these notes,
you will have observed that the class name \verb'Rands'
is derived from the word {\bf operands},
and that the name \verb'args' in the \verb'evalRands' method
is derived from the word {\bf arguments}.
\end{minipage}
\clearpage
