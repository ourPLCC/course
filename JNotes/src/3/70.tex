\begin{minipage}[t]{\sw}
\slidenumber
\LARGE
{\bf Drawing Envinronments} (continued)\exx
Since a \verb'let' expression is an expression,
it must evaluate to a value,
so a \verb'let' expression can occur
as the RHS of a binding in another \verb'let' expression.
Consider this example:\exx
\Large
\begin{verbbox}
let % outer
  x = 3
  y = 5
in
  let % inner
    x = let y=x in +(x,y) % LHS x is bound to 6
    y = let x=y in +(x,y) % LHS y is bound to 10
  in
    +(x,y) % evaluates to 16
\end{verbbox}
\emm\theverbbox\\
\LARGE
The environment defined by the outer \verb'let'
has one node with bindings for \verb'x' and \verb'y'
(to 3 and 5, respectively).
The inner let extends the environment of the outer let,
so the inner environment has two nodes.
The RHS expressions for the bindings of the inner \verb'let'
are evaluated in the environment defined by the outer \verb'let'.
Since each of these RHS expressions are themselves \verb'let' expressions,
each of them extends the environment defined by the outer \verb'let'.
A total of four environments get created:
the outer \verb'let' (extending the initial null environment),
the inner \verb'let' (extending the outer \verb'let'),
and one for each of the RHS expressions in the inner \verb'let'
(extending the outer \verb'let').
The next slide shows all of these environments.
\end{minipage}
\clearpage
