\begin{minipage}[t]{\sw}
\slidenumber
\LARGE
{\bf Language V3} (continued)\exx
In the third example, two new environments are created.
The outer \verb'let' binds \verb'x' to 3.
The inner \verb'let' binds \verb'x'
to the value of \verb'add1(x)'
and \verb'y' to the value of \verb'add1(x)'.
Both of these \verb'add1(x)' RHS expressions in the inner \verb'let',
are evaluated {\em using the outer [enclosing] environment}
which has \verb'x' bound to 3.
Thus \verb'add1(x)' evaluates to 4 {\em in both cases}.
Thus, in the inner environment,
\verb'x' is bound to 4 and \verb'y' is bound to 4,
so that the \verb'+(x,y)' expression evaluates to 8.
{\Large
\begin{qv}
let x = 3
in
  let
    x = add1(x)
    y = add1(x)
  in
    +(x,y)
  % => 8
\end{qv}
}
Observe also that the \verb'add1' primitive is {\em not} side-effecting.
This means that the expression \verb'add1(x)'
does {\em not} modify the value bound to \verb'x':
\verb'add1(x)' in our languages
behaves like \verb'x+1' and {\em not} like \verb'++x'
as you would find in languages such as C++ and Java.
\end{minipage}
\clearpage
