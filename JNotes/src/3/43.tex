\begin{minipage}[t]{\sw}
\slidenumber
\LARGE
{\bf Language V3}\exx
Language V3 is the same as Language V2
with the addition of a \verb'let' expression.
Here are the relevant grammar rules and abstract syntax representations:\exx
{\Large
\emm\begin{tabular}{@{}ll}
\verb'<exp>:LetExp' & \verb'::= LET <letDecls> IN <exp>'\\
  & \VerbBox{\fbox}{\verb'LetExp(LetDecls letDecls, Exp exp)'}\\
\verb'<letDecls>' & \verb'**= <VAR> EQUALS <exp>'\\
  & \VerbBox{\fbox}{\verb'LetDecls(List<Token> varList, List<Exp> expList)'}\\
\end{tabular}
}\exx
Notice that we need to change our lexical specification
to allow for token names \verb'LET', \verb'IN', and \verb'EQUALS'.
Here are the relevant lexical specifications:
{\Large
\begin{qv}
LET    'let'
IN     'in'
EQUALS '='
\end{qv}
\LARGE
Here is an example program in Language V3
that evaluates to \verb'7':
\Large
\begin{qv}
let
  three = 2
  four  = 5
in
  +(three, four)
\end{qv}
}
The purpose of a \verb'let' expression is to create an environment
with new variable bindings and to evaluate an expression
using these variable bindings.
\end{minipage}
\clearpage
