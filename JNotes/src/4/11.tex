\begin{minipage}[t]{\sw}
\slidenumber
\LARGE
{\bf Language TYPE0} (continued)\\
\Large
Our \verb'TYPE0' language has two new expressions:
a \verb'TrueExp' and a \verb'FalseExp',
with obvious concrete and abstract syntax:\exx
{\Large
\emm\begin{tabular}{ll}
\verb'<exp>:TrueExp' & \verb'::= TRUE'\\
    & \VerbBox{\fbox}{\verb'TrueExp()'} \\
\verb'<exp>:FalseExp' & \verb'::= FALSE'\\
    & \VerbBox{\fbox}{\verb'FalseExp()'} \\
\end{tabular}\exx
}
What should the \verb'eval' method return for these classes?
In Language SET, we used an \verb'IntVal' of zero to be false
and everything else to be true.
Once we have syntax for \verb'true' and \verb'false' expressions
in Language TYPE0,
we introduce corresponding semantics
using the \verb'Val' type \verb'BoolVal'
that represent the values of these expressions.
The \verb'BoolVal' class is simply a Java wrapper class
for Java \verb'boolean' values.
The code for the \verb'BoolVal' class is in the \verb'val' file.\exx
Once we have the \verb'BoolVal' class,
we can define the \verb'eval' behavior
for the \verb'TrueExp' and \verb'FalseExp' classes.
Here's the code for \verb'TrueExp' --
the code for \verb'FalseExp' is nearly identical:
{\Large
\begin{qv}
public Val eval(Env env) {
    return new BoolVal(true);
}
\end{qv}
}
Obviously the \verb'isTrue' Java method applied
to a \verb'BoolVal' object constructed with \verb'true'
must return true (in Java),
and similarly the \verb'isTrue' Java method applied
to a \verb'BoolVal' object constructed with \verb'false'
must return false.
The \verb'isTrue' method applied to {\em any} other value
(either an \verb'IntVal' or a \verb'ProcVal')
must throw an exception.
\end{minipage}
