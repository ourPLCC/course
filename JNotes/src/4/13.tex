\begin{minipage}[t]{\sw}
\slidenumber
\LARGE
{\bf Language TYPE1}\exx
\Large
Now that we have built the syntax for type expressions
and incorporated it into our grammar,
we begin to implement type checking.
Our strategy is to determine the type of any expression,
making type checks along the way.\exx
Our first step is to define \verb'Type'
as a {\em semantic} entity in our implementation.
We implement \verb'Type' as an abstract class
with three subclasses:
\begin{qv}
IntType
BoolType
ProcType
\end{qv}
These correspond to the three classes
that extend the \verb'Val' class.
Observe that \verb'Val' objects are used to evaluate expressions,
whereas \verb'Type' objects are used for type checking.
They serve similar purposes but are used in different ways.\exx
Neither \verb'BoolType' nor \verb'IntType' has any instance variables.
A \verb'ProcType' object has two instance variables,
as described here:\exx
\emm\VerbBox{\fbox}{%
\verb'ProcType(List<Type> paramTypeList, Type returnType)'}\exx
An expression is of type \verb'IntType'
if it evaluates to an integer.
Similarly for a \verb'BoolType'.\exx
When we evaluate an expression,
we use an environment to determine its expressed value.
When we are doing type checking,
we don't care what the expressed value actually is:
only its type matters.
\end{minipage}
