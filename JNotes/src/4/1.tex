\begin{minipage}[t]{\sw}
\slidenumber
\LARGE
{\bf Types}\exx
Our approach to types is to create a language that is
{\em strongly static typed}.
We add syntax to provide type information for values,
and every program is checked
for proper type matching prior
to evaluating the program.\exx
A {\em static typed} language is one in which type information 
can be determined at ``compile time'' rather than at runtime.
In this way, any type errors are found
before the program is evaluated.
Saying that a language is {\em strongly typed}
means that there are no type ``holes'' in the system:
the declared type of a symbol
(a variable or procedure parameter)
dictates the types of the values that can be bound to the symbol.
Everything is type checked, without exception,
and all expressions must conform to type rules.\exx
Our typed language is based on the \verb'SET' language.
In the \verb'SET' language,
only the integer value \verb'0' is considered false,
and all other values -- including \verb'ProcVal's -- are true
when used in conditional expressions.
The Java programming language, which is staticaly typed,
requires that the test expression
in an \verb'if' statement evaluates to a boolean.
We follow this example in our typed language
by introducing {\em primitive types}, \verb'int' and \verb'bool',
corresponding to integer and boolean expressed values, respectively.
To implement this, we add a new \verb'BoolVal' subclass
of the \verb'Val' class.
With this addition, we can require that the test expression
in an \verb'if' expression be an instance of \verb'BoolVal'.\exx
\end{minipage}
