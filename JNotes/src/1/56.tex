\begin{minipage}[t]{\sw}
\slidenumber
\LARGE
{\bf Static Properties of Variables} (continued)\exx
\Large
Here is the \verb'Foo' class given on the second previous slide:
\large
\begin{qv}
public class Foo {
   public static int y;
   public int z;
   public static void main(String [] args) {
     Foo f = new Foo(); // f is local to main
     int x = 1; // x is local in main
     y = 2; // y is static throughout in Foo
     f.z = 3; // z is known only within instances of Foo
   }
}
\end{qv}
\Large
Consider just the \verb'main' procedure in this class:
\large
\begin{qv}
   public static void main(String [] args) {
     Foo f = new Foo(); // f is local to main
     int x = 1; // x is local in main
     y = 2; // y is static throughout in Foo
     f.z = 3; // z is known only within instances of Foo
   }
\end{qv}
\Large
In this method, the identifiers
\verb'f' and \verb'x' are explicitly defined.
In these cases, we say that these identifiers {\em occur bound}
in the \verb'main' procedure.\exx
However, the variable \verb'y' is not defined anywhere
in the procedure \verb'main'.
In this case, we say that the identifier \verb'y' {\em occurs free}
in the \verb'main' method,
but it {\em occurs bound} in the enclosing class \verb'Foo'.
\end{minipage}
\clearpage
