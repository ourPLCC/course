\begin{minipage}[t]{\sw}
\slidenumber
\LARGE
{\bf Parsers} (continued)\exx
Here's a sample interaction using this trace feature
for Language \verb'LON',
with the output edited for the sake of readability:
{\Large
\begin{qv}
$ java -cp Java Rep -t
--> (14 6)
<lon>
| LPAREN "("
| <nums>:NumsNode
| | NUM "42"
| | <nums>:NumsNode
| | | NUM "6"
| | | <nums>:NumsNull
| RPAREN ")"
Lon@372f7a8d
\end{qv}
}
In this example,
the root of the parse tree is a \verb'Lon' object
whose \verb'nums' field is an instance of \verb'NumsNode'.
This instance in turn has a \verb'nums' field that is also
an instance of \verb'NumsNode'.
And finally, this instance has a \verb'nums' field
that is an instance of \verb'NumsNull'.
A \verb'NumsNull' object has no fields.
The trace also shows how each token is matched by the parser,
along with the lexeme
from the input program that matched the token.
The line `\verb'Lon@372f7a8d'' shows the result
of the parse, which is a \verb'Lon' object
representing the root of the parse tree.
The `\verb'@372f7a8d'' part represents the location
in memory where the \verb'Lon' object resides.
\end{minipage}
\clearpage
