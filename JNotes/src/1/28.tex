\begin{minipage}[t]{\sw}
\slidenumber
\LARGE
{\bf Parsers} (continued)\exx
\LARGE
\emm\LightBox{\MYlonGrammarOnly}\exx
PLCC also generates an interactive parser/evaluator
called \verb'Rep' that resides in the \verb'Java' subdirectory
along with the \verb'Parse' program.
\verb'Rep' executes a ``Read-Eval-Print'' loop
that displays a prompt,
{\em Reads} program input from standard input and parses it --
producing a parse tree,
{\em Evaluates} the parse tree ({\em i.e.}, ``runs'' the program),
and {\em Prints} the result obtained
from ``running'' the program.
Normally, this result is a representation
of the ``value'' of the program's parse tree:
an instance of the Java class defined
by the start symbol of the language's BNF grammar.\exx
How to determine the ``value'' of a parse tree
is the essence of {\em semantic analysis},
which we describe in detail
for each of the languages we consider in these notes.
In some cases, this value may be just
a re-cast version of the program text;
in others, it may be the numeric value of a function application.\exx
\end{minipage}
\clearpage
