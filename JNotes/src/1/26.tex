\begin{minipage}[t]{\sw}
\slidenumber
\LARGE
{\bf Parsers} (continued)\exx
\Large
%\emm\LightBox{\MYlonGrammarOnly}\exx
From a particular language defined by a language specification file,
PLCC generates a stand-alone parser with class name \verb'Parse'
that uses the Java classes created
from processing the lexical and syntax specification sections.
To run this parser in language directory \verb'LON', for example,
run the \verb'Parse' class file 
using `\verb'-cp Java'' on the \verb'java' command line
so the Java interpreter will know where
to find the \verb'.class' files.
This program displays a prompt `\verb'--> '',
after which it reads, from standard input,
``programs'' in Language \verb'LON' to parse.
(Standard input typically comes from your keyboard,
sometimes called the {\em console}.)
If the parse for a particular program succeeds,
the \verb'Parse' program displays the string \verb'"OK"'.
If the parse fails, it displays
an error message indicating how the parse failed.\exx
For example, in the \verb'LON' directory (after having run \verb'plccmk'),
enter the following command:
\begin{qv}
java -cp Java Parse
\end{qv}
With a standard input of `\verb'( 14 6 )'',
your input and output looks like this:
\begin{qv}
--> ( 14 6 )
OK
\end{qv}
With a standard input of `\verb'( 14 ( 6 )'',
your input and output looks like this:
\begin{qv}
--> ( 14 ( 6 )
%%% Parse error: Nums cannot begin with LPAREN
\end{qv}
You can also use the \verb'parse' shell script,
which is equivalent to running \verb'java -cp Java Parse'.
\end{minipage}
\clearpage
