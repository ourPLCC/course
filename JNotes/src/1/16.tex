\begin{minipage}[t]{\sw}
\slidenumber
\LARGE
{\bf Parsers} (continued)\exx
The PLCC tool set is designed
to make it easy to build a scanner and parser
for a large collection of programming languages.
While it is not ``industrial strength'' like many of the standard tool sets,
it is easy to learn.
The target language for PLCC is Java,
so all of the source code that PLCC generates
consists of self-contained Java programs.
To use PLCC, you only need to be comfortable
reading and writing programs in Java.\exx
To say that a parser returns only success or failure
is cold comfort,
since in most cases you want to ``run'' a program
once it parses successfully.
So a parser generally does one of two things:
it ``runs'' the program {\em as it carries out the parse algorithm}
(which is called {\em interpretation on the fly}),
or it produces some form of output
that can later be used to run the program {\em once the parse is complete}.
PLCC uses the latter approach,
since generating a parser is simpler
if it is divorced from any attempts to carry out
run-time behavior during the parse.\exx
The input to a PLCC parser is a token stream --
specifically, the tokens produced by an instance of the \verb'Scan' class
(see Slide Set 0).
The output of a PLCC parser is a {\em parse tree} of a program:
more specifically, it is a Java object
that is the root of the parse tree.
\end{minipage}
\clearpage
