\begin{minipage}[t]{\sw}
\slidenumber
\LARGE
{\bf Specifying Syntax}\exx
A {\em language},
in computer science theory,
is a set of strings,
where a {\em string} is a finite sequence of {\em symbols}
chosen from a given {\em alphabet}.
Computer science theory
deals in part with how to specify languages
using things such as Nondeterministic Finite Automata (NFAs),
Context-Free Grammars (CFGs),
and Turing Machines (TMs).\exx
A {\em programming language}
also defines a language in the theory sense,
except that the strings in a programming language are called {\em programs},
and the symbols are called {\em tokens}
(which we discussed in Slide Set 0).
The {\em syntax} of a programming language is a set
of rules used to specify the programs in the language.
Most programming languages use a context-free grammar
(or something close to it)
to specify what sequences of tokens belong to the language.\exx
A programming language also defines the run-time behavior
of a program, called its {\em semantics},
which we discuss at length later.\exx
Our first step is to describe a formal way in which
we can define the syntax of a programming language.
We start out with two simple examples
of languages that describe familiar data structures.
(These are not ``programs'' in the usual sense of the word,
but they will at least get us started with how
to specify syntax.)
\end{minipage}
\clearpage
