\begin{minipage}[t]{\sw}
\slidenumber
\LARGE
{\bf Parsers} (continued)\exx
{\Large
\emm\LightBox{\MYlonGrammarOnly}\exx
}
When invoked with command-line arguments,
the \verb'Parse' program reads the files given as filename arguments
and processes them as if they were entered
from standard input.\exx
Three command-line arguments have special meaning
when running the \verb'Parse' program:
\begin{itemize}
\parskip -0.5ex
\item The `\verb'-n'' command-line argument disables
      displaying the `\verb'--> '' prompt
      when reading programs from standard input.
\item The `\verb'-t'' command-line argument 
      toggles displaying a {\em parse trace}
      as programs are parsed.
      A parse trace is a text representation
      of the parse tree generated by the parser:
      an example of such a parse trace appears
      on Slide 1.30.
      It defaults to not displaying the parse trace.
\item The `\verb'-v'' command-line argument
      toggles displaying the names of the command-line files
      when processing them in left-to-right order.
      It defaults to not displaying the name.
\end{itemize}
The same command-line arguments pertain
to the \verb'Rep' program which we describe
beginning with the next slide.
\end{minipage}
\clearpage
