\begin{minipage}[t]{\sw}
\slidenumber
\LARGE
{\bf Parsers} (continued)\exx
Let's return to our tree example
(directory {\color{red}\verb'TREE'}).
Here is a modified grammar for a tree
as originally given on slide 1.12
that takes into account the PLCC requirements
for unique class names and the introduction
of fields for the \verb'NUM' and \verb'SYMBOL' tokens.
Notice the Java signatures for the class constructors
shown in boxes.\exx
\Large
\emm\begin{tabular}{ll}
\verb'<tree>:Leaf' & \verb'::= <NUM>'\\
    & \VerbBox{\fbox}{\verb'Leaf(Token num)'}\\
\verb'<tree>:Interior' & \verb'::= LPAREN <SYMBOL> <tree> <tree> RPAREN'\\
    & \VerbBox{\fbox}{\verb'Interior(Token symbol, Tree tree, Tree tree)'}\\
\end{tabular}\exx
\LARGE
But there is a problem
with the \verb'Interior' constructor signature.
Java does not allow multiple field names
or constructor formal parameters
with the same name:
specifically, in this case, there cannot be two field names
with the name \verb'tree'.
Here is what PLCC has to say about this when given
the above grammar (the line has been folded for clarity):
\begin{qv}
duplicate field name tree in rule RHS
LPAREN <SYMBOL> <tree> <tree> RPAREN
\end{qv}
\end{minipage}
\clearpage
