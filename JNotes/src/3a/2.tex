\begin{minipage}[t]{\sw}
\slidenumber
\LARGE
{\bf Language SET} (continued)\exx
We introduce the notion of a {\em reference},
something that {\em refers to} a mutable location in memory.
Instead of binding a variable directly to an expressed value,
we bind the variable to a reference containing an expressed value.
From a computer architecture point of view,
a reference is simply the address of a memory location:
the address never changes,
but the memory contents at the address can change.
\begin{quote}
\begin{tabular}{@{}l@{ $=$ }l}
Expressed value & Val $=$ IntVal+ProcVal \\
Denoted value & Ref(Expressed) \\
\end{tabular}
\end{quote}
To mutate a variable bound to a reference,
we change the {\em contents} of the reference;
the variable is still bound to the same reference.
\cfig{3a.2.eps}
\vspace{2ex}
The two right-hand diagrams depict the same environment.
The rightmost one uses a more compact representation.
\end{minipage}
\clearpage
