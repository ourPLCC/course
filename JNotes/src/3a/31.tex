\begin{minipage}[t]{\sw}
\slidenumber
\LARGE
{\bf Language NAME} (continued)\exx
We proceed to implement call-by-name.
We take our Language REF (call-by-reference)
implementation as a starting point.\exx
If an actual parameter is a literal expression (such as \verb'4'),
we bind the formal parameter to (a reference to) the literal value.
If an actual parameter is a procedure,
we bind the formal parameter
to (a reference to) the procedure's closure in the calling environment.
If an actual parameter is an identifier,
we bind the formal parameter
to the same reference as the actual parameter,
using reference semantics.\exx
If an actual parameter is any other kind of expression,
we bind the formal parameter
to a \verb'Ref' object that captures the expression
in the environment in which it was called
and that can be evaluated, when needed, by the called procedure.
We call such an object a {\em thunk}.\exx
Following our terminology
for defining {\em value semantics} and {\em reference semantics}
discussed earlier, we call this {\em name semantics}.
\end{minipage}
\clearpage
