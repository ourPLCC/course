\begin{minipage}[t]{\sw}
\slidenumber
\LARGE
{\bf Language NAME} (continued)\exx
In the presence of side-effects,
call-by-reference and call-by-name
may give different results.
Consider evaluating the following expression in Language REF:
{\Large
\begin{qv}
let
  x = 1
  f = proc(t,u)
        { set t = add1(t) ; u }
in
  .f(x, +(x,5))
\end{qv}
}
With call-by-reference,
when we evaluate the application \verb'.f(x, +(x,5))',
the formal parameter \verb't' in the definition of \verb'f' denotes
the same reference that \verb'x' denotes (initially containing 1),
whereas the formal parameter \verb'u' denotes a new reference
to the value 6 (using value semantics for the \verb'+(x,5)' expression).
Modifying \verb't' in the body of \verb'f' changes
the expressed value of \verb'x'
(because \verb't' and \verb'x' denote the same reference)
but does not change the expressed value of \verb'u'.
Thus this expression evaluates to 6.
\end{minipage}
\clearpage
