\begin{minipage}[t]{\sw}
\slidenumber
\LARGE
{\bf Language REF} (continued)\exx
Our \verb'REF' language has exactly the same grammar rules
as our \verb'SET' language.
The {\em only} differences are in the bindings
of formal parameters during procedure application.
As the discussion on the previous slides show,
we need to handle
actual parameters that are variables
differently from
actual paramemeters that are expressions.
The idea here is to define an \verb'evalRef' method
for instances of the \verb'Exp' classes
that takes care of how to translate
themselves into a reference:
for anything but a \verb'VarExp', 
\verb'evalRef' evaluates the expression
and returns a new reference to the value.
For a \verb'VarExp',
\verb'evalRef' returns the same reference
that the actual parameter denotes.\exx
So in the \verb'Exp' class,
the \verb'evalRef' method has the following {\em default} behavior:
{\Large
\begin{qv}
public Ref evalRef(Env env) {
    return new ValRef(eval(env)); // value semantics
}
\end{qv}
}
For the \verb'VarExp' subclass -- {\em and only for this class},
\verb'evalRef' is implemented as:
{\Large
\begin{qv}
public Ref evalRef(Env env) {
    return env.applyEnvRef(var); // reference semantics
}
\end{qv}
}
The \verb'evalRef' method in the \verb'VarExp' class
overrides the \verb'evalRef' method in the \verb'Exp' class.
In all other classes that extend the \verb'Exp' class,
the default definition in the parent \verb'Exp' class is used.
\end{minipage}
\clearpage
