\begin{minipage}[t]{\sw}
\slidenumber
\LARGE
{\bf Language {\color{red} NAME}} (continued)\exx
Now consider the same expression in Language NAME.
\Large
\begin{qv}
let
  x = 1
  f = proc(t,u)
        { set t = add1(t) ; u }
in
  .f(x, +(x,5))
\end{qv}
\LARGE
Consider what happens when we evaluate \verb'.f(x, +(x,5))'
using call-by-name:
The formal parameter \verb't' still denotes 
the same reference that \verb'x' denotes (initially containing 1),
but the formal parameter \verb'u' denotes
the (un-evaluated) expression \verb'+(x,5)'.\exx
The \verb'set' operation in the body of this procedure
increments the formal parameter \verb't';
but since \verb't' denotes the same reference as \verb'x',
the value of \verb'x' changes too, to two.
When we then evaluate the formal parameter \verb'u'
at the end of the \verb'proc' body,
we evaluate the expression \verb'+(x,5)' denoted by \verb'u'
{\em in the environment of the caller}.
Since this expression gets evaluated after the \verb'set',
and the value of \verb'x' is now 2,
the value of the expression \verb'+(x,5)'
(and thus the value returned by the procedure application)
is \verb'+(2,5)' or 7.
Thus the entire expression evaluates to 7.
\end{minipage}
\clearpage
