\begin{minipage}[t]{\sw}
\slidenumber
\LARGE
{\bf Aliasing} (continued)
\begin{qv}
let
  a = 3
  addplus1 = proc(x,y) {set x = add1(x) ; +(x,y)}
in
  .addplus1(a,a)
\end{qv}
Using call-by-reference,
when \verb'addplus1' is applied
to the actual parameters \verb'a' and \verb'a',
both formal parameters \verb'x' and \verb'y' of \verb'addplus1'
refer to the {\em same cell} as \verb'a'.
Therefore the \VerbBox{\fbox}{\verb'set x = add1(x)'} expression
is equivalent to the expression
\VerbBox{\fbox}{\verb'set a = add1(a)'}
which increments \verb'a' to 4,
and the next expression \verb'+(x,y)' is essentially equivalent
to the expression \verb'+(a,a)' which now evaluates to 8.
Thus the value of the program is 8.\exx
{\em Aliasing} occurs when two different formal parameters
refer to the same actual parameter.
As this example shows, aliasing can lead
to unexpected side-effects and should be avoided.
{\bf Of course, the best way to avoid problems such as
order of evaluation ambiguities and aliasing
is to avoid using langauges with side-effects!}
\end{minipage}
\clearpage
