\begin{minipage}[t]{\sw}
\slidenumber
\LARGE
{\bf Language NEED}\exx
The call-by-need parameter passing mechanism
is the same as call-by-name,
except that a thunk is called at most once,
and its value is remembered ({\em memoized}).\exx
Suppose a procedure with formal parameter \verb'x' is invoked
with actual parameter \VerbBox{\fbox}{\verb'set z = add1(z)'},
using call-by-need semantics.
As with call-by-name, the formal parameter \verb'x' is bound to the thunk
with \VerbBox{\fbox}{\verb'set z = add1(z)'} as its body,
in an enclosing environment
that we will assume has \verb'z' bound to (a reference to) the value 8.
When \verb'x' is referenced in the body of calling procedure,
its corresponding thunk is dereferenced,
producing a result of 9 for \VerbBox{\fbox}{\verb'set z = add1(z)'}.
The thunk now remembers ({\em memoizes}) the value 9,
and any further references to the formal parameter \verb'x'
in the body of the procedure will continue to evaluate to 9
without making any further changes to the variable \verb'z'.\exx
If call-by-name had been used in the above example,
additional evaluations of \verb'x' in the body of the procedure would result
in evaluating the body of the thunk for each such evaluation,
further modifying \verb'z'
and yielding values 10, 11, 12, and so forth.\exx
In both call-by-need and call-by-name
(and unlike call-by-reference),
if the formal parameter is never referenced
in the body of the procedure, the thunk is never evaluated.
Compared to call-by-name,
call-by-need reduces the overhead of repeatedly evaluating
a thunk when evaluating the corresponding formal parameter.
\end{minipage}
