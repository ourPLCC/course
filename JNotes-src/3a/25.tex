\begin{minipage}[t]{\sw}
\slidenumber
\LARGE
{\bf Language {\color{red} NAME}} (continued)
\Large
\begin{qv}
let
  x = 1
  f = proc(t,u)
        {
          set t = add1(t) ;
          u
        }
in
  .f(x, +(x,5))
\end{qv}
\LARGE
In call-by-name, the formal parameter \verb't' still refers
to the same cell that \verb'x' refers to (initially containing 1),
but the formal parameter \verb'u' is bound
to the (un-evaluated) expression \verb'+(x,5)'.\exx
Consider now what happens when \verb'.f(x, +(x,5))' is invoked.
The \verb'set' operation in the body of this procedure
increments the formal parameter \verb't';
but since \verb't' refers to the same cell as \verb'x',
the value of \verb'x' is changed, too, to two.
When the formal parameter u is referenced at the end of the \verb'proc',
the expression \verb'+(x,5)' is evaluated
{\em in the environment of the caller},
where \verb'x' was originally bound to 1.
Since this expression is evaluated after the \verb'set',
and the value of \verb'x' is now 2,
the value of the expression \verb'+(x,5)'
(and thus the value returned by the procedure)
is \verb'+(2,5)' or 7.
Thus this expression evaluates to 7.
\end{minipage}
