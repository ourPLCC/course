\begin{minipage}[t]{\sw}
\slidenumber
\LARGE
{\bf Language SET}\exx
In this version of our defined language,
we allow for the assignment of values to variables.
Languages that allow for the mutation of variables
are called {\em side-effecting}; such languages
are inherently more difficult to reason about,
which accounts for why functional programming has received
so much attention and also for why
it is so difficult to produce high-quality software
in most side-effecting programming languages.\exx
So far, our language has treated
denoted values (the things that variables are bound to)
as being the same as expressed values
(the values that an expression can have).
This is because a variable \verb'x', for example,
will always mean the same thing no matter where it appears
in its scope.\exx
When we add assignment, such as with
\begin{qv}
set x = add1(x)
\end{qv}
the meaning of \verb'x' on the LHS is different
from its meaning on the RHS of the assignment.
The RHS of this ``assignment'' represents an expressed value,
whereas the LHS represents a change in the denoted value.
In order to implement variable assignment,
we need to find a way
to disconnect denoted values from expressed values.
\end{minipage}
