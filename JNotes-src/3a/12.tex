\begin{minipage}[t]{\sw}
\slidenumber
\LARGE
{\bf Language SET} (continued)\exx
{\em A variable that occurs on the LHS
of a \verb'set' expression is not an expression}.
In particular, observe that the grammar rule
for a \verb'set' expression uses \verb'<VAR>', not \verb'VarExp',
to the left of \verb'EQUALS',
and we use reference semantics, not value semantics,
for such an occurrence,
because we are not treating the LHS variable as an expression.\exx
When we evaluate a \verb'VarExp'
(this {\em is} an expression),
either by itself or as part of another expression,
we use value semantics in Language SET.
This is clear once you examine the \verb'eval' semantics
for a \verb'VarExp':
{\Large
\begin{qv}
public Val eval(Env env) {
    return env.applyEnv(var); // value semantics!
}
\end{qv}
}
Even though all variables are bound to references in an \verb'Env',
the \verb'applyEnv' method de-references the reference bound to \verb'var'
to return its value semantics.\exx
Observe that expressed values (instances of \verb'Val')
get wrapped into references (instances of \verb'Ref')
when they are bound to variables in creating environments:
for example, when evaluating the RHS expressions in a \verb'let/letrec'
or when evaluating the actual parameter expressions
during a procedure application.
We evaluate these expressions
using value semantics in Language SET
before wrapping them into \verb'ValRef' objects.
\end{minipage}
\clearpage
