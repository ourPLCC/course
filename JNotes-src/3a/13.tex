\begin{minipage}[t]{\sw}
\slidenumber
\LARGE
{\bf Language SET} (continued)\exx
To illustrate how \verb'set' expressions evaluate to something,
consider the following \verb'let' expression in Language SET:
{\Large
\begin{qv}
let
  t = 3
  u = 42
  v = 0
in
  { set v = set u = set t = add1(t) ; +(t,+(u,v)) }
\end{qv}
}
This expression evaluates to 12.
The first expression in the body of this \verb'let'
(a sequence expression)
gets evaluated like this:
{\Large
\begin{qv}
    set v = { set u = { set t = add1(t) } }
\end{qv}
}
The innermost \verb'set' expression sets \verb't' to 4
and evaluates to 4.
Proceeding outwards,
the next \verb'set' expression sets \verb'u' to 4
(the value of the innermost \verb'set')
and evaluates to 4.
Finally, the outermost \verb'set' expression sets \verb'v' to 4
and evaluates to 4.
This final value gets discarded by the sequence expression,
but the last expression in the \verb'let' body uses
the modified values of \verb't', \verb'u', and \verb'v'.
\end{minipage}
\clearpage
