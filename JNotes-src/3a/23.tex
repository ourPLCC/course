\begin{minipage}[t]{\sw}
\slidenumber
\LARGE
{\bf Language NAME} (continued)\exx
In the presence of side-effects, call-by-name has interesting
properties that make it very powerful
but often difficult to reason about.
The language ALGOL 60 had call-by-value and call-by-name
as its parameter passing mechanisms.
ALGOL 60 had its greatest influence on languages
such as Pascal, C/C++, and Java.
Although call-by-name has been all but abandoned
by modern imperative (side-effecting) programming languages --
mostly because of its inefficiency,
it still plays a role in functional programming.
Scheme supports a form of call-by-name
by means of \verb'promise/force'.
Other functional languages such as Haskell use a variant, {\em call-by-need}.
We proceed to implement both call-by-name and call-by-need.
\end{minipage}
