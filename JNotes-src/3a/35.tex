\begin{minipage}[t]{\sw}
\slidenumber
\LARGE
{\bf Language NEED} (continued)\exx
You might have noticed in the \verb'code' file that an instance
of the class \verb'ValRORef' is constructed
by the \verb'evalRef' methods
in the \verb'LitExp' and \verb'ProcExp' classes.
This is a slight change from the \verb'NAME' language,
where the instances were \verb'ValRef's.
The \verb'RO' part stands for ``Read Only''.
The reason for doing this is that it doesn't make sense
for a literal or procedure to be modified.\exx
Consider, for example, the following code:
\begin{qv}
let
  f = proc(x) set x=add1(x)
in
  .f(3)
\end{qv}
One's intuition would be to think
of the procedure application \verb'.f(3)' as saying:
\begin{qv}
set 3=add1(3)
\end{qv}
But of course this doesn't make any sense.\exx
We have already seen that the \verb'setRef' method
in the \verb'ThunkRef' class throws an exception.
What we are now doing is to have this same behavior
for {\em any} actual parameter expression
except for a variable (where call-by-reference is the default).
\end{minipage}
