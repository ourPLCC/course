\begin{minipage}[t]{\sw}
\slidenumber
\LARGE
{\bf Language REF} (continued)\exx
Our \verb'REF' language has exactly the same grammar rules
as our \verb'SET' language.
The {\em only} differences are in the bindings
of formal parameters during procedure application.
As the discussion on the previous slide shows,
we need to handle
actual parameters that are variables
differently from
actual paramemeters that are expressions.
The idea here is to let instances of the \verb'Exp' classes
take care of how to translate
themselves into a reference:
for anything but a \verb'VarExp', 
we evaluate the expression and return a new reference to the value --
this can be handled by a single method in the \verb'Exp' class.
For a \verb'VarExp',
we return a reference to the same reference
that the actual parameter variable referred to.\exx
So in the \verb'Exp' class,
the \verb'evalRef' method has the following {\em default} behavior:
\Large
\begin{qv}
public Ref evalRef(Env env) {
    return new ValRef(eval(env));
}
\end{qv}
\LARGE
For the \verb'VarExp' subclass -- {\em and only for this class},
\verb'evalRef' is implemented as:
\Large
\begin{qv}
public Ref evalRef(Env env) {
    return env.applyEnvRef(var);
}
\end{qv}
\LARGE
The \verb'evalRef' method in the \verb'VarExp' class
overrides the \verb'evalRef' method in the \verb'Exp' class.
In all other classes that extend the \verb'Exp' class,
the definition in the parent \verb'Exp' class is used.
\end{minipage}
