\begin{minipage}[t]{\sw}
\slidenumber
\LARGE
{\bf Order of evaluation} (continued)\exx
You can see that both \verb'evalRands' and \verb'evalRandsRef'
use \verb'for-each' loops (also called enhanced \verb'for' loops)
to traverse and evaluate the expressions in the list of actual parameters.
The traversal is guaranteed by the Java API specification to be ``natural'',
in the sense that the elements of the list
are visited in ascending item number order.
Here is the code for \verb'evalRands' in the \verb'Rands' class:
\begin{qv}
public List<Val> evalRands(Env env) {
    List<Val> valList = new ArrayList<Val>();
    for (Exp e : expList)
        valList.add(e.eval(env));
    return valList;
}
\end{qv}
\end{minipage}
