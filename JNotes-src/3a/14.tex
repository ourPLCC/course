\begin{minipage}[t]{\sw}
\slidenumber
\LARGE
{\bf Language REF}\exx
A parameter passing semantics that evaluates actual parameters
and that binds the formal parameters
to these actual parameter values
is called {\em call-by-value}.
This is what is used in languages \verb'V1' to \verb'V6'.
In the language \verb'SET',
where bindings are to references instead of values,
the actual parameter values are turned
into {\em new} references,
and these references are bound to the formal parameters.\exx
Suppose we {\em want} the behavior illustrated
by the dashed line in the previous slide.
That is, if a variable that is passed
as an actual parameter to procedure,
the corresponding formal parameter is bound
to the {\em same} reference as the actual parameter variable,
not a new reference with a copied value.\exx
Such a parameter passing semantics is called
{\em call-by-reference}.
We explore call-by-reference next,
along with variants on this theme.
\end{minipage}
