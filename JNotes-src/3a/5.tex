\begin{minipage}[t]{\sw}
\slidenumber
\LARGE
{\bf Language SET} (continued)\exx
For our purposes, we want a reference to be
a Java object whose contents can be mutated.
When we bind a variable to a reference (its denoted value),
this binding does not change,
but the contents of the reference itself --
the thing it refers to -- can change.\exx
The \verb'Ref' abstract class embodies our notion of a reference --
the thing that a variable can be bound to.
For now, the only subclass of the \verb'Ref' class is
the \verb'ValRef' class.\exx
\emm\VerbBox{\fbox}{\verb'ValRef(Val val)'}\exx
The contents of a \verb'ValRef' object
is a \verb'Val',
and we say that such an object is a {\em reference to a value}.
(Recall that a \verb'Val' object is either
an \verb'IntVal' or a \verb'ProcVal' --
the only two \verb'Val' types that we currently have.)\exx
A \verb'Ref' object has two methods:
\begin{qv}
public abstract Val deRef();
public abstract Val setRef(Val v);
\end{qv}
In the \verb'ValRef' class,
The \verb'deRef' (dereference) method simply returns
the \verb'Val' object stored in the object's \verb'val' field,
and the \verb'setRef' (set reference) method
modifies the \verb'val' field
by changing it to the \verb'Val' parameter \verb'v'
(and returning the new \verb'Val' object as well).
\end{minipage}
