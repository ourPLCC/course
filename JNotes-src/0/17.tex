\begin{minipage}[t]{\sw}
\slidenumber
\LARGE
{\bf Tokens} (continued)\exx
Our first \verb'grammar' file examples contain only
token specifications.
Later, we will use \verb'grammar' files
to define language syntax, and then semantics.
For now, we concentrate only on token specifications.\exx
Each of these examples can be put in a file named \verb'grammar'
for processing by \verb'plccmk'.
These examples define what input should be skipped
and what input should be treated as tokens.
Comments in a language specification file begin
with the \verb'#' character and go to the end of a line.
{\Large
\begin{itemize}
\item
\begin{qv}
# Every character in the file is a token, including whitespace
token CHAR '.'
\end{qv}
\item
\begin{qv}
# Every line in the file is a token
token LINE '.*'
\end{qv}
\item
\begin{qv}
# Tokens in the file are 'words' consisting of one or more
# letters, digits or underscores -- skip everything else
skip NONWORD '\W+'  # skip non-word characters
token WORD '\w+'    # keep one or more word characters
\end{qv}
\item
\begin{qv}
# Tokens in the file consist of one or more non-whitespace
# characters, skipping all whitespace.
# Gives the same output as Java's 'next()' Scanner method.
skip WHITESPACE '\s+'  # skip whitespace characters
token NEXT '\S+'       # keep one or more non-space characters
\end{qv}
\end{itemize}
}
\end{minipage}
\clearpage
