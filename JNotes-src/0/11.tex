\begin{minipage}[t]{\sw}
\slidenumber
\LARGE
{\bf Tokens} (continued)\exx
\centerline{\psfig{figure=scanning.eps,width=5in}}
In this course, we describe a tool set called PLCC,
which stands for ``Programming Language Compiler Compiler''.
The PLCC tool set takes a programming language {\em specification}
and ``compiles'' it into a set of Java programs.
These Java programs implement the three phases of program interpretation:
lexical, syntax, and semantic analysis.\exx
A programming language specification in PLCC
is a text file normally named \verb'grammar'.
Each language that we describe in this course
has its own \verb'grammar' file.
A \verb'grammar' file has three sections:
the lexical specification,
the syntax specification,
and the semantic specification.
These sections appear in the \verb'grammar' file in this order:
first lexical, then syntax, and finally semantic.
A line consisting of a single '\verb'%'' separate the sections.\exx
We start by describing the structure
of the lexical specification section.
\end{minipage}
\clearpage
