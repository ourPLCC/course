\begin{minipage}[t]{\sw}
\slidenumber
\LARGE
{\bf Tokens} (continued)\exx
Writing a scanner is somewhat involved,
so we have provided you with a tool set
that produces a Java scanner automatically
from a file that specifies the tokens using regular expressions.
This tool set, named \verb'PLCC',
consists of a program written in Python 3
along with some support files.
\verb'PLCC' stands for a ``Programming Languages Compiler Compiler''.
You should be able to use this tool set with any system
that supports Python 3 and Java.
The \verb'plcc.py' Python program
and the \verb'Std' subdirectory that contains its support files
are on the Ubuntu lab systems in this directory:
\begin{qv}
/usr/local/pub/plcc/PLCC
\end{qv}
This directory also contains a shell script called \verb'plcc'
that runs the \verb'plcc.py' Python program,
along with a script called \verb'plccmk'
that also compiles the Java programs created by PLCC.\exx
If you are working on one of our Ubuntu lab systems,
you can simply run \verb'plcc' or \verb'plccmk'
to process the various languages we proceed to specify.\exx
If you are working on some other system (like a Mac or Windows),
copy the PLCC files and directories
into a suitable directory structure on that system
where you intend to do your work.
If you do so successfully, you may want to share your experiences
with others, but you are on your own.\exx
\end{minipage}
\clearpage
