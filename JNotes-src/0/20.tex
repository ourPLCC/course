\begin{minipage}[t]{\sw}
\slidenumber
\LARGE
{\bf Tokens} (continued)\exx
For example, consider a \verb'grammar' file with the following lines:
{\Large%
\begin{qv}
skip WHITESPACE '\s+'
NUM '\d+'        # one or more decimal digits
ID '[A-Za-z]\w*' # a letter followed by zero or more "word" chars
\end{qv}
}
In this case, a token that matches the \verb'ID' pattern
has token enum value named \verb'ID',
and a token that matches the \verb'NUM' pattern
has token enum value named \verb'NUM'.
Once the \verb'plcc' program is run on this specification,
the \verb'Token.java' file in the Java directory
will have a public inner \verb'enum' class named \verb'Val'
whose elements consist of the following identifiers
and associated patterns:
{\Large%
\begin{qv}
NUM ("\\d+")
ID ("[A-Za-z]\\w*")
\end{qv}
}
Any Java file that needs to use either of these \verb'enum' values can refer
to them symbolically as \verb'Token.Val.ID' and \verb'Token.Val.NUM'.\exx
The \verb'Scan' class method \verb'cur()' is designed to be {\em lazy}:
If a token needs to be gotten from the input stream,
calling \verb'cur()' gets the token and returns it.
If you call \verb'cur()' again, it returns the {\em same token}.
The \verb'adv()' method tells the \verb'Scan' object
to force the next \verb'cur()' call
to get and return the {\em next} token instead of returning the same token.
The \verb'cur()' call returns a special token \verb'$EOF'
if there are no more tokens left in the input stream.
\end{minipage}
