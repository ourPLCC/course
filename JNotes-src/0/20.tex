\begin{minipage}[t]{\sw}
\slidenumber
\LARGE
{\bf Tokens} (continued)\exx
When specifying tokens in a \verb'grammar' file,
you can omit the \verb'token' term (but not the \verb'skip' term).
This means that both
{\Large
\begin{qv}
WORD '\S+'
\end{qv}
}
and
{\Large
\begin{qv}
token WORD '\S+'
\end{qv}
}
are considered as equivalent.
We follow this convention in all of our subsequent examples.\exx
The important pieces of the \verb'Scan' class
are the constructor and two methods: \verb'cur()' and \verb'adv()'.
The \verb'Scan' constructor must be passed
a \verb'BufferedReader',
which is the input stream of characters to be read by the scanning process.
A \verb'BufferedReader' can be constructed from a \verb'File' object,
from \verb'System.in', or from a filename given in a \verb'String'.
The \verb'Scan' program reads characters
from this \verb'BufferedReader' object
line-by-line, extracts tokens from these lines
(skipping characters if necessary),
and delivers the current token with the \verb'cur()' method --
\verb'cur' stands for {\em cur}rent.\exx
\end{minipage}
\clearpage
