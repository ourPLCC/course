\begin{minipage}[t]{\sw}
\slidenumber
\LARGE
{\bf Tokens} (continued)\exx
The \verb'plccmk' script calls the \verb'plcc.py' translator
on the \verb'grammar' specification file.
The \verb'plcc' translator takes the \verb'Token.template' file
in the \verb'Std' directory
and modifies it using the grammar specification,
creating a Java source file \verb'Token.java'
in your \verb'Java' subdirectory.
It also copies the \verb'Scan.java' file in the \verb'Std' directory
into your \verb'Java' subdirectory.
The \verb'plccmk' script then compiles these two Java programs.
If you have made any mistakes in your \verb'grammar' file,
these mistakes may show up during translation
(with the \verb'plcc' program)
or during compilation of the Java source files.\exx
The important pieces of the \verb'Scan' class
are the constructor and two methods: \verb'cur()' and \verb'adv()'.
The \verb'Scan' constructor must be passed
a \verb'BufferedReader',
which is the input stream of characters to be read by the scanning process.
A \verb'BufferedReader' can be constructed from a \verb'File' object,
from \verb'System.in', or from a \verb'String'.
The \verb'Scan' program reads characters
from this \verb'BufferedReader' object
line-by-line, extracts tokens from these lines
(skipping characters if necessary),
and delivers the current token with the \verb'cur()' method --
\verb'cur' stands for {\em cur}rent.\exx
\end{minipage}
