\begin{minipage}[t]{\sw}
\slidenumber
\LARGE
{\bf Lexical Analysis, Syntax, and Semantics}\exx
The {\em semantics} of a programming language refers
to the behavior of a program written in the language
when the program is executed.
When a program produces some output, for example,
the specific output that is produced is defined
by the language semantics.
For example, the defined semantics of Java dictates
that the following Java program sends,
to the standard output stream, the decimal character 3
followed by a newline:
\Large
\begin{qv}
public class Div {
    public static void main(String [] args) {
        System.out.println(18/5);
    }
}
\end{qv}
\LARGE
This and other examples can be run in the \verb'JNotes/0Examples' directory.\exx
This course is about programming language syntax and semantics,
with an emphasis on semantics.
Syntax doesn't matter if you don't understand semantics.
\end{minipage}
