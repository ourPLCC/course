\begin{minipage}[t]{\sw}
\slidenumber
\LARGE
{\bf Lexical Analysis, Syntax, and Semantics}\exx
The {\em syntax} of a programming language refers
to the rules governing the structure
of a program written in the language.
A program is said to be {\em syntactically correct}
if it follows the syntax rules defining the language.
Every programming language has syntax rules,
and these rules are part of the programming language specification.\exx
Before the syntax of a programming language can be given,
the language specification must define the {\em tokens} of the language --
its ``atomic structure''.
Programming language tokens normally consist
of things such as numbers (``23'' or ``54.7''),
identifiers (``foo'' or ``x''),
reserved words (``for'', ``while''),
and punctuation symbols (``.'', ``['').
{\bf A language specification always starts with defining its tokens}.\exx
The process of reading input text
to isolate its tokens is called {\em lexical analysis}.
The term {\em scanning} refers to the activity of lexical analysis;
think of this as what you do
when you ``scan'' a line of printed text on a page
for the words (tokens) in the text.
A program or procedure that carries out lexical analysis
is called a {\em lexical analyzer}.
The terms {\em lexer} and {\em scanner}
are also used to refer to a lexical analyzer.
We will have more to say about these concepts later in this section.\exx
\end{minipage}
