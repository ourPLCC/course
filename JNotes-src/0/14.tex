\begin{minipage}[t]{\sw}
\slidenumber
\LARGE
{\bf Tokens} (continued)\exx
Whenever we are faced with two token specifications
whose regular expressions match a string of consecutive input characters,
we {\em always choose the one with longest possible match}.
For example, if the next characters in the input stream are
\begin{qv}
procedure
\end{qv}
the above specifications would match an \verb'ID' token
with lexeme \verb'procedure' as well as a \verb'PROC' token
with lexeme \verb'proc'.
Both of these specifications match the beginning (`\verb'proc'') of input,
but the \verb'ID' match is longer.\exx
If two or more token specifications
match the same input (longest match), we always
{\em choose the first specification line
in the \verb'grammar' file that matches} to identify the token.\exx
{\bf In summary, for a given input string, we always
\begin{enumerate}
\item choose the token specification with the longest match, and
\item among those with the longest match, 
      choose the first token specification 
      that appears in the \verb'grammar' file.
\end{enumerate}
We use the phrase {\em first longest match}
to describe these rules for token processing.}
\end{minipage}
\clearpage
