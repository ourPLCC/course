\begin{minipage}[t]{\sw}
\slidenumber
\LARGE
{\bf Tokens} (continued)\exx
For the languages we deal with in this course,
we specify the structure of tokens
by means of {\em regular expressions}.
(Other approaches use {\em deterministic finite automata},
also called {\em finite state machines}, to do the same thing.)
A regular expression is a formal description of a pattern
that can match a sequence of characters in a character stream.
For example, the regular expression `\verb'd'' matches the letter \verb'd',
the regular expression `\verb'\d'' matches any decimal digit,
and the regular expression `\verb'\d+'' matches
one or more decimal digits.
You should read the Java documentation for the \verb'Pattern' class
for information about how to specify patterns.\exx
When specifying tokens,
one of the first things we do is to specify
what characters do {\em not} appear in a token.
Typically, tokens do not have whitespace -- spaces, tabs, and newlines --
so these characters must be skipped.
We express these skipped characters using a notation such as
\begin{qv}
skip WHITESPACE '\s+'
\end{qv}
The regular expression `\verb'\s'' stands for ``space''
(the space character, a tab, or a newline),
and the regular expression `\verb'\s+'' stands for one or more spaces.\exx
We adopt one simplifying rule for the languages we discuss in this class:
{\em tokens cannot cross line boundaries}.
Be warned, however, that not all programming languages conform to this rule.

\end{minipage}
