\begin{minipage}[t]{\sw}
\slidenumber
\LARGE
{\bf Syntax and semantics} (continued)\exx
This course is about programming languages,
and particularly about {\em specifying} programming languages.
A programming language {\em specification} is a document that describes:
\begin{enumerate}
\item the lexical structure of the language: its tokens;
\item the rules that dictate how to write a program in the language
      so that it is syntactically correct; and
\item the behavior of a program when it is run.
\end{enumerate}
Along the way, we show how to implement
programming language behaviors (semantics)
with regard to specifics such as
how variables are bound to values,
how functions are defined,
and how parameters are passed when functions are called.\exx
Because a program in a language must be syntactically correct
before its semantic behavior can be determined,
part of this course is about syntax.
But in the final anaysis, semantics is paramount.

\end{minipage}
