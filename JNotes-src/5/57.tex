\begin{minipage}[t]{\sw}
\slidenumber
\LARGE
{\bf Language PROP} (continued)\exx
So far, a property is only defined in the conext of a class definition,
where it plays a role in object instantiation.
It turns out that the behavior of properties could be useful even outside
of the context of an object,
especially to manage access to variables defined in a \verb'let' expression.
To make this explicit, we create a \verb'letprop' construct
that has the following concrete syntax and abstract class structure:\exx
\large
\emm\begin{tabular}{ll}
\verb'<exp>:LetpropExp' & \verb'::= LETPROP <letpropDecls> IN <exp>'\\
    & \VerbBox{\fbox}{\verb'LetpropExp(LetpropDecls letpropDecls, Exp exp)'}\\
\verb'<letpropDecls>' & \verb'**= <VAR> EQUALS <prop>'\\
    & \VerbBox{\fbox}{\verb'LetpropDecls(List<Token> varList, List<Prop> propList)'}\\
\end{tabular}\exx
\LARGE
Consider this (somewhat strange) example:
\Large
\begin{qv}
let
  x = 3 
in
  letprop
    x = prop x : set x = 5
  in
    {set x = 42 ; x} % => 5
\end{qv}
\LARGE
This expression evaluates to 5.\exx
\end{minipage}
