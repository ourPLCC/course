\begin{minipage}[t]{\sw}
\slidenumber
\LARGE
{\bf Language OBJ} (continued)\exx
%\Large
%\emm\begin{tabular}{ll}
%\verb'<mangle>:Mangle' & \verb'**= RANGLE <exp> LPAREN <rands> RPAREN' \\
%    & \VerbBox{\fbox}{\verb'Mangle(List<Exp> expList, List<Rands> randsList)'}\\
%\end{tabular}\exx
\LARGE
Here is the code for \verb'eval(Val v, Env env)' in the \verb'Mangle' class:
\Large
\begin{qv}
public Val eval(Val v, Env env) {
    Iterator<Exp> expIter = expList.iterator();
    Iterator<Rands> randsIter = randsList.iterator();
    while (expIter.hasNext()) {
        // expIter.next() ProcExp to apply
        // v.env() is the environment in which to build the ProcVal
        Val p = expIter.next().eval(v.env());
        // evaluate this method's rands in env
        List<Val> valList = randsIter.next().evalRands(env);
        v = p.apply(valList);
    }
    return v;
}
\end{qv}
\LARGE
Each \verb'Exp' is evaluated in the environment defined
by the value \verb'v' (which must, perforce, be a class or object),
and this must evaluate to a \verb'ProcVal' (named \verb'p' in this code).
The \verb'valList' arguments to \verb'p' are 
evaluated in the outside environment \verb'env'
({\em not} in the environment defined by \verb'v')
from the corresponding \verb'Rands' object.
The procedure \verb'p' is then applied to these arguments,
and the resulting application becomes the new \verb'v'.
This is repeated until all of the \verb'Mangle' applications are performed.
The final value \verb'v' is returned
as the result of the \verb'EenvExp' expression.
\end{minipage}
