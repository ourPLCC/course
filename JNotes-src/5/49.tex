\begin{minipage}[t]{\sw}
\slidenumber
\LARGE
{\bf Language PROP} (continued)\exx
If a field named \verb'x' in a class has a property also named \verb'x',
referring to \verb'x' in the contect of this object
uses the \verb'PropRef' instead of the variable.
Consider this example:
\begin{qv}
define c =
  class
    static p = proc(t) set t=add1(t)
    field x
    method init = proc(x) {set <this>x=x ; self}
    property x = prop x : set x=$
    property y = prop x : set x=+($,$)
  end
define o = .<new c>init(3)
<o>{.p(x) ; x} % evaluates to 4
<o>{.p(y) ; x} % evaluates to 10
\end{qv} %%$
In the expression \verb'.p(x)', \verb'x' refers
to the property \verb'x'.
Since we are using call-by-reference parameter passing semantics,
the variable \verb't' is bound to this property,
and the expression \verb'set t=add1(t)' is evaluated
using this binding.
Remember that the RHS parts of property definitions
are all evaluated in the environment that includes
only statics, fields, and methods,
but not properties.
\end{minipage}
