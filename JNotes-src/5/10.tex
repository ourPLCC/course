\begin{minipage}[t]{\sw}
\slidenumber
\LARGE
{\bf Language OBJ} (continued)\exx
Before an object builds the environment that is specific
to the fields and methods of the class,
it creates an instance of the superclass of the class
and adopts the environment of the superclass object
(bound to the variable \verb'super')
before adding \verb'static', \verb'field', and \verb'method' bindings.
Since creating the superclass object may itself involve
creating an instance of {\em its} superclass,
object creation continues up the class hierarchy
until a parentless class is found,
at which point there is no further \verb'super' object to create.\exx
At the top of the chain of \verb'super' objects,
the identifier \verb'self' is bound
to a reference to the object being created.
In this way, the object can refer to itself
through the \verb'self' identifier.
(In Java, we call it \verb'this' instead of \verb'self'.)
Methods declared in superclasses
that refer to \verb'self' will ``see'' the original object,
allowing for dynamic dispatch of method calls,
an important feature of object-oriented languages.
We call this a {\em deep} binding.\exx
As objects are created up the superclass chain,
the identifier \verb'this' is bound
to the object created by that class.
We call this a {\em shallow} binding.
\end{minipage}
