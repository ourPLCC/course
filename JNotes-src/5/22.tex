\begin{minipage}[t]{\sw}
\slidenumber
\LARGE
{\bf Language OBJ} (continued)\exx
The static environment of a class ultimately ends up extending
the top-level program environment,
not the environment in which the class is defined.
Consider the following code:
\Large
\begin{qv}
define x = 3
let
  x = 5
in
  <class end>x
\end{qv}
\LARGE
In this example,
the class is defined in the \verb'let' environment,
but its static environment
extends the top-level environment,
not the \verb'let' environment,
so the value of this expression is 3, not 5.\exx
There are situations in which we may want
to retrieve the value of a variable in the ``local'' environment
in which the class is defined
and not in the static environment of the class.
To do so, we predefine a static ``variable'' \verb'!@'
(called ``bang-at'') in every class
and bind it to an object that captures the (local) environment
in which the class is defined.
This binding becomes part of the static environment of the class.
The token `\verb'!@'' is not really a variable,
so it cannot appear in the LHS of an assignment.
Its principal use is in expressions of the form
\Large
\begin{qv}
<!@>exp
\end{qv}
\LARGE
which evaluates to the value of the expression \verb'exp'
in the local environment.
\end{minipage}
