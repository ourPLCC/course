\begin{minipage}[t]{\sw}
\slidenumber
\LARGE
{\bf Language OBJ} (continued)\exx
Keep in mind that the following two expressions are not equivalent:
\begin{qv}
.<c>f(...)
<c>.f(...)
\end{qv}
In the first expression, \verb'f' is evaluated
in the static environment of \verb'c'.
Then the procedure bound to \verb'f' is applied
to the actual parameters \verb'(...)'
which are evaluated in the current environment,
{\em not} in the static environment of \verb'c'.\exx
In the second expression, the entire expression
\verb'.f(...)' is evaluated in the static environment of \verb'c',
which means that the actual parameters \verb'(...)'
are also evaluated in this static environment.\exx
In the example on the previous slide,
if the final expression was \verb'<c>.f(x)' instead of \verb'.<c>f(x)',
it would evaluate to 3,
since \verb'x' is bound to 3 in the static environment of \verb'c'.\exx
To clarify, the above two expressions can be re-written as follows:
\begin{qv}
.{<c>f}(...)
<c>{.f(...)}
\end{qv}
\end{minipage}
