\begin{minipage}[t]{\sw}
\slidenumber
\LARGE
{\bf Language PROP} (continued)\exx
The \verb'C#' language championed by Microsoft
solves these problems using the notion of a {\em property}.
A property acts like a field
but it provides built-in getter and setter code.
When the field is {\em accesssed}, the getter code is executed;
when the field is {\em assigned to} with a \verb'set' statement,
the setter code is executed.\exx
Here is the same class as described on the previous slide,
with a property instead of a getter and setter:
\Large
\begin{qv}
define c = 
  class
    field x
    property x = prop x:set x=$
  end
define cc = new c
<cc>set x = 5 % ok - sets the field value to 5
<cc>x         % returns 5
\end{qv} %%$
\LARGE
The property \verb'x' shadows the field \verb'x'.
This means that the field \verb'x' cannot be accessed directly
except through the property.\exx
The \verb'PROP' language is based on the \verb'OBJ' language
with the addition of call-by-reference semantics
and {\em properties}, as we proceed to describe.
\end{minipage}
