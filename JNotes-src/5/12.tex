\begin{minipage}[t]{\sw}
\slidenumber
\LARGE
{\bf Language OBJ} (continued)\exx
A parentless class is an instance of the Java \verb'EnvClass' class,
which has one instance variable:
\Large
\begin{qv}
    public Env staticEnv;
\end{qv}
\LARGE
Here is the code for the constructor,
which is called when processing the \verb'<ext>:Ext0' grammar rule.
\Large
\begin{qv}
public EnvClass(Env env) {
    // the static environment of this class
    staticEnv = env;
}
\end{qv}
\LARGE
As noted earlier, the static environment
of a parentless class is the top-level program environment,
which is passed into the \verb'Env env' parameter
when the class is created.
(See the \verb'Ext0' class for details.)\exx
A standard class (an instance of the Java \verb'StdClass' class)
has a similar constructor,
except that it builds on the environment of its superclass
as described earlier.
A standard class defines bindings
for the variables \verb'myclass' and \verb'superclass',
whose values are self-explanatory.
You can find the code for this constructor in the file \verb'class'.
\end{minipage}
