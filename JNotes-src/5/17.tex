\begin{minipage}[t]{\sw}
\slidenumber
\LARGE
{\bf Language OBJ} (continued)\exx
The \verb'makeObject' method for \verb'EnvClass' is simple,
since it's always the last superclass object that needs to be constructed.
It extends the environment in which the \verb'EnvClass' is created
({\em not} the top-level environment)
with a single field binding of \verb'self'
to (a reference to) the object being created.
\Large
\begin{qv}
    public ObjectVal makeObject(Ref objRef) {
        // start with the static environment of this class
        Env env = staticEnv;
        // add the field binding 'self' to refer to this object
        Bindings fieldBindings = new Bindings();
        fieldBindings.add("self", objRef);
        env = env.extendEnvRef(fieldBindings);
        return new ObjectVal(env);
    }
\end{qv}
\LARGE
Such an \verb'ObjectVal' can access the static environment
of this class, which is the top-level program environment.
\end{minipage}
