\begin{minipage}[t]{\sw}
\slidenumber
\LARGE
{\bf Language PROP}\exx
In many object-oriented programming languages,
the fields of an object can be made {\em private} --
that is, inaccessible outside of the object's methods.
For private fields, special publically accessible methods
can be used to retrieve the field values
or to modify them.
These methods are often called {\em getters} and {\em setters},
respectively.\exx
Suppose, for example, we provided a special designator called \verb'private'
that served to identify a field whose value was inaccessible outside
of the class methods.
Consider the following code:
\Large
\begin{qv}
define c =
  class
    private x
    method get_x = proc() x
    method set_x = proc(v) set x = v
  end
define cc = new c
.<cc>set_x(5) % ok - sets value of x to 5
.<cc>get_x()  % ok - returns 5
<cc>x            % illegal - x is private
\end{qv}
\LARGE
While this sort of code is common,
there are two problems with this approach.
The first is that every private field we want to access
needs a getter and a setter.
The second is that code such as
\verb'<cc>set x = 5'
does not work, but is easier to write and understand than
\verb'.<cc>set_x(5)'.
\end{minipage}
