\begin{minipage}[t]{\sw}
\slidenumber
\LARGE
{\bf Language OBJ} (continued)\exx
As described on slides 9 and 10,
as an object is created,
the \verb'self' identifier is bound
to the \verb'objRef' reference that is passed to
the \verb'makeObject' method in \verb'EnvClass'.
But how can this binding take place
when the object has not yet been completely created?
We do this by creating a Java \verb'ValRef' object named \verb'objRef',
with a dummy initial value (\verb'nil', to be precise),
and pass this to the \verb'makeObject' method.\exx
The \verb'objRef' is passed up the superclass chain
through successive \verb'makeObject' calls.
When \verb'makeObject' reaches the \verb'EnvClass' Java class,
the \verb'self' identifier is put into the object environment,
bound to \verb'objRef'.
After the original object has been completely created --
that is, after all of the \verb'makeObject' calls have returned,
\verb'objRef' is rebound to the newly created object
with a call to \verb'setRef'.
Here is the complete \verb'eval' code for a \verb'new' expression
in the \verb'NewExp' part of the \verb'code' file: 
\Large
\begin{qv}
    public Val eval(Env env) {
        // get the class from which this object is created
        Val val = exp.eval(env);
        // create a reference to a dummy value (nil)
        Ref objRef = new ValRef(Val.nil);
        // let the class create the object
        ObjectVal objVal = val.makeObject(objRef);
        // set the reference to the newly created object
        return objRef.setRef(objVal);
    }
\end{qv}
\end{minipage}
