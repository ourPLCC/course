\begin{minipage}[t]{\sw}
\slidenumber
\LARGE
{\bf Language INFIX} (continued)\exx
We borrow the syntax of \verb'let' expressions in language \verb'V4'
to create a new syntactic category called a \verb'block',
with the following grammar rules:
\Large
\begin{qv}
<factor>:BlockFactor ::= <block>
<block>              ::= LBRACE <blockDecls> <exp> RBRACE
<blockDecls>         **= DEF <VAR> EQUALS <exp> SEMI
\end{qv}
\LARGE
As shown in the \verb'INFIX' code file,
the evaluation semantics of a \verb'<blockDecls>' rule is
to create bindings of the LHS variables (\verb'<VAR>')
to the values of their RHS values (\verb'<exp>'),
adding the bindings one-by-one as in top-level defines.
Here is an example that evaluates to 8, once full evaluation
semantics for the language has been implemented.
\Large
\begin{qv}
{
  def x=3;
  def y=5;
  x+y
}
\end{qv}
\LARGE
We use semicolons to terminate
each of the block's variable definitions (using \verb'def').
As shown on the next slide,
\verb'block' is also used to define the body of a procedure.\exx
The evaluation semantics of a \verb'block' is
to add the \verb'def' bindings to a new environment
and to evaluate the body of the block in this extended environment.
This is similar to what a \verb'let' expression does in language \verb'V4'.
\end{minipage}
