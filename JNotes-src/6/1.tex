\begin{minipage}[t]{\sw}
\slidenumber
\LARGE
{\bf Language INFIX}\exx
In all of our languages so far, the following primitive operations --
addition (\verb'+'),
subtraction (\verb'-'),
multiplication (\verb'*'),
and division (\verb'/') --
have grammar rules that apply these primitives in {\em prefix} form,
where the operator occurs before the operands.
However, most programming languages use {\em infix} mathematical notation
for these operations,
so that instead of writing (as we would in \verb'V6', for example)
\begin{qv}
+(x, *(4,y))
\end{qv}
one would write
\begin{qv}
x+4*y
\end{qv}
It turns out that grammar rules that support infix notation
are more complicated than the prefix notation we have been using,
but not enormously more so.
We proceed to illustrate this in our language \verb'INFIX'.\exx
\end{minipage}
