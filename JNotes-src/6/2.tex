\begin{minipage}[t]{\sw}
\slidenumber
\LARGE
{\bf Language INFIX}\exx
A naive attempt to define grammar rules that support infix operations might be
to replace our \verb'PrimappExp' grammar rule with something like this:
\begin{qv}
<exp>:PrimappExp  ::= <exp>arg1 <prim> <exp>arg2
\end{qv}
Unfortunately, this won't even pass the PLCC grammar rules checker,
since the grammar rule is left recursive:
the rule has the nonterminal \verb'<exp>' on its LHS,
and the same nonterminal \verb'<exp>' appears
on the left of its RHS.
Left recursive grammar rules are not allowed in LL1 grammars,
and PLCC expects only LL1 grammars.\exx
Even if we were to ignore the left recursive issue,
there's a basic semantic problem with infix expressions
called {\em associativity}.
Specifically, how do we deal with an expression like this?
\begin{qv}
1-2+3
\end{qv}
Should this be interpreted with \verb'arg1' being \verb'1'
and \verb'arg2' being the expression \verb'2+3'
(with the \verb'<prim>' being \verb'SUBOP'),
or should \verb'arg1' be the expression \verb'1-2'
with \verb'arg2' being \verb'3'
(with the \verb'<prim>' being \verb'ADDOP')?
In other words, if we were to fully parenthesize this expression,
should it be `\verb'1-(2+3)'' or `\verb'(1-2)+3''?
Mathematically, these two interpretations are not the same,
but both interpretations would satisfy our grammar rule,
and PLCC would not know which interpretation to choose.
\end{minipage}
