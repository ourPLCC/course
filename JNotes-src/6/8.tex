\begin{minipage}[t]{\sw}
\slidenumber
\LARGE
{\bf Language INFIX} (continued)\exx
Unlike \verb'let' expressions,
the environments created by \verb'def' variable definitions in a block
are implemented as in top-level defines.
This makes it possible for the RHS of a \verb'def'
to refer to variables defined in the same block.\exx
A \verb'proc' definition in \verb'INIFX' is similar to \verb'V4',
except that the body of the procedure is a block instead of an \verb'exp',
meaning that it can have its own ``local'' variables using \verb'def's.
Here's an example:
\begin{qv}
proc(x) {def y=3; def z=add1(y); x+y+z}
\end{qv}
A procedure application in \verb'INIFX' is expressed
in exactly the same way as in \verb'V4'.
The following program evaluates to 120:
\begin{qv}
{
  def f = 
    proc(x) {if x then x*.f(sub1(x)) else one endif};
  def one = 1;
  .f(5)
} ;
\end{qv}
Observe that \verb'f' can refer to itself recursively,
and even refer to the variable \verb'one' before it is defined,
since the \verb'def' semantics of a \verb'block'
behave as in top-level \verb'define's.
\end{minipage}
