\begin{minipage}[t]{\sw}
\slidenumber
\LARGE
{\bf Logic Programming Puzzles}\exx
The idea is to define what it means for there to be a {\em path}
from one state to another.
First, we consider every crossing
\begin{qv}
cross(X,P,Y)
\end{qv}
as a path from \verb'X' to \verb'Y':
\begin{qv}
path(X,Y) :- cross(X,_,Y).
\end{qv}
Then we develop the transitive closure of the relation \verb'path'
\begin{qv}
path(X,Z) :- path(X,Y), cross(Y,_,Z).
\end{qv}
to get all paths.
Notice that the second clause in the RHS just needs
the \verb'cross' premise
instead of the more general \verb'path(Y,Z)' (why?).\exx
The query
\begin{qv}
path(s0000, s1111)?
\end{qv}
will produce a result if and only if there is a path
from state \verb's0000' (everyone on the starting side)
to state \verb's1111' (everyone on the ending side).
You can check that this is the case!
\end{minipage}
