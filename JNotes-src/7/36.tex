\begin{minipage}[t]{\sw}
\slidenumber
\LARGE
{\bf Exception Handling} (continued)\exx
When an exception is thrown --
always by name, and never anonymously --
the name is looked up in the exception environment
and the handler is applied,
returning its value to its saved continuation instead
of the continuation in which the exception is thrown:
\large
\begin{qv}
ThrowExp
%%%
    public ACont eval(Env env, Env xenv, Cont vcont) {
        HandlerVal handler =
            xenv.applyEnv(var.toString()).handlerVal();
        return handler.apply(rands.expList, env, xenv,
                             handler.xenv, handler.vcont);
    }
%%%
\end{qv}
\LARGE
Notice that evaluating the handler's actual parameters
or its body
may result in throwing additional exceptions,
which can result in a cascade
of exception handling, as shown on the following slide.
Notice, too, that when the handler body is evaluated,
its exception environment is the one
in which the \verb'catch' expression is evaluated.
This means if an exception is thrown
when evaluating a handler body,
its handler is searched for outside
of the \verb'catch' expression handlers named in the expression.
If evaluating an expression results
in a \verb'throw' that refers to a handler name
that is not in the current exception environment,
the value of the expression is undefined.
\end{minipage}
