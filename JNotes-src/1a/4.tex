\begin{minipage}[t]{\sw}
\slidenumber
\LARGE
{\bf PLCC -- Tokens}\\
The PLCC toolset uses the skip and token specifications
to create source files \verb'Token.java' and \verb'Scan.java'.\exx
The \verb'Token.java' file defines the \verb'Token.Val' \verb'enum' class --
one \verb'enum' for each skip and token specification name.
It also defines the structure of a \verb'Token' object
consisting of a \verb'val' field of type \verb'Token.Val',
a \verb'str' field of type \verb'String',
and a \verb'lno' field of type \verb'int'.
The \verb'str' field consists of the characters in the input stream
that match the token specification -- the token's {\em lexeme} --
and the \verb'lno' field contains the line number on the input stream
where the token appears.
The \verb'str' field always contains at least one character.\exx
The \verb'Scan.java' file defines an object
that is constructed from an input stream (a \verb'BufferedReader').
The \verb'cur()' method delivers the current \verb'Token' object to its client
(skipping over strings that match skip specifications),
and the \verb'adv()' method advances to the next token.
Multiple calls to \verb'cur' without any intervening calls to \verb'adv'
returns the same token repeatedly.\exx
Upon end of input, \verb'cur' returns a special \verb'Token' object
whose \verb'val' field is \verb'$EOF'
and whose \verb'str' field is \verb'EOF'.\exx
The \verb'cur' method is {\em lazy},
meaning that the \verb'Scan' class does not read any characters
from the input stream until necessary
to satisfy an explicit \verb'cur' method call.
\end{minipage}
\clearpage
