\begin{minipage}[t]{\sw}
\slidenumber
\LARGE
{\bf PLCC -- Semantics}\exx
For a given ``program'' in the language defined by the grammar,
the \verb'parse' method of the start symbol class
(the first nonterminal in the language grammar rules)
returns an instance of this class.
This instance is the root of a parse tree for the program.
For example, given the grammar
{\Large
\begin{qv}
<tree>:Leaf     ::= <NUMBER>
<tree>:Interior ::= LPAREN <SYMBOL> <tree>left <tree>right RPAREN
\end{qv}
}
the \verb'parse' method returns an instance of a \verb'Tree' object.\exx
The default semantics of a PLCC program is
to return the \verb'toString' value of this instance.
This means that, in this example,
a \verb'toString' method must be defined
in the \verb'Tree' class.
Since this class appears more than once on the LHS of the grammar rules,
each derived class \verb'Leaf' and \verb'Interior' must define
its respective \verb'toString' method.\exx
For example, the \verb'Leaf' class has a field \verb'num'
of type \verb'Token'.
The \verb'toString' method in this class can be easily written as follows:
{\Large
\begin{qv}
public String toString() {
    return num.toString();
}
\end{qv}
}
\end{minipage}
\clearpage
