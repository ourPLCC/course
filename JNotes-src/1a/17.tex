\begin{minipage}[t]{\sw}
\slidenumber
\LARGE
{\bf PLCC -- Parsing (continued)}\exx
The code samples shown above
omit a \verb'Trace' parameter named \verb'trace$'
that plays a role in calling the \verb'parse' methods.
If this parameter is not \verb'null' --
when the `\verb'-t'' command-line option is given
to the \verb'Parse' or \verb'Rep' programs --
parsing the program
will display a visual ``parse trace'' to standard error (by default).
In Language \verb'LON', an example parse trace
for the program `\verb'(1 3 5)'' looks like this:
{\Large
\begin{qv}
   1: <lon>
   1: | LPAREN "("
   1: | <nums>:NumsNode
   1: | | NUM "1"
   1: | | <nums>:NumsNode
   1: | | | NUM "3"
   1: | | | <nums>:NumsNode
   1: | | | | NUM "5"
   1: | | | | <nums>:NumsNull
   1: | RPAREN ")"
\end{qv}
}
Each line in the parse trace displays
either a nonterminal or a token.
If it displays a nonterminal,
this shows that the parser calls
the \verb'parse' method for that nonterminal
(possibly in the given subclass).
If it displays a token,
then the scanner also displays the token's lexeme.
The decimal number at the beginning of each line
is the line number in the source file where the parse found
the nonterminal or token;
the number of vertical bars indicates
the recursive depth of the parse.
\end{minipage}
\clearpage
