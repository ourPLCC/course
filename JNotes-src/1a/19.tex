\begin{minipage}[t]{\sw}
\slidenumber
\LARGE
{\bf PLCC -- Semantics (continued)}\exx
To get a behavior different from this default behavior,
the \verb'$run()' method can be redefined
in the start symbol class or any of its subclasses.
For example, the \verb'$run()' method may be redefined
in the \verb'Tree' class so that
it displays a more human-readable \verb'toString' representation of the object
than the somewhat uninformative default representation.
Here is code for the \verb'$run()' method in the \verb'Tree' class
that redefines this behavior:
{\Large
\begin{qv}
    public void $run() {
       System.out.println("Tree: " + this.toString());
    }
\end{qv}
}
Since the \verb'Tree' class appears more than once on the LHS of the grammar rules,
each derived class \verb'Leaf' and \verb'Interior' can define
its respective \verb'toString' method.\exx
For example, the \verb'Leaf' class has a field \verb'num'
of type \verb'Token'.
The \verb'toString' method in this class can be easily written as follows:
{\Large
\begin{qv}
public String toString() {
    return num.toString();
}
\end{qv}
}
\end{minipage}
\clearpage
