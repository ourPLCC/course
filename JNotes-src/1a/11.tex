\begin{minipage}[t]{\sw}
\slidenumber
\LARGE
{\bf PLCC -- Syntax (continued)}\exx
A BNF grammar rule may have the same nonterminal name appearing
more than once on its RHS, as in
{\Large
\begin{qv}
<tree>:Interior ::= LPAREN <SYMBOL> <tree> <tree> RPAREN
\end{qv}
}
(see Slide 1.2).
PLCC requies that there are no duplicates of class field names
associated with the RHS of a given grammar rule,
so the above grammar rule would not be acceptable to PLCC.
To solve the problem of duplicate field names,
we annotate the duplicate field entries
by appending an alternate field name
(any Java identifier will do, but the convention is
to have it start with a lowercase letter)
which becomes the name of the corresponding field.
The following annotations fix the above example:
{\Large
\begin{qv}
<tree>:Interior ::= LPAREN <SYMBOL> <tree>left <tree>right RPAREN
\end{qv}
}
The same requirement that there be no duplicate field entries
applies to fields associated with tokens instead of nonterminals.
So for a BNF rule such as
{\Large
\begin{qv}
<hhmmss> ::= <TWOD> COLON <TWOD> COLON <TWOD> NL
\end{qv}
}
we would annotate the three \verb'<TWOD>' token fields with different names:
{\Large
\begin{qv}
<hhmmss> ::= <TWOD>hh COLON <TWOD>mm COLON <TWOD>ss NL
\end{qv}
}
\end{minipage}
\clearpage
