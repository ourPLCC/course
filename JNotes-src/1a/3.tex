\begin{minipage}[t]{\sw}
\slidenumber
\LARGE
{\bf PLCC -- Tokens}\\
\Large
A skip specification is used
to identify things in a program
that are otherwise not meaningful
to the behavioral structure of the program.
We generally skip whitespace (spaces, tabs, and newlines) and comments.
The format of a comment is language-dependent,
but in our languages, a comment starts with a `\verb'%'' character
and continues to the end of the line.
Here is an example of skip specifications for whitespace and comments:
\begin{qv}
skip WHITESPACE '\s+'
skip COMMENT '%.*'
\end{qv}
A token specification is used to identify things in a program
that are meaningful to the behavioral structure of the program.
Examples are a language reserved word (such as \verb'if'),
a punctuation mark (such as a left bracket `\verb'[''),
a numeric literal (such as \verb'346'),
or a variable symbol (such as \verb'xyz').
Here are specifications that match these tokens.
\begin{qv}
token IF 'if'
token LBRACK '\['
token LIT '\d+'
token VAR '[A-Za-z]\w*'
\end{qv}
For token specifications, the initial `\verb'token'' can be omitted.
Notice that regular expression meta-characters such as `\verb'[''
must be quoted with a backslash `\verb'\''
if they are to be treated as ordinary characters.

\end{minipage}
\clearpage
