\begin{minipage}[t]{\sw}
\slidenumber
\LARGE
{\bf Language TYPE0}\exx
We define a language named \verb'TYPE0'
that is based on the \verb'SET' language
and that implements type syntax.
We first require that a procedure definition --
in the \verb'<proc>' grammar rule --
declare the types of each of its formal parameters
and the return type of the procedure.
Each of these declarations are given by a type expression,
described earlier,
in the following grammar rules related to a \verb'<proc>'.
Notice that these grammar rules are similar
to the original grammar rules for the \verb'SET' language,
with the exception that they include type declarations.\exx
\Large
\emm\begin{tabular}{ll}
\verb'<proc>' & \verb'::= PROC LPAREN <formals> RPAREN COLON <typeExp> <exp>'\\
    & \VerbBox{\fbox}{\verb'Proc(Formals formals, TypeExp typeExp, Exp exp)'}\\
\verb'<formals>' & \verb'**= <VAR> COLON <typeExp> +COMMA' \\
    & \VerbBox{\fbox}{\verb'Formals(List<Token> varList, List<TypeExp> typeExpList)'}\\
\end{tabular}\exx
\LARGE
These changes do {\em not} affect
the behavior of the \verb'eval' methods
that define the semantics of the \verb'SET' language.
Thus the resulting \verb'TYPE0' language has the same semantics
as the \verb'SET' language,
except for the presence of \verb'bool' values:
the type information in a program is simply ignored.
\end{minipage}
