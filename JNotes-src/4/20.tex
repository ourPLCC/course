\begin{minipage}[t]{\sw}
\slidenumber
\LARGE
{\bf Language TYPE1} (continued)\exx
When processing declaring the formal parameter types of procedures
and processing \verb'let' and \verb'letrec' expressions,
we want to ensure that there are no duplicate variable names.
This amounts to checking the \verb'varList' instance variables
in the \verb'Formals', \verb'LetDecls', and \verb'LetrecDecls' classes.
This must be done after parsing,
since the parser only uses the grammar rules
and these rules do not have any way to check for duplicates.
We check for duplicates during type checking
and before expression evaluation.\exx
One way to check for duplicates is to use a Java \verb'Map' object:
for each variable in the list of variables,
if it's already in the map, return an error,
otherwise add it to the map.
Here is the code:
\Large
\begin{qv}
public void checkDuplicates(List<String> varList) {
    Map<String,String> map = new HashMap<String,String>();
    for (String id : varList) {
        String s = map.get(id);
        if (s != null)
            throw new RuntimeException("duplicate identifier: "+id);
        map.put(id, id);
    }
}
\end{qv}
\end{minipage}
