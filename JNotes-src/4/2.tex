\begin{minipage}[t]{\sw}
\slidenumber
\LARGE
{\bf Types}\exx
Observe that the \verb'int' type in our (as yet unnamed) language
is {\em not} the same as the \verb'int' type in Java,
even though they are closely related and have the same name.
There should be no confusion about \verb'bool', 
since in Java the corresponding type is \verb'boolean'.\exx
Our objective is to associate a type
with every instance of the class \verb'Val'.
An \verb'IntVal' has type \verb'int',
and a \verb'BoolVal' has type \verb'bool'.
The only other \verb'Val' object is a \verb'ProcVal',
so we need a way to give a type to a \verb'ProcVal'.\exx
When we define a procedure, we {\em declare} the types
of each of the procedure's formal parameters
and the procedure's return type.
We show how to do this shortly.
Once we know these declared types,
we can define the type of the procedure.\exx
We use the notation
\[ [\ t_1\ ,\ t_2\ ,\ \cdots\ ,\ t_n\ =>\ t\ ] \]
to represent the type of a procedure
that has $n$ formal parameters of types $t_1$, $t_2$, $\cdots$ $t_n$,
respectively, and that returns a value of type $t$.\exx
To simplify things, we do not permit the creation
of new (named) types,
so every type is either an integer, boolean, or procedure.
\end{minipage}
