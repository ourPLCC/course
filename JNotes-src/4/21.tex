\begin{minipage}[t]{\sw}
\slidenumber
\LARGE
{\bf Language TYPE1} (continued)\exx
While the type environment classes are similar
to those for expression environments,
it is unnecessary to use references,
since we never modify the type of a variable.
Therefore type environments are similar
to the environment implementation in language \verb'V6'.
The essential changes are to replace \verb'Var' with \verb'Type',
\verb'Env' with \verb'TypeEnv',
and \verb'Binding' with \verb'TypeBinding'.\exx
We are now prepared to write the code to type-check expressions.
We need to initialize a top-level type environment,
which we call \verb'initTypeEnv'.
This happens in the \verb'Program' class,
where we also define the top-level expression environment.
\Large
\begin{qv}
    static TypeEnv tenv = TypeEnv.initTypeEnv();
\end{qv}
\LARGE
In the \verb'Eval' class's \verb'toString' method,
we type-check by applying the \verb'evalType' on the \verb'exp' object
-- the ``program'' that we are to evaluate.
The actual return value of this is irrelevant,
since its purpose is simply to carry out type checking
prior to expression evaluation.
If the type checking succeeds,
the \verb'toString' method returns the string representation
of the value of the expression.
\Large
\begin{qv}
    public String toString() {
        exp.evalType(Program.tenv); // type check, then ...
        return exp.eval(Program.env).toString(); // ... evaluate
    }
}
\end{qv}
\end{minipage}
