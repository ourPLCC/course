\begin{minipage}[t]{\sw}
\slidenumber
\LARGE
{\bf Parsers} (continued)\exx
\Large
\emm\LightBox{\MYlonGrammarKleene}\exx
\LARGE
Here is an (edited) example
of a parse trace for the list of numbers \verb'(3 5 8 13)':
\begin{qv}
--> (3 5 8 13)
<lon>
| LPAREN "("
| <nums>
| | NUM "3"
| | NUM "5"
| | NUM "8"
| | NUM "13"
| RPAREN ")"
\end{qv}
If you compare this with the parse trace using the previous grammar
that does not use the \verb'**=' construct,
you see that the earlier parse trace drifts to the right
as additional \verb'NUM' entries are encountered.
Using the \verb'**=' construct,
the parse trace becomes flat.\exx
PLCC grammar rules that use this construct
are called {\em repeating grammar rules}.
Repeating rules are useful in specifying most of our languages.
\end{minipage}
\clearpage
