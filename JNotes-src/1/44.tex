\begin{minipage}[t]{\sw}
\slidenumber
\LARGE
{\bf Static Properties of Variables}\exx
A {\em variable} in a program is a symbol
that has an associated value at run-time.
One of the principal issues in determining the behavior of a program
is determining {\em how} to find the value of a variable at run-time.
At any instance in time, the value associated with a variable
is called a {\em binding} of the variable to the value.\exx
An {\em expression} is a syntactic construct that has a value at run-time.
A variable, by itself, is therefore an expression,
but other syntactic constructs can also have values:
for example \verb'x+y' is an expression
if \verb'x' and \verb'y' are numeric-valued variables.\exx
A programming language that is constructed solely for the purpose
of evaluating expressions is called an {\em expression-based language}.
Most of the languages we construct in these notes are expression-based.
\verb'Scheme', \verb'ML', and \verb'Haskell' are examples
of expression-based languages used in practice.
A programming language that is constructed for the purpose of
``doing something'' with expressions
(such as assigning the value of an expression to a variable
or printing the value of an expression to standard output)
is called an {\em imperative language}.
\verb'C', \verb'Java', and \verb'Python' are examples
of imperative languages used in practice.\exx
Expression-based languages get their power
from defining and applying {\em functions},
so another term describing such languages is {\em functional}.

\end{minipage}
\clearpage
