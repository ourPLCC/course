\begin{minipage}[t]{\sw}
\slidenumber
\LARGE
{\bf Parsers} (continued)\exx
\Large
\emm\LightBox{\MYlonGrammarOnly}\exx
PLCC generates a stand-alone parser with class name \verb'Parser'
that uses the Java classes
created from processing the lexical and grammar specification sections.
To run this parser,
change to the Java subdirectory (of the LON directory, in this case)
and run the \verb'Parser' class file with command-line arguments
consisting of strings that you want to parse.
For example:
\begin{qv}
java Parser "(14 6)" "(14 (6)" "(42)" "()"
\end{qv}
The output produced looks something like this:
\begin{qv}
(14 6) -> Lon@15db9742
(14 (6) -> java.lang.RuntimeException: Nums cannot begin with LPAREN
(42) -> Lon@7852e922
() -> Lon@6d06d69c
\end{qv}
For each command-line argument,
the parser prints a copy of the argument,
then a string `\verb'->'',
and finally a string of the form \verb'Lon@...'.\exx
The \verb'Lon@...' strings are simply the \verb'toString()' values
of the instances of the \verb'Lon' class
that are created for each {\bf successful} parse.
For any grammar, an instance of the class associated
with the {\em start symbol} of the language
is always the root of the parse tree generated by the parser.
If the parse fails for a particular command-line string,
the parser prints an error message giving the nature of the error.
\end{minipage}
\clearpage
