\begin{minipage}[t]{\sw}
\slidenumber
\LARGE
{\bf Parsers} (continued)\exx
{\em
For every BNF grammar rule,
PLCC creates a Java class uniquely associated with the rule.
For BNF rules that have the same nonterminal appearing
on the LHS of multiple rules,
PLCC creates an abstract class based on the nonterminal name,
and the annotated class names
become derived classes of this abstract class.}\exx
The first BNF nonterminal
in a language's syntax specification section
is the {\em start symbol} for the language.
Given a program in the language,
the parser produces an instance of the class of the start symbol,
which is the root of the parse tree for the program.\exx
Consider Language \verb'LON',
whose \verb'grammar' file syntax specification section is shown below.
The start symbol is \verb'<lon>',
and the corresponding class name is \verb'Lon'.
Given a program in Language \verb'LON',
the output of the parser is an instance of the \verb'Lon' class.
We will see shortly how to interpret this instance
as the root of a {\em parse tree}.\exx
\emm\LightBox{\MYlonGrammarOnly}
\end{minipage}
\clearpage
