\begin{minipage}[t]{\sw}
\slidenumber
\LARGE
{\bf Parsers} (continued)\exx
To remedy this situation,
we want to define a field
in the \verb'NumsNode' class that captures
the \verb'NUM' token obtained by the parser,
which in turn allows us to retrieve the token's lexeme.
We already use angle brackets
for all {\em nonterminals} on the RHS of grammar rules,
and these nonterminals automatically become fields in the Java class.
So to capture {\em tokens} on the RHS,
we use angle brackets for these tokens as well.
This means that the \verb'NumsNode' line now looks like this:
\begin{qv}
<nums>:NumsNode  ::= <NUM> <nums>
\end{qv}
The `\verb'<NUM>'' entry creates a field named \verb'num'
(which is the token name `\verb'NUM'' converted to lowercase)
of type \verb'Token' (since \verb'NUM' is the name of a token 
in the \verb'grammar' file).\exx
Observe that it's unnecessary
to capture tokens that always have the same lexeme,
like \verb'LPAREN':
every instance of the \verb'LPAREN' token
looks like every other instance,
so there's no need to distinguish among them.
This is not so with tokens that can take
on multiple lexeme values, such as \verb'NUM'.
Indeed, for a list of numbers,
knowing exactly {\em what} numbers are in the list
can be essential --
for example, if you want to find the sum of the numbers in the list.
If these items do not appear as fields in the PLCC-generated classes,
their values do not appear in the parse tree,
and so their values are not retrievable after the parse.
\end{minipage}
\clearpage
