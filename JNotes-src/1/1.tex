\begin{minipage}[t]{\sw}
\slidenumber
\LARGE
{\bf Specifying Syntax}\exx
A {\em language},
in computer science theory,
is a set of strings,
where a {\em string} is a finite ordered sequence of {\em symbols}
chosen from a given {\em alphabet}.
Computer science theory
deals principally
with how to specify languages
using things such as Nondeterministic Finite Automata (NFAs),
Context-Free Grammars,
and Turing Machines (TMs).\exx
A {\em programming language}
also defines a language in the theory sense,
except that the strings in a programming language are called {\em programs},
and the symbols are called {\em tokens}.
The {\em syntax} of a programming language is a set
of rules used to specify the programs in the language.
A programming language also defines the run-time behavior
of a program, called its {\em semantics},
which we discuss at length later.\exx
Our first step is to describe a way in which
we can define the syntax of a programming language.
We start out with two simple examples
of languages that describe familiar data structures.
\end{minipage}
\clearpage
