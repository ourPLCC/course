\begin{minipage}[t]{\sw}
\slidenumber
\LARGE
{\bf Parsers} (continued)\exx
\Large
\emm\LightBox{\MYlonGrammarOnly}\exx
The \verb'Rep' program prints only
the Java object that represents the root of the program's parse tree.
To see how the objects in the parse tree are being constructed
as the parse proceeds,
you can use the `\verb'-t'' switch when invoking \verb'Rep'.
The `\verb'-t'' stands for \verb'trace'.
Here's a sample interaction using this feature,
with the output edited for the sake of readability:
\begin{qv}
$ java Rep -t
--> (14 6)
<lon>
| LPAREN "("
| <nums>:NumsNode
| | NUM "42"
| | <nums>:NumsNode
| | | NUM "6"
| | | <nums>:NumsNull
| RPAREN ")"
\end{qv}
In this example,
the root of the parse tree is a \verb'Lon' object
whose \verb'nums' field is an instance of \verb'NumsNode'.
This instance in turn has a \verb'nums' field that is also
an instance of \verb'NumsNode'.
And finally, this instance has a \verb'nums' field
that is an instance of \verb'NumsNull'.
A \verb'NumsNull' object has no fields.
The trace also shows how each token is matched by the parser,
along with the lexeme
from the input program that matched the token.
\end{minipage}
\clearpage
