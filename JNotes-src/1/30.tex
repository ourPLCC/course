\begin{minipage}[t]{\sw}
\slidenumber
\LARGE
{\bf Parsers} (continued)\exx
To remedy this situation, we need to have a field
in the \verb'NumsNode' class that captures the value
of the \verb'NUM' token.
We use angle brackets for all nonterminals on the RHS of grammar rules,
and these nonterminals automatically become fields in the Java class.
To capture values of the tokens on the RHS,
we use angle brackets for these token names as well.
This means that the \verb'NumsNode' line now looks like this:
\begin{qv}
<nums>:NumsNode  ::= <NUM> <nums>
\end{qv}
The `\verb'<NUM>'' entry creates a field named \verb'num'
(which is the name `\verb'NUM'' converted to lowercase)
of type \verb'Token' (since \verb'NUM' is the name of a token 
in the \verb'grammar' file).\exx
Observe that it's unnecessary
to capture tokens that always have the same string representation,
like \verb'LPAREN'.
Every instance of the \verb'LPAREN' token
looks like every other instance,
so there's no need to distinguish among them.
This is not so with tokens that can take on multiple string values,
such as \verb'NUM'.
Indeed, for a list of numbers,
knowing exactly {\em what} numbers are in the list
can be essential --
for example, if you want to find the sum of the numbers in the list.
If these items do not appear as fields in the PLCC-generated classes,
their values do not appear in the parse tree,
and so their values are not retrievable after the parse.
\end{minipage}
\clearpage
