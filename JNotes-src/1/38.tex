\begin{minipage}[t]{\sw}
\slidenumber
\LARGE
{\bf Parsers} (continued)\exx
\emm\LightBox{\MYlonGrammarKleene}\exx
How exactly can we access the values of the \verb'NUM' fields
from the parse tree?
The \verb'Rep' program first parses the program,
yielding an instance of the start symbol class
-- an instance of \verb'Lon', in this case.
It then prints the \verb'toString()' value of this instance,
which as we have seen defaults to something like `\verb'Lon@...''.\exx
But a \verb'Lon' object has a \verb'nums' field of type \verb'Nums',
and a \verb'Nums' object has a \verb'numList' field of type \verb'List<Token>',
so perhaps we can redefine the \verb'toString()' method
in the \verb'Lon' class to print the values in this list.\exx
To do this, we need to modify the \verb'Lon.java' file
to incorporate this new \verb'toString()' behavior.
We can do this directly by editing the file
(once it has been created by PLCC).
But every change we make to the language specification involves
re-running the \verb'plccmk' script,
which will clobber any edits we may have made to the previous version.\exx
Fortunately, PLCC allows us to add methods to PLCC-generated source files
by including the added methods in the language specification file.
Every time \verb'plccmk' is run,
these added methods are incorporated
into the Java source files automatically.\exx
\end{minipage}
\clearpage
