\begin{minipage}[t]{\sw}
\slidenumber
\LARGE
{\bf Parsers} (continued)\exx
The solution is to use different identifiers for these field names
and to their corresponding constructor formal parameters.
PLCC allows duplicate RHS fields (in angle brackets)
to be annotated -- much like we have seen
for duplicate LHS nonterminal names
-- with alternate names that avoid this conflict.\exx
In the case we are considering, we can resolve this issue in the RHS
of the \verb'<tree>:Interior' grammar rule
by using the identifier \verb'left' for the first \verb'<tree>' field
and the identifier \verb'right' for the second \verb'<tree>' field,
as shown here:\exx
\Large
\emm\begin{tabular}{ll}
\verb'<tree>:Interior' & \verb'::= LPAREN <SYMBOL> <tree>left <tree>right RPAREN'\\
    & \VerbBox{\fbox}{\verb'Interior(Token symbol, Tree left, Tree right)'}\\
\end{tabular}\exx
\LARGE
In summary, this grammar rule creates:
\begin{itemize}
\itemsep -0.1in
\item a class \verb'Interior'
\item that extends the abstract class \verb'Tree' and
\item that has a field \verb'symbol' of type \verb'Token',
\item a field \verb'left' of type \verb'Tree', and
\item a field \verb'right' of type \verb'Tree'.
\end{itemize}
\end{minipage}
\clearpage
