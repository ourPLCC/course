\begin{minipage}[t]{\sw}
\slidenumber
\LARGE
{\bf Parsers} (continued)\exx
This sort of looping occurs frequently
in programming language specifications,
and PLCC has a way to encode this.
Instead of having two \verb'<nums>' rules,
with the first being right-recursive and the second having an empty RHS,
we can re-write these rules using a special `\verb'**='' notation:\exx
\emm\begin{tabular}{ll}
\verb'<nums>' & \verb' **= <NUM>' \\
              & \VerbBox{\fbox}{\verb'Nums(List<Token> numList)'} \\
\end{tabular}\exx
The parser accumulates all of the \verb'NUM' tokens
into a single \verb'numList' field.
The \verb'numList' field name is obtained from the \verb'NUM' token name
by converting all of its characters to lowercase
and appending the string \verb'List'.\exx
{\bf Note:} The use of `\verb'**'' in the notation we have just introduced
should suggest the {\em Kleene star} repetition notation
used in EBNF as well as in regular expressions.\exx
The modified list-of-numbers grammar is as follows:\exx
\emm\LightBox{\MYlonGrammarKleene}\exx
\end{minipage}
\clearpage
