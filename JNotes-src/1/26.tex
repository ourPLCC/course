\begin{minipage}[t]{\sw}
\slidenumber
\LARGE
{\bf Parsers} (continued)\exx
\Large
\emm\LightBox{\MYlonGrammarOnly}\exx
PLCC also generates an interactive parser
called \verb'Rep' that resides in the \verb'Java' subdirectory
along with the \verb'Parser' program.
\verb'Rep' executes a loop
that prints a prompt,
{\em Reads} program input from the keyboard,
{\em Evaluates} (parses) the program,
and {\em Prints} the result:
a human-readable representation
of the root of the program's parse tree.
Recall that for the language given above,
the root of the parse tree is
an instance of the \verb'Lon' class.\exx
Here is a sample interaction with the \verb'Rep' program
using this grammar:
\begin{qv}
$ java Rep
--> (14 6)
Lon@15db9742
--> (14 (6)
java.lang.RuntimeException: Nums cannot begin with LPAREN
--> (42)
Lon@6d06d69c
--> ()
Lon@7852e922
--> 
\end{qv}
Observe that the output is similar to that produced
by the \verb'Parser' program
\end{minipage}
\clearpage
