\begin{minipage}[t]{\sw}
\slidenumber
\begin{SaveVerbatim}{\MYlon}
<lon>  ::= LPAREN <nums> RPAREN
<nums> ::= NUM <nums>
<nums> ::=
\end{SaveVerbatim}
\begin{SaveVerbatim}{\MYonemodthree}
<onemodthree>  ::= ONE
<onemodthree>  ::= THREE PLUS <onemodthree>
\end{SaveVerbatim}
\begin{SaveVerbatim}{\MYtree}
<tree>  ::= NUM
<tree>  ::= LPAREN SYMBOL <tree> <tree> RPAREN
\end{SaveVerbatim}
\begin{SaveVerbatim}{\MYtreeTOK}
<tree>:Leaf     ::= <NUM>
<tree>:Interior ::= LPAREN <SYMBOL> <tree> <tree> RPAREN
\end{SaveVerbatim}
\LARGE
{\bf Backus-Naur Form (BNF)}\exx
BNF is a meta-language used
to specify a context-free grammar.
Almost every modern programming language uses some sort
of BNF notation to define its syntax.
We work first with examples of languages
that define two simple data structures:
lists and trees.
Remember that the ``alphabet'' of a programming language
is a set of tokens,
so any definition of the syntax of the language
must first specify its tokens.
In the previous section, we showed how to do this
in the context of PLCC.\exx
Our first example is to define a language whose ``programs''
are lists of numbers.
Some sample ``programs'' in this language are:
\begin{qv}
(3 4 5)
(    7 11  )
()
\end{qv}
\end{minipage}
\clearpage
