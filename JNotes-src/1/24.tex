\begin{minipage}[t]{\sw}
\slidenumber
\Large
\begin{SaveVerbatim}{\MYlonGrammarOnly}
<lon>           ::= LPAREN <nums> RPAREN
<nums>:NumsNode ::= NUM <nums>
<nums>:NumsNull ::= 
\end{SaveVerbatim}
\begin{SaveVerbatim}{\MYlonGrammarTOK}
<lon>           ::= LPAREN <nums> RPAREN
<nums>:NumsNode ::= <NUM> <nums>
<nums>:NumsNull ::= 
\end{SaveVerbatim}
\begin{SaveVerbatim}{\MYlonGrammarKleene}
<lon>  ::= LPAREN <nums> RPAREN
<nums> **= <NUM>
\end{SaveVerbatim}
\LARGE
{\bf Parsers} (continued)\exx
\emm\LightBox{\MYlonGrammarOnly}\exx
(The above text shows just the syntax section
of the \verb'grammar' file we have been examining.)
In your \verb'LON' directory,
change to the subdirectory named \verb'Java'.
The \verb'plccmk' command has created this subdirectory
and populated it with Java source code generated by \verb'plcc.py'.
In this subdirectory you will find (among other things)
the following Java source files:
\begin{qv}
Lon.java
Nums.java
NumsNode.java
NumsNull.java
\end{qv}
Each of these corresponds to one or more of the grammar rule lines.
For example, the line beginning with \verb'<lon>'
results in the file \verb'Lon.java' being created
in the \verb'Java' subdirectory.
As you can see from looking at the Java code in the \verb'Java' subdirectory
for the \verb'NumsNode' and \verb'NumsNull' classes,
both of these classes extend the \verb'Nums' abstract class.
This is because the \verb'<nums>' nonterminal
appears as the LHS of two grammar rules.
\end{minipage}
\clearpage
