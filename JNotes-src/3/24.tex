\begin{minipage}[t]{\sw}
\slidenumber
\LARGE
{\bf Language V1}\exx
Now that you see how a parse tree
for language \verb'V0' can {\em print} itself,
let's show how a parse tree can {\em evaluate itself}.\exx
The term {\em evaluate} can have many meanings
(one of which is to produce a \verb'String' representation of itself),
but for our purposes,
to evaluate an arithmetic expression such
as \VerbBox{\fbox}{\verb'add1( + (2,3))'}
is to produce the integer value 6.
In other words, the value of an arithmetic expression
is its numeric value using usual rules for arithmetic.\exx
(Remember that we are abstracting the notion of {\em value}
to refer to an instance of the \verb'Val' class.
In this setting, a numeric value is an instance
of the \verb'IntVal' subclass.)\exx
If an expression involves an identifier (symbol),
we need to determine the value bound to that identifier
in order to evaluate the expression.
For example, suppose the identifier \verb'"x"' is bound
to the integer value 10:
then the expression \VerbBox{\fbox}{\verb'sub1(x)'}
would evaluate to 9.\exx
{\em Every expression is evaluated in an environment.
This environment determines how to obtain
the values of the the variables that occur in the expression}\exx
\end{minipage}
