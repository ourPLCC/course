\begin{minipage}[t]{\sw}
\slidenumber
\LARGE
{\bf Language V1} (continued)\exx
Four of the \verb'Prim' objects need two arguments
(`\verb'+'', `\verb'-'', `\verb'*'', and `\verb'/''),
and three of them need one argument
(\verb'add1', \verb'sub1', and \verb'zerop').
Since \verb'va' is an array of \verb'Val' arguments,
we can grab the appropriate items from this array --
one or two of them, depending on the operation --
to evaluate the result.
Here is the code for the \verb'AddPrim' class:
{\Large
\begin{qv}
AddPrim
%%%
    public Val apply(Val [] va) {
        if (va.length != 2)
            throw new PLCCException("two arguments expected");
        int i0 = va[0].intVal().val;
        int i1 = va[1].intVal().val;
        return new IntVal(i0 + i1);
    }
%%%
\end{qv}
}
The \verb'intVal()' method calls shown in this
code convert \verb'Val' objects (such as \verb'Va[0]')
into \verb'IntVal' objects -- essentially like ``downcasting''.
These objects, in turn, have Java \verb'int' fields named \verb'val'.
So both \verb'i0' and \verb'i1' are legitimate Java \verb'int's
that can be added together to return the resulting \verb'IntVal' object.
The \verb'Val' class defines the \verb'intVal()' method behavior:
an attempt to apply the \verb'intVal()' method
to a \verb'Val' object that is {\em not} an \verb'IntVal'
throws an exception.
\end{minipage}
\clearpage
