\begin{minipage}[t]{\sw}
\slidenumber
\LARGE
{\bf Language V4} (continued)\exx
Though language V4 does not support direct recursion,
its support of procedures as first-class entities --
that is, they are values that are treated
the same way as other values,
so they can be passed as parameters and returned as values --
is as powerful as recursion.
Here is another example that recursively computes factorials
using an ``accumulator'' and tail recursion
(we will return to this topic later):
\Large
\begin{qv}
let
  fact = proc(x)
    let
      fact = proc(fact, x, acc)
        if zero?(x)
        then acc
        else .fact(fact, sub1(x), *(x, acc))
    in
      .fact(fact, x, 1)
  in
    .fact(5)
\end{qv}
\LARGE
Observe that the identifier \verb'fact' that appears
in the \verb'proc(fact,...)' definition
is a {\em formal parameter} name
that binds all occurrences of \verb'fact'
that appear in the procedure body.
You may find it instructive to display all of the environments
that get created during the evaluation of this expression.
(Replace `5' by `2' to make things easier.)
\end{minipage}
