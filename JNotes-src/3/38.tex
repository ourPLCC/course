\begin{minipage}[t]{\sw}
\slidenumber
\LARGE
{\bf Language V2} (continued)\exx
We define an \verb'isTrue()' method for an \verb'IntVal' object as follows.
This code is part of the \verb'IntVal' class that is defined
in the \verb'val' file --
only the definition of \verb'isTrue' is given here:
{\Large
\begin{qv}
public boolean isTrue() {
    return val != 0; // nonzero is true, zero is false
}
\end{qv}
}
Observe that the \verb'eval()' method in the \verb'IfExp' class applies
the \verb'isTrue()' method to a \verb'Val' object,
so we must include a declaration for the \verb'isTrue()' method
in the \verb'Val' base class.
Since we treat {\em any} \verb'Val' object as true
if it's not an \verb'IntVal' of zero,
our default \verb'isTrue()' method in the \verb'Val' class defaults
to returning \verb'true'.
{\Large
\begin{qv}
Val
%%%
...
    public boolean isTrue() {
        return true;
    }
...
%%%
\end{qv}
}
\end{minipage}
