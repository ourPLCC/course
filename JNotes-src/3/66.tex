\begin{minipage}[t]{\sw}
\slidenumber
\LARGE
{\bf Drawing Envinronments}\exx
On Slide 2.20, we showed how to display environments as a linked lists.
Each node in the list is a pair (\verb'EnvNode')
consisting of a reference to a \verb'Bindings' object
(which we have called {\em local bindings})
and a reference to the next node in the list.
The end of the list is an empty environment,
an \verb'EnvNull' object,
which appears as a box with a slash through it.
We usually display the linked list nodes {\em from left to right},
with the head of the list at the left
and the empty environment at the right.
We display the local bindings as an array
(it's actually an \verb'ArrayList') of bindings stacked vertically.
Each binding is a pair consisting of an identifier string
and a value.\exx
The {\em initial environment} in Language V4
is a linked list consisting of an \verb'EnvNode'
with an empty local environment (no bindings)
and a reference to an \verb'EnvNull' object.
Here is how we display the initial environment:
\centerline{\psfig{figure=init.eps,width=2in}}
{\bf To simplify things in Languages V4 and V5,
we omit displaying the node with the empty local environment,}
so we display the initial environment as follows:\exx
\centerline{\psfig{figure=init0.eps,width=0.4in}}
\end{minipage}
\clearpage
