\begin{minipage}[t]{\sw}
\slidenumber
\LARGE
{\bf Language V3} (continued)\exx
\Large
\emm\VerbBox{\colorbox{light}}{%
\begin{tabular}{@{}ll}
\verb'<exp>:LetExp' & \verb'::= LET <letDecls> IN <exp>'\\
  & \VerbBox{\fbox}{\verb'LetExp(LetDecls letDecls, Exp exp)'}\\
\verb'<letDecls>' & \verb'**= <VAR> EQUALS <exp>'\\
  & \VerbBox{\fbox}{\verb'LetDecls(List<Token> varList, List<Exp> expList)'}\\
\end{tabular}%
}\exx
\LARGE
Observe that a \verb'LetDecls' object
has two instance variables:
\verb'varList' is a list of tokens
representing the \verb'<VAR>' part of the grammar rule,
and \verb'expList' is a list of expressions
representing the \verb'<exp>' part of the grammar rule.
(The reason that these are \verb'List's is because
the grammar rule has a `\verb'**='' instead of a `\verb'::=''.)\exx
The plan for defining the \verb'addBindings' method
in the \verb'LetDecls' class
is to evaluate each of the expressions in \verb'expList'
in the enclosing environment
and bind them to their corresponding token strings in \verb'varList'.
These bindings are then used to extend the enclosing environment
given by the \verb'env' parameter,
and this new environment is returned to the \verb'eval' method
in the \verb'LetExp' class.\exx
\end{minipage}
