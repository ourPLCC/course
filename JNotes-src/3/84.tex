\begin{minipage}[t]{\sw}
\slidenumber
\LARGE
{\bf Language V5} (continued)\exx
Here are the grammar rules and associated abstract syntax classes:\exx
\Large
\begin{tabular}{@{}ll}
\verb'<exp>:LetrecExp' & \verb'::= LETREC <letrecDecls> IN <exp>'\\
  & \VerbBox{\fbox}{\verb'LetrecExp(LetrecDecls letrecDecls, Exp exp)'}\\
\verb'<letrecDecls>' & \verb'**= <VAR> EQUALS <proc>'\\
  & \VerbBox{\fbox}{\verb'LetrecDecls(List<Token> varList, List<Proc> procList)'}\\
\end{tabular}\exx
\LARGE
The environment in which each \verb'proc' of a \verb'letrec' is evaluated
should include bindings of each of the variables in the \verb'letrec'
to their corresponding closures.
This is unlike a normal \verb'let',
in which the values to which the variables are bound
are evaluated in the {\em enclosing} environment.\exx
The big question is,
how can a closure (the value of a \verb'Proc' object)
capture an environment that hasn't been created yet?\exx
The steps are shown on the next slide:
\end{minipage}
