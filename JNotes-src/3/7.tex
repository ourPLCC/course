\begin{minipage}[t]{\sw}
\slidenumber
\LARGE
{\bf Language V0} (continued)\exx
The term {\em abstract syntax} might seem odd
because it refers to a collection of very concrete Java classes.
Instead, the term {\em abstract} here means that
these classes keep only the information on the right-hand-side (RHS)
of the grammar rules that can change,
principally by ignoring RHS tokens.
For example, the \verb'<exp>:PrimappExp' grammar rule
has tokens \verb'LPAREN' and \verb'RPAREN' on its RHS,
but the generated \verb'PrimappExp' class
does not have fields corresponding to these tokens:
they are ``abstracted away''.\exx
Because the \verb'<exp>' and \verb'<prim>' nonterminals
appear on the LHS of two or more grammar rules,
their corresponding grammar rules are disambiguated
by annotating their LHS nonterminals with appropriate class names.
For these grammar rules, the LHS nonterminal corresponds to
the name of an abstract (base) Java class
whose name is obtained by capitalizing the first letter
of the nonterminal name.
The annotated classes \verb'LitExp', \verb'VarExp', and \verb'PrimappExp'
extend the abstract base class \verb'Exp'.
Similarly, the \verb'AddPrim', \verb'SubPrim',
\verb'Add1Prim' and \verb'Sub1Prim'
classes extend the abstract base class \verb'Prim'.\exx
Once you have run \verb'plccmk' on the specification file,
you can examine the Java code in the \verb'Java' subdirectory
to see, for example,
that the \verb'LitExp' class extends the \verb'Exp' class
and that the \verb'AddPrim' class extends the \verb'Prim' class.
\end{minipage}
