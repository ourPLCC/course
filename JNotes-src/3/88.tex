\begin{minipage}[t]{\sw}
\slidenumber
\LARGE
{\bf Language V5} (continued)\exx
\large
This picture illustrates the three steps
for creating the environment \verb'nenv'
used to evaluate the body of the following \verb'letrec' example:
\begin{qv}
letrec
    f = proc(x) ...
    g = proc(y) ...
in
    ...
\end{qv}
\begin{enumerate}
\item Create a new environment \verb'nenv'
      by extending the old environment with \verb'null' bindings;
\item Create a \verb'Bindings' object that binds
      the LHS identifiers (\verb'f' and \verb'g' in this example)
      to closures of the RHS procedures,
      where these closures capture the extended environment \verb'nenv'
      created in Step 1;
\item Replace the \verb'null' bindings of \verb'nenv'
      with the bindings created in Step 2.
\end{enumerate}
{\center\ \psfig{figure=letrec-steps.eps,width=.7\textwidth}\ \\}
\end{minipage}
