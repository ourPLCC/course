\begin{minipage}[t]{\sw}
\slidenumber
\LARGE
{\bf Language V5} (continued)\exx
{\Large
\emm\VerbBox{\colorbox{light}}{%
\begin{tabular}{@{}ll}
\verb'<exp>:LetrecExp' & \verb'::= LETREC <letDecls> IN <exp>'\\
  & \VerbBox{\fbox}{\verb'LetrecExp(LetDecls letDecls, Exp exp)'}\\
\end{tabular}%
}}\exx
Recall that the \verb'LetDecls' constructor checks
for duplicate LHS identifiers during parsing.
Since the \verb'LetrecExp' grammar rule uses \verb'LetDecls',
a \verb'letrec' expression also makes this check.\exx
We can now evaluate a \verb'LetrecExp' object
in exactly the same way as a \verb'LetExp' object:
{\Large
\begin{qv}
LetrecExp
%%%
    public Val eval(Env env) {
        Env nenv = letDecls.addLetrecBindings(env);
        return exp.eval(nenv);
    }
%%%
\end{qv}
}
The principal idea, then,
is to evaluate the RHS expressions of a \verb'letrec'
in an environment that (self-referentially) includes all of the bindings
in the \verb'letrec'.
\end{minipage}
\clearpage
