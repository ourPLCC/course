\begin{minipage}[t]{\sw}
\slidenumber
\LARGE
{\bf Language V2}\exx
Language \verb'V2' is the same as language \verb'V1'
with the addition of the syntax and semantics of an \verb'if' expression.
Here is the relevant grammar rule and abstract syntax representation:\exx
{\Large
\emm\begin{tabular}{@{}ll}
\verb'<exp>:IfExp' & \verb'::= IF <exp>testExp THEN <exp>trueExp ELSE <exp>falseExp' \\
  & \VerbBox{\fbox}{\verb'IfExp(Exp testExp, Exp trueExp, Exp falseExp)'}\\
\end{tabular}
}

Notice that we need to change our lexical specification
to allow for token names \verb'IF', \verb'THEN', and \verb'ELSE'.\exx
The RHS of this grammar rule has three occurrences
of the \verb'<exp>' nonterminal.
Since the \verb'<...>' items on the RHS of a grammar rule define
the instance variables of the class,
we have named these instance variables
\verb'testExp', \verb'trueExp', and \verb'falseExp', respectively.
Each of these objects refers to an instance of the \verb'Exp' class.
Thus the \verb'IfExp' class has three instance variables:

{\Large
\begin{qv}
Exp testExp;
Exp trueExp;
Exp falseExp;
\end{qv}
}
[{\bf Exercise} (not to hand in):
See what would happen if you used `\verb'<IF>''
in the RHS of this grammar rule instead of `\verb'IF''.]
\end{minipage}
