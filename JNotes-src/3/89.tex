\begin{minipage}[t]{\sw}
\slidenumber
\LARGE
{\bf Language V5} (continued)\exx
The \verb'letrec' construct allows us
to define {\em mutually recursive procedures} --
two or more procedures that call each other in a recursive fashion.
Here's a classic example:
\begin{qv}
letrec
  even? = proc(x) if zero?(x) then 1 else .odd?(sub1(x))
  odd? = proc(x) if zero?(x) then 0 else .even?(sub1(x))
in
  .even?(11) % => 0 (false)
\end{qv}
[{\bf Exercise} (not to hand in): See if you can do this in language V4
without \verb'letrec'.]\exx
Notice that we have used \verb'?'
in the variable names for the \verb'even?' and \verb'odd?' procedures.
This is an additional feature we have added
to languages \verb'V5' and beyond.

\end{minipage}
