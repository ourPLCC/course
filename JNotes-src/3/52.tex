\begin{minipage}[t]{\sw}
\slidenumber
\LARGE
{\bf Language V4} (continued)\exx
In language \verb'V4', procedures are treated as values
just like integers.
In particular, we create a \verb'ProcVal' class
that extends the \verb'Val' class.
This means that a \verb'ProcVal' object can occur
anywhere a \verb'Val' object is expected.\exx
Here is an example of a \verb'V4' program
that includes a procedure definition and application.
\Large
\begin{qv}
let
  f = proc(x) +(x,3)
in
  .f(5)
\end{qv}
\LARGE
A procedure definition starts with the \verb'PROC' token,
and a procedure application starts with a \verb'DOT'.
It is possible that you can define and apply a procedure in one expression,
such as
\Large
\begin{qv}
.proc(x) +(x,3) (5)
\end{qv}
\LARGE
Both of these expressions return the same value,
namely the integer \verb'8'.
Notice, too, that
\Large
\begin{qv}
proc(x) +(x,3)
\end{qv}
\LARGE
also returns a value, but the value is a procedure, not an integer.
(One's intent when defining a procedure is eventually to apply it,
although this is not a requirement.)
\end{minipage}

