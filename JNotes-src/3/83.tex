\begin{minipage}[t]{\sw}
\slidenumber
\LARGE
{\bf Language V5}\exx
We normally prefer to use direct recursion instead
of using the contrived (but workable) tricks on Slides 3.56 and 3.80.
For example, we would like to write
\Large
\begin{qv}
let
  fact = proc(x) if zero?(x) then 1 else *(x,.fact(sub1(x)))
in
  .fact(5)
\end{qv}
\LARGE
But this does not work! Why??\exx
Remember that in a \verb'let',
the RHS expressions
(the expressions to the right of the `\verb'='' tokens)
are all evaluated
in the environment that encloses the \verb'let';
only after all the RHS expressions have been evaluated
do we bind each of the LHS symbols to their RHS values.\exx
In the definition of the \verb'proc' above,
the \verb'proc' body refers to the identifier \verb'fact'
which is free in the procedure definition --
there is no \verb'fact' in its list of formal parameters --
but this identifier is not bound to a value
in the enclosing environment.
Thus an attempt to apply the \verb'proc'
fails because of an unbound identifier.
\end{minipage}
\clearpage
