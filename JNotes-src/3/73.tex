\begin{minipage}[t]{\sw}
\slidenumber
\LARGE
{\bf Drawing Envinronments} (continued)\exx
The rules for {\em applying} a procedure are given on slide 3.59.
With respect to environments,
the key ideas are:
(2a) the procedure's formal parameters are bound
to the values of the actual parameter expressions
(a \verb'Bindings' object), and then
(2b)
the captured environment is extended
with these bindings to create a new environment; finally,
(3) the value of the procedure application
is the value of the procedure body
using the new environment.\exx
Consider the following expression,
which is the same as the previous expression except
that the procedure is applied to the actual parameter 5:\\[1.5ex]
\Large
\begin{verbbox}
let
  x = 3
in
  .proc(t) +(t,x) (5)
\end{verbbox}
\emm\theverbbox\\[1.5ex]
\LARGE
From the above discussion,
this procedure application creates a local binding
of the formal parameter \verb't' to the value 5
(the actual parameter expression's value).
This binding is used to extend the environment captured
by the procedure,
and this extended environment is used
to evaluate the body of the procedure.
The value of this application is 8.\exx
The following page displays the environment created by this application,
marked with an asterisk `\verb'*''.
\end{minipage}
