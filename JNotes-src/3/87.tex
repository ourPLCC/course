\begin{minipage}[t]{\sw}
\slidenumber
\LARGE
{\bf Language V5} (continued)\exx
\emm\large\VerbBox{\colorbox{light}}{%
\begin{tabular}{@{}ll}
\verb'<exp>:LetrecExp' & \verb'::= LETREC <letrecDecls> IN <exp>'\\
  & \VerbBox{\fbox}{\verb'LetrecExp(LetrecDecls letrecDecls, Exp exp)'}\\
\verb'<letrecDecls>' & \verb'**= <VAR> EQUALS <proc>'\\
  & \VerbBox{\fbox}{\verb'LetrecDecls(List<Token> varList, List<Proc> procList)'}\\
\end{tabular}%
}\exx
\LARGE
We can now evaluate a \verb'LetrecExp' object
in exactly the same way as a \verb'LetExp' object:
\Large
\begin{qv}
LetrecExp
%%%
    public Val eval(Env env) {
        Env nenv = letrecDecls.addBindings(env);
        return exp.eval(nenv);
    }
%%%
\end{qv}
\LARGE
The principal idea, then,
is to create the RHS closures of a \verb'letrec'
in an environment that (self-referentially) includes all of the bindings
in the \verb'letrec'.
\end{minipage}
