\begin{minipage}[t]{\sw}
\slidenumber
\LARGE
{\bf Language V4} (continued)\exx
\Large
\emm\VerbBox{\colorbox{light}}{\begin{tabular}{@{}ll}
\verb'<exp>:AppExp' & \verb'::= DOT <exp> LPAREN <rands> RPAREN'\\
  & \VerbBox{\fbox}{\verb'AppExp(Exp exp, Rands rands)'}\\
\end{tabular}}\exx
\LARGE
We provided the structure of the fields
in the \verb'ProcVal' class on slide 58,
and we described the semantics of a procedure application on slide 59.
We now proceed to give Java code to implement application semantics.\exx
We start with the \verb'eval' method in the \verb'AppExp' class.
As shown on the next slide, 
this method carries out steps 0 and 1
of procedure application semantics given on slide 59:
it evaluates the \verb'exp' expression --
which should evaluate to a \verb'ProcVal' --
and then it evaluates the operand expressions
to get a list of \verb'Val's.\exx
It then passes these arguments along
to the \verb'apply' method in the \verb'ProcVal' class
to carry out steps 2 and 3 of application semantics.
This method returs the value of the original \verb'AppExp' expression.
(The operand expressions are called
the {\em operands} or {\em actual parameters},
and their corresponding values are called the {\em arguments}.)\exx
You can find the code on the following two slides.
\end{minipage}
