\begin{minipage}[t]{\sw}
\slidenumber
\LARGE
{\bf Language V2} (continued)\exx
{\Large
\VerbBox{\colorbox{light}}{\begin{tabular}{@{}ll}
\verb'<exp>:IfExp' & \verb'::= IF <exp>testExp THEN <exp>trueExp ELSE <exp>falseExp' \\
  & \VerbBox{\fbox}{\verb'IfExp(Exp testExp, Exp trueExp, Exp falseExp)'}\\
\end{tabular}}\exx
}
Here is the \verb'eval' code for the \verb'IfExp' class:
{\Large
\begin{qv}
IfExp
%%%
    public Val eval(Env env) {
        Val v = testExp.eval(env);
        if (v.isTrue())
            return trueExp.eval(env);
        else
            return falseExp.eval(env);
    }
%%%
\end{qv}
}
The \verb'isTrue()' boolean method applies
to any instance of \verb'Val'.
It is a Java helper method used only
to implement the semantics of the \verb'if...then...else' expression;
it is {\em not} part of the source language.
On the other hand, \verb'zero?' is a {\em primitive}
in the source language (starting with Language V1),
not a method in Java.
The \verb'zero?' primitive applies only to integer values
in the source language.
We define the {\em semantics} of the \verb'zero?' primitive
using the \verb'apply' Java method in the \verb'ZeropPrim' class.
You may find this somewhat confusing.
\end{minipage}
\clearpage
