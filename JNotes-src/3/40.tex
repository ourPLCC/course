\begin{minipage}[t]{\sw}
\slidenumber
\LARGE
{\bf Language V2} (continued)\exx
\Large
\VerbBox{\colorbox{light}}{\begin{tabular}{@{}ll}
\verb'<exp>:IfExp' & \verb'::= IF <exp>testExp THEN <exp>trueExp ELSE <exp>falseExp' \\
  & \VerbBox{\fbox}{\verb'IfExp(Exp testExp, Exp trueExp, Exp falseExp)'}\\
\end{tabular}}\exx
\LARGE
In the code for \verb'IfExp',
observe that {\em only one}
of the \verb'trueExp' or \verb'falseExp' expressions
gets evaluated, not both.
This is a semantic feature -- not a syntax feature --
of the definition of \verb'eval' for this object.
The term {\em special form} refers
to semantic structures that look like expressions
but that, when evaluated, don't evaluate all of their constituent parts.
An \verb'if' expression is an example of a special form.\exx
Some examples of \verb'if' expressions are on the next slide.
\end{minipage}
