\begin{minipage}[t]{\sw}
\slidenumber
\LARGE
{\bf Language V1} (continued)\exx
We now have the pieces necessary to define the \verb'eval' method
in the \verb'PrimappExp' class:
\Large
\begin{qv}
PrimappExp
%%%
    public Val eval(Env env) {
        // evaluate the terms in the expression list
        // and apply the prim to the array of Vals
        List<Val> args = rands.evalRands(env);
        Val [] va = Val.toArray(args);
        return prim.apply(va);
    }
%%%
\end{qv}
\LARGE
In summary, to evaluate a primitive application expression (a \verb'PrimappExp'),
evaluate the operands (a \verb'Rands' object) in the given environment
and pass the resulting arguments to the \verb'apply' method
of the primitive (a \verb'Prim') object,
which returns the appropriate value.\exx
What's left is to define the behavior of the \verb'apply' method
in the various \verb'Prim' classes.
Observe that by the time the \verb'Prim'
object gets its arguments,
the environment no longer plays a role,
since the arguments are already evaluated.
\end{minipage}
