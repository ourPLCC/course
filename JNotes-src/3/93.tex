\begin{minipage}[t]{\sw}
\slidenumber
\LARGE
{\bf Language V6} (continued)\exx
Notice that for top-level procedure definitions,
\verb'define' works the same way as \verb'letrec'
in terms of being able to support direct recursion.
This is because every top-level procedure definition captures
(in a closure)
the initial environment,
which gets modified every time another top-level definition is encountered.
When a new binding is added to the top-level environment,
all of the top-level closures can access this binding
(these closures all capture the top-level environment),
as well as any others that may crop up later!
Thus the following works:
\begin{qv}
define even? = proc(x)
  if zero?(x) then 1 else .odd?(sub1(x))
.even?(11) % => Error: unbound procedure odd?
define odd? = proc(x)
  if zero?(x) then 0 else .even?(sub1(x))
.even?(11) % => 0
.odd?(11)  % => 1
\end{qv}
\LARGE
Observe that a top-level define can {\em shadow} a previous definition,
since new bindings are added at the head of the list of top-level bindings
using the \verb'addFirst' method.
A previous definition still appears in the list of bindings,
but because of the way \verb'applyEnv' works,
the one that appears closer to the head of the list will always be returned
when looking up the identifier.
\end{minipage}
